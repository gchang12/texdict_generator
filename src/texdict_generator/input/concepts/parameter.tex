\input macros
\beginconcepts
\concept parameter

The term \defterm{parameter} has two different meanings---it can refer
either to a \TeX\ parameter or to a macro parameter.

A \TeX\ parameter is a \refterm{control sequence} that names
a value.
The value of a parameter can be a \refterm{number}, a \refterm{dimension},
an amount of \refterm{glue} or muglue, or a \refterm{token list}.
For example, the ^|\parindent| parameter
specifies the distance that \TeX\ skips at the start of an
indented  paragraph.

You can use the control sequence for a parameter either to retrieve the value
of the parameter or to set that value.  \TeX\ interprets the control sequence
as a request for a value if it appears in a context where a value is expected,
and as an \refterm{assignment} otherwise.
^^{assignments}
For example:
\csdisplay
\hskip\parindent
|
produces horizontal \refterm{glue} whose natural size is given by |\parindent|,
while:
\csdisplay
\parindent = 2pc  % (or \parindent 2pc)
|
sets |\parindent| to a length of two picas.  The assignment:
\csdisplay
\parindent = 1.5\parindent
|
uses |\parindent| in both ways.  Its effect is to multiply the value of
|\parindent| by $1.5$.

You can think of a parameter as a built-in \refterm{register}.
^^{registers//parameters as}
You'll find a complete list of the \TeX\ parameters on \knuth{pages~272--275}.

A \refterm{macro} parameter is a placeholder for text that is to be
plugged into the definition of a macro.  See \conceptcit{macro}
for more information about this kind of parameter.

\endconcept



\endconcepts
\end