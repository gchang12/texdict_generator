\input macros
\beginconcepts
\concept outer

\bix^^{macros//outer}
An \defterm{outer} macro is one that you can't use in certain contexts
where \TeX\ is processing tokens at high speed.
The purpose of making a command outer is to enable \TeX\ to catch
errors before it's gone too far.
When you define a macro, you can make it outer with the 
^|\outer| command \ctsref\outer.

You cannot use an outer macro in any of the following contexts:
\ulist\compact
\li within an argument to a macro
\li in the parameter text or replacement text of a definition
\li in the preamble to an alignment
\li in the unexecuted part of a conditional test
\endulist
\noindent
An outer context is a context in which you can use an outer macro,
i.e., it's any context other than the ones just listed.

For example, the following input would be a forbidden use of an
outer macro:
\csdisplay
\leftline{\proclaim Assertion 2. That which is not inner
   is outer.}
|
The |\proclaim| macro (\xref{\@proclaim}) is defined in \plainTeX\
to be outer, but it's being used here in a macro argument to |\leftline|.
\eix^^{macros//outer}

\endconcept


\endconcepts
\end