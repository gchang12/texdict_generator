\input macros
\beginconcepts
\concept dimension

A \defterm{dimension} specifies a distance, that is, a linear measure of
space.  You use dimensions to specify sizes of things, such as the length
of a line.  Printers in English-speaking countries traditionally measure
distance in points and picas, while printers in continental Europe
traditionally measure distance in did\^ot points and ciceros.  You can
use these units or others, such as inches, that may be more
familiar to you.  The font-independent
^{units of measure} that \TeX\ understands are:

\xrdef{dimdefs}
\nobreak\vskip\abovedisplayskip
\halign{\indent\hfil\tt #\qquad&#\hfil\cr
pt&^{point} (72.27 points = 1 inch)\cr
pc&^{pica} (1 pica = 12 points)\cr
bp&big point (72 big points = 1 inch)\cr
in&^{inch}\cr
cm&^{centimeter} (2.54 centimeters = 1 inch)\cr
mm&^{millimeter} (10 millimeters = 1 centimeter)\cr
dd&^{\didotpt} (1157 {\didotpt}s = 1238 points)\cr
cc&^{cicero} (1 cicero = 12 {\didotpt}s)\cr
sp&^{scaled point} (65536 scaled points = 1 point)\cr
}
\vskip\belowdisplayskip

Two additional units of measure are associated with every font: `^|ex|',
a vertical measure usually about the height of the letter `x'
in the font, and `^|em|', a
horizontal measure usually equal to the point size of the font and
about the width of the letter `M' in the font. Finally,
\TeX\ provides three ``infinite'' units of measure: `^|fil|', `^|fill|', and
`^|filll|', in increasing order of~strength.

A dimension is written as a ^{factor}, i.e, a multiplier, 
followed by a unit of measure.
^^{units of measure}
The factor can be either a whole \refterm{number} or
a \refterm{decimal constant} containing a decimal point
or decimal comma.  
The factor can be preceded by a plus or minus sign, so a dimension
can be positive or negative.
^^{dimensions//negative}
The unit of measure must be there, even if the number is
zero.  Spaces between the number and the unit of measure are permitted
but not required.  You'll find a precise definition of a
dimension on \knuth{page~270}.  Here are some examples of dimensions:

\csdisplay
5.9in    0pt    -2,5 pc    2fil
|
The last of these represents a first-order infinite distance.

An infinite distance outweighs any finite distance or any weaker infinite
distance.  If you add |10in| to |.001fil|, you get |.001fil|; if you add
|2fil| to |-1fill| you get |-1fill|; and so forth. 
\TeX\ accepts infinite distances
only when you are specifying the \refterm{stretch} and \refterm{shrink}
of \refterm{glue}.

\TeX\ multiplies all dimensions in your document by a
\refterm{magnification} factor $f/1000$,
where $f$ is the value of the ^|\mag| parameter.
^^{magnification}
Since the default value of
|\mag| is $1000$, the normal case is that your document is
typeset just as specified.  You can specify a dimension as it will be
measured in the final document independent of magnification by putting
`|true|' in front of the unit.  For instance, `|\kern 8 true pt|'
produces a kern of $8$ points whatever the magnification.
\endconcept


\endconcepts
\end