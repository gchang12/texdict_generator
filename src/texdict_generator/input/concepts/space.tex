\input macros
\beginconcepts
\concept space

You can cause \TeX\ to put \defterm{space} between two items in 
several~ways: 

\olist
^^{end of line}
\li You can write something that \TeX\ treats as a space
\refterm{token}: one or more blank characters, the end of a line (the
end-of-line character acts like a space), or any \refterm{command} that
expands into a space token.  \TeX\ generally treats several consecutive
spaces as equivalent to a single one, including the case where the
spaces include a single end-of-line.  (An empty line
indicates the end of a paragraph; it
causes \TeX\ to generate a |\par| token.)
^^|\par//from empty line|
\TeX\ adjusts the size of
this kind of space to suit the length required by the context.

^^{glue//creating space with}
\li You can write a skip command that produces the glue
you specify in the command.  The glue can
\refterm{stretch} or \refterm{shrink},
producing more or less space.  You can have vertical glue as
well as horizontal glue.  Glue disappears whenever it is next to a
line or page break.

^^{kerns//creating space with}
\li You can write a \refterm{kern}.  A kern produces a fixed amount of
space that does not stretch or shrink and does not disappear at a line
or page break (unless it is immediately followed by glue).  The most
common use of a kern is to establish a fixed spatial relationship
between two adjacent \refterm{boxes}.
\endolist

Glue and kerns can have negative values.  Negative glue or a negative kern
between adjacent items brings those items closer together.
\endconcept


\endconcepts
\end