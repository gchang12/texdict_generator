\input macros
\beginconcepts
\concept {conditional test}

A \defterm{conditional test} is a command that tests whether or not a certain
condition is true and 
causes \TeX\ either to expand or to skip some text, depending on
the outcome.
The general form of a conditional test is either:
\display{
{\tt \\if}$\alpha$\<true text>{\tt \\else}\<false text>{\tt \\fi}}
^^|\else|^^|\fi|
\noindent or:\hfil\
\display{
{\tt \\if}$\alpha$\<true text>{\tt \\fi}}
\noindent where $\alpha$ specifies the particular test.
For example, |\ifvmode| tests the condition that \TeX\
is currently in a \refterm{vertical mode}.
If the condition is true, \TeX\ expands \<true text>.
If the condition is false, \TeX\ expands \<false text> (if it's present).
Conditional tests are interpreted in \TeX's gullet
\seeconcept{\anatomy}, so any expandable \minref{token}s in
the interpreted text are expanded after the test has been resolved.
The
various conditional tests are explained in \headcit{Conditional tests}%
{conds}.

\endconcept



\endconcepts
\end