\input macros
\beginconcepts
\concept magnification

When \TeX\ typesets your document, it multiplies all dimensions 
by a 
\refterm{magnification} factor $f/1000$,
where $f$ is the value of the ^|\mag| parameter \ctsref\mag.
Since the default value of |\mag| is $1000$, the normal case is that
your document is typeset just as specified.
Increasing the magnification is often useful when you're typesetting a document
that will later be photoreduced.

You can also apply magnification to a single \refterm{font} so as to get
a smaller or larger version of that font than its ``^{design size}''.  You
need to provide the device driver with a ^{shape file}
\seeconcept{font} for 
each magnification of a font that you're using---%
unless the fonts are built into your printer and your device driver
knows about them.
When you're defining a font with
the |\font| command \ctsref{\font}, you can specify a magnification with
the word `|scaled|'.  For example:

\csdisplay
\font\largerbold = cmbx10 scaled 2000
|
defines `|\largerbold|' as a font that is
twice as big as |cmbx10| (Computer Modern
Bold Extended $10$-point) and has the character shapes
uniformly enlarged by a factor of~$2$.

Many computer centers find it convenient to provide fonts scaled by a ratio
of $1.2$, corresponding to magnification values of $1200$, $1440$, etc.  
\TeX\ has special names for these values:
^^|\magstep|
`|\magstep1|' for $1200$,
`|\magstep2|' for $1440$, and so forth up to `|\magstep5|'.  The special
value `^|\magstephalf|' corresponds to magnification by $\sqrt{1.2}$, which
is visually halfway between `|\magstep0|' (no magnification) and
`|\magstep1|'.  For example:

\csdisplay
\font\bigbold = cmbx10 scaled \magstephalf
|

You can specify a \refterm{dimension} as it will be
measured in the final document independent of magnification by putting
`^|true|' in front of the unit.  For instance, `|\kern 8 true pt|' 
produces a kern of $8$ points whatever the magnification. 

\endconcept

\endconcepts
\end