\input macros
\beginconcepts
\concept page

\TeX\ processes a document by assembling \defterm{pages} one at a time
and passing them to the output routine.
As it proceeds through your document, \TeX\ maintains a list of lines
and other 
items to be placed on the page.  (The lines are actually hboxes.)
This list is called the ``^{main vertical list}''. 
Periodically \TeX\ goes through a process called ``exercising the
^{page builder}''.
The items added to the main vertical list between exercises of the
page builder are called ``^{recent contributions}''.

The page builder first examines the main vertical list to see if it's
necessary to ship out a page yet, either because the items on the main
vertical list won't all fit on the page or because of an explicit item,
such as |\eject| \ctsref\eject, that tells \TeX\ to end the page.
If it's not necessary to ship out a page, then the page builder is done
for the time being.

Otherwise the page builder analyzes the main vertical
list to find what it considers to be the best possible page break.
It  associates penalties with various kinds of unattractive page
breaks---a break that would leave an
isolated line at the top or bottom of a page, a break just before a
math display, and so forth.  It then
chooses the least costly page break,
where the cost of a break is increased by any penalty associated with that
break and by the badness of the page that would result
(see \knuth{page~111} for the cost formula).  If it finds several
equally costly page breaks, it chooses the last one.

{\tighten
Once the page builder has chosen a page break,
it places the items on the list that are before that break
into ^|\box255| and leaves the remaining ones for the next page.
It then calls the output routine. |\box255| acts as a mailbox, with the
page builder as the sender and the output routine as the receiver.
Ordinarily the output routine processes |\box255|, adds
other items, such as insertions, headers, and footers, to the page, and
ships out the page to the \dvifile\
^^{\dvifile//material from output routine}
with a |\shipout| command.
(Specialized output routines may behave differently.)
From \TeX's standpoint, it doesn't matter whether or not the output
routine ships out a page;
the only
responsibility of the output routine is to process |\box255| one way or
another.
\par}

{\tighten
It's important to realize that the best place to break a page isn't
necessarily the last possible place to break the page.
Penalties and other considerations may cause the page break
to come earlier.
Furthermore, \TeX\ appends items to the main vertical list in batches,
not just singly.
The lines of a paragraph are an example of such a batch.
For these reasons the page builder usually has items left over when it
breaks a page.
These leftover items then form the beginning of the main vertical list
for the next page (possibly in the middle of a batch).
Because items are carried over from one page to another,
you can't assume that as \TeX\ is processing
input, the current page number accurately reflects the page on which the
corresponding output will appear.  See \knuth{pages~110--114} for a full
description of \TeX's page-breaking rules.
\par}

\endconcept



\endconcepts
\end