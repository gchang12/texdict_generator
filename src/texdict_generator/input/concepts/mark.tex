\input macros
\beginconcepts
\concept mark

A \defterm{mark} is an item that you can insert into a
horizontal, vertical, or math list and later recover from within your
output routine.  Marks are useful for purposes such as
keeping track of topics to appear in page headers.
Each mark has a list of tokens---the ``^{mark text}''---associated with it.
The ^|\mark| command \ctsref{\mark} expects such a token list as its argument,
and appends an item containing that token list (after
expansion) to whatever list \TeX\ is
currently building.  The ^|\topmark|, ^|\firstmark|, and ^|\botmark| commands
\ctsref{\topmark} can be used to retrieve various marks on a page.
These commands are most often used in page headers and footers.
^^{footers//marks used in}
^^{headers//marks used in}

\margin{This example of {\tt\\mark} replaces the previous explanatory
paragraph.}
Here is a simplified example.
Suppose you define a section heading macro as follows:
\csdisplay
\def\section#1{\medskip{\bf#1}\smallskip\mark{#1}}
% #1 is the name of the section
|
^^|\mark|
This macro, when called, will produce a section heading in boldface and
will also record the name of the section as a mark.
You can now define the header for each printed page
as follows:
\csdisplay
\headline = {\ifodd\pageno \hfil\botmark\quad\folio
   \else \folio\quad\firstmark\hfil \fi}
|
Each even (left-hand) page will now have the page number followed by the
name of the first section on that page, while each odd (right-hand) page
will have the page number followed by the name of the last section on
that page.  Special cases, e.g., no sections starting on a page, will
generally come out correctly because of how ^|\firstmark|
and ^|\botmark| work.

When you split a page using the |\vsplit| command \ctsref{\vsplit} you can
retrieve the mark texts of the first and last marks of the split-off
portion with the ^|\splitfirstmark| and ^|\splitbotmark| commands
\ctsref{\splitfirstmark}.

See \knuth{pages~258--260} for a more precise explanation of how
to create and retrieve marks.
\endconcept


\endconcepts
\end