\input macros
\beginconcepts
\concept leaders

You can use \defterm{leaders} to fill a space with copies of
a pattern, e.g.,
to put repeated dots between a title and a page number in a table of contents.
A leader is a single copy of the pattern.
The specification of leaders contains three pieces of information:
\olist\compact
\li what a single leader is
\li how much space needs to be filled
\li how the copies of the pattern should be arranged within the space
\endolist

\bix^^|\leaders|
\bix^^|\cleaders|
\bix^^|\xleaders|
{\tighten
\TeX\ has three commands for specifying leaders:
|\leaders|, \hbox{|\cleaders|}, and |\xleaders| (\xref\leaders).  The 
\refterm{argument} of each command specifies the leader.
The command must be followed by \refterm{glue}; the size of the glue
specifies how much space is to be filled.  The choice of command determines
how the leaders are arranged within the space.}

Here's an example showing how |\leaders| works:
\csdisplay
\def\dotting{\leaders\hbox to 1em{\hfil.\hfil}\hfil}
\line{The Political Process\dotting 18}
\line{Bail Bonds\dotting 26}
|
Here we've put the leaders and their associated glue into a \refterm{macro}
definition so that we can conveniently
use them in two places. This input produces:

\vdisplay{\advance\hsize by -\parindent
\def\dotting{\leaders\hbox to 1em{\hfil{.}\hfil}\hfill}%
\line{The Political Process{\dotting}18}
\line{Bail Bonds{\dotting}26}
}

The \refterm{hbox} following |\leaders| specifies 
the leader, namely, an hbox 1\em\ wide containing a dot
at its center.
The space is filled with copies of this box,
effectively filling it
with dots whose centers are 1\em{} apart.
The following |\hfil| (the one at the
end of the macro definition) is glue that 
specifies the space to be filled.
In this case it's whatever space is needed to fill out the line.
By choosing |\leaders| rather than |\cleaders| or |\xleaders| we've insured
that the dots on different lines line up with each other.

In general, the space to be filled acts as a window
on the repeated copies of the leader.
\TeX\ inserts as many copies as possible, but some space is
usually left over---either because of where the leaders fall
within the window or because
the width of the window isn't an exact multiple of the width of the
leader.
The difference among the three  commands is in how they arrange the leaders
within the window and how they distribute any leftover space:

\ulist
\li For |\leaders|, \TeX\ first produces a row of copies of the leader.
It then aligns the start of this row with the left end of the innermost 
box $B$ that is to contain the result of the |\leaders| command.
In the two-line example above, $B$ is a box produced by |\line|.
Those leaders that fit entirely in the window are placed into $B$,
and the leftover space at the left and right ends is left empty.
The picture is like this:
\vdisplay{%
\def\dotting{\leaders\hbox to 1em{\hfil{.}\hfil}\hfill}%
\def\pp{The Political Process}
\line{\dotting}
\line{\hphantom\pp\hfil$\Downarrow$\hfil\hphantom{18}}
\vskip 4pt
\setbox0 = \hbox{\pp}
\setbox1 = \hbox{18}
\dimen0 = \hsize \advance\dimen0 by -\wd0 \advance \dimen0 by -\wd1
\advance\dimen0 by -0.8pt
\hbadness=10000
\line{\pp
   \vrule\vbox{\hrule width \dimen0\vskip 2pt
   \hbox to \dimen0{\hfil window\strut\hfil}
   \vskip 2pt\hrule width \dimen0}%
   \vrule 18}
\line{\hphantom\pp\hfil$\Downarrow$\hfil\hphantom{18}}
\vskip 2pt
\line{\pp{\dotting}18}
}
\vskip\medskipamount
{\tighten
\noindent
This procedure ensures that in the two-line example on the previous page,
the dots in the two lines
are vertically aligned (since the \refterm{reference points:reference point}
of the hboxes produced by |\line| are vertically aligned).
\par}

\li For |\cleaders|, \TeX\ centers the leaders within the window
by dividing the leftover space between the two ends of the window.
The leftover space is always less than the width of a single leader.

\li For |\xleaders|, \TeX\ distributes the
leftover space evenly within the window.
In other words, if the leftover space is $w$ and the
leader is repeated  $n$ times,
\TeX\ puts space of width $w/(n+1)$ between adjacent leaders and
at the two ends of the leaders.
The effect is usually to spread out the leaders a little bit.
The leftover space for |\xleaders|, like that for |\cleaders|,
is always less than the width of a single leader.
\endulist

So far we've assumed that the leaders consist of hboxes arranged
horizontally.  Two variations are possible:
\olist
\li You can use a
rule instead of an hbox for the leader.
\TeX\ makes the rule as wide as necessary to extend
across the glue (and the three commands are equivalent).
\li You
can produce vertical leaders that run down the page by including them in
a \refterm{vertical list} rather than a \refterm{horizontal list}. In
this case you need vertical glue following the leaders.
\endolist
\noindent
See \knuth{pages~223--225} for the precise rules that \TeX\ uses
in typesetting leaders.
\eix^^|\leaders|
\eix^^|\cleaders|
\eix^^|\xleaders|
\endconcept



\endconcepts
\end