\input macros
\beginconcepts
\concept {style} 

Material in a math formula is set in one of eight \defterm{styles},
depending on the context.  Knowing about styles can be useful if you want to
set part of a formula in a different size of type than the one
that \TeX\ has chosen according to its usual rules.

\eject
The four primary styles are:

\vdisplay{%
\halign{\refterm{# style}\hfil&\hskip .25in(for #)\hfil\cr
display&formulas displayed on a line by themselves\cr
text&formulas embedded in ordinary text\cr
script&superscripts and subscripts\cr
scriptscript&superscripts on superscripts, etc.\cr
}}

The other four styles are so-called ^{cramped variants}.  In these
variants superscripts aren't raised as high as usual, and so the formula
needs less vertical space than it otherwise would.  See
\knuth{pages~140--141} for the details of how \TeX\ selects the style.

\TeX\ chooses a size of type according to the style:

\ulist\compact
^^{display style}^^{text style}
\li Display style and text style are set in \refterm{text size}, like
`$\rm this$'.

^^{script style}
\li Script style is set in \refterm{script size}, like `$\scriptstyle
\rm this$'.

^^{scriptscript style}
\li Scriptscript style is set in \refterm{scriptscript size}, like
`$\scriptscriptstyle \rm this$'.
\endulist

See \conceptcit{family} for more information about these three sizes.

\TeX\ doesn't have a ``scriptscriptscript'' style because such a style
would usually have to be set in a size of type too small to read.  \TeX\
therefore sets third-order subscripts, superscripts, etc., using the
scriptscript style.

Once in a while you may find that \TeX\ has set a formula in a different style
than the one you'd prefer.  You can override \TeX's choice with the
^|\textstyle|, ^|\displaystyle|, ^|\scriptstyle|, and ^|\scriptscriptstyle|
commands \ctsref{\textstyle}.
\endconcept

\endconcepts
\end