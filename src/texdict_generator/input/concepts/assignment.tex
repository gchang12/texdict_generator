\input macros
\beginconcepts
\concept assignment

An \defterm{assignment} is a construct that tells \TeX\ to assign a
value to a register,
^^{registers//assignment to}
to one of its internal
\refterm{parameters:parameter},
^^{parameters//assignments to}
to an entry in one of its internal tables,
or to a \refterm{control sequence}. Some examples of assignments are:

\csdisplay
\tolerance = 2000
\advance\count12 by 17
\lineskip = 4pt plus 2pt
\everycr = {\hskip 3pt \relax}
\catcode\`@ = 11
\let\graf = \par
\font\myfont = cmbx12
|

The first assignment indicates that \TeX\ should assign the numeric value
|2000| to the numeric parameter |\tolerance|, i.e., make the value of
|\tolerance| be $2000$.  The other assignments are similar.  The `|=|'
and the spaces are optional, so you could also write the first
assignment more tersely as:

\csdisplay
\tolerance2000
|

See \knuth{pages~276--277} for the detailed syntax of assignments.
\endconcept


\endconcepts
\end