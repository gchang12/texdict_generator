\input macros
\begindescriptions
\begindesc
\bix^^{page numbering}
\cts pageno {\param{number}}
\explain
This parameter contains the current page number as an integer.  The page
number is normally negative for front-matter pages that are numbered
with small roman numerals instead of arabic numerals.  If you change the
page number within a page,
the changed number will be used in any headers or footers that
appear on that page.
The actual
printing of page numbers is handled by \TeX's \minref{output routine},
which you can modify.

\PlainTeX\ keeps the page number in the \minref{register} ^|\count0|.
(|\pageno| is, in fact, a synonym for |\count0|.)  
Whenever it ships out a page to the \dvifile,
^^|\shipout//{\tt\\count} registers displayed at|
\TeX\ displays the current value of |\count0| on your
terminal so that you can tell which page it is working on.
It's possible to use registers |\count1|--|\count9| for nested
levels of page numbers (you must program this yourself).
If any of these registers are nonzero, \TeX\ displays them on your
terminal also.\footnote{
More precisely, it displays all registers in sequence from
|\count0| to |\count9|, but omits trailing zero registers.
For instance, if the values of |\count0|--|\count3|
are $(17, 0 , 0, 7)$ and the others are $0$,
\TeX\ displays the page number as {\tt [17.0.0.7]}.}
\example
This explanation appears on page \number\pageno\
of our book.
|
\produces
This explanation appears on page \number\pageno\
of our book.

\nextexample
\pageno = 30 % Number the next page as 30.
Don't look for this explanation on page \number\pageno.
|
\produces
Don't look for this explanation on page 30.
\endexample
\enddesc
\enddescriptions
\end