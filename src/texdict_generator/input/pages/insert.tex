\input macros
\begindescriptions
\begindesc
\cts insert {\<number> \rqbraces{\<vertical mode material>}}
\explain
\minrefs{insertion}
This primitive command provides the 
underlying mechanism for constructing insertions, but
it is hardly ever used outside of a \minref{macro} definition.
The definitions of the
|\foot!-note|, |\vfoot!-note|,
|\top!-insert|, |\mid!-insert|, and |\page!-insert| commands are 
all built around |\insert|.
^^|\topinsert|
^^|\midinsert| ^^|\pageinsert| ^^|\footnote| ^^|\vfootnote|

When you design insertions for a document, you should assign a different
integer code\footnote
{\texbook{} uses the term ``class'' for a code.
We use a different term to avoid confusion with the other meaning of
``class'' (\xref{class}).}
$n$ to each kind of insertion,
using the ^|\newinsert| command (\xref{\@newinsert}) to obtain the
integer codes.
The |\insert| command itself appends the \<vertical mode material>
to the current horizontal or \minref{vertical list}.
Your \minref{output routine} is responsible for
moving the inserted material from where it resides in |\box|$\,n$
to an output page.
^^{output routine}

\TeX\ groups together all insertions having the same code
number.  Each insertion
code $n$ has four \minref{register}s associated with it:
\ulist
\li |\box|$\,n$ is where \TeX\ accumulates the material for insertions
with code $n$.  When \TeX\ breaks a page, it puts into |\box|$\,n$
as much insertion $n$ material as will fit on the page.
Your output routine should then move this material to the actual page.
You can use ^|\ifvoid| \ctsref{\@ifvoid}
to test if there is any material in |\box|$\,n$.
If not all the material fits, \TeX\ saves the leftovers for the next page.
\li |\count|$\,n$ is a magnification factor $f.$  When \TeX\ is computing
the vertical space occupied on the page
by insertion $n$ material, it multiplies the
vertical extent of this material by $f/1000$.
Thus you would ordinarily set $f$ to $500$ for a double-column insertion
and to $0$ for a marginal~note.
\li |\dimen|$\,n$ specifies the maximum amount
of insertion $n$ material that \TeX\ will put on a single page.
\li |\skip|$\,n$ specifies extra space that \TeX\ allocates on the page
if the page contains any insertion $n$ material.
This space is in addition to the space occupied by the insertion itself.
For example, it would account for the space on a page above the footnotes
(if there are~any).
\endulist
\noindent
\TeX\ sets |\box|$\,n$, and you should set the other three registers
so that \TeX\ can correctly compute the vertical space required by the
insertion.
See \knuth{pages~122--125} for further details of how \TeX\ processes this
command and of how insertions interact with page breaking.
\xrdef{endofinsert}
\enddesc
\enddescriptions
\end