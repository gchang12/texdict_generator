\input macros
\begindescriptions
\begindesc
\cts vsplit {\<number> {\bt to} \<dimen>}
\explain
This command causes \TeX\ to split the \minref{box} numbered
\<number>, which we'll call $B_2$, into two parts.
It uses the same algorithm that it would use if $B_2$ was a page
and it was breaking that page;
the division point then corresponds to the page break that it would find.
The box $B_2$ must be a vbox, not an hbox.
% we avoid starting the previous sentence with a symbol, a copyediting no-no.
\TeX\ puts the material preceding the division point into
another box $B_1$ and leaves the material after the division point in $B_2$.
The |\vsplit| command then produces $B_1$.
Normally you'd assign $B_1$ to a different
box register, as in the example below.
If the division point is at the end of $B_2$,
$B_2$ will be empty after the |\vsplit|.

\TeX\ employs its usual page-breaking algorithm
^^{page breaks//in split lists}
for the split.
It uses \<dimen> for ^|\pagegoal|, the desired height of $B_1$.
The vertical extent of $B_1$ may not be exactly
\<dimen> because \TeX\ may not be able to achieve its page goal perfectly.
\TeX\ does not consider insertions in calculating the split,
so insertions in the original vertical list of $B_2$ will be retained
but won't affect the split point.

\example
\setbox 20 = \vsplit 30 to 7in
% Split off the first seven inches or so of material from
% box 30 and place that material in box 20.
|
\endexample
\enddesc
\enddescriptions
\end