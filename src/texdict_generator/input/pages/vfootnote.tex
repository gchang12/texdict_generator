\input macros
\begindescriptions
\begindesc
\easy\cts footnote {\<argument$_1$> \<argument$_2$>}
\cts vfootnote {\<argument$_1$> \<argument$_2$>}
\explain
These commands produce footnotes.
\<argument$_1$> is the ``^{reference mark}''
for the footnote and \<argument$_2$> is its text.
The text can be several paragraphs long if necessary and can contain
constructs such as math displays, but it shouldn't contain any 
\minref{insertion}s (such as other footnotes).

You shouldn't use these commands inside a subformula of a math formula,
in a box within a box being contributed to a page,
or in an insertion of any kind. 
If you're unsure whether these restrictions apply, you can be safe by
only using |\footnote| and |\vfootnote| directly within a paragraph or
between paragraphs.

These restrictions aren't as severe as they seem because you can use
|\vfootnote| to footnote most anything.
Both |\foot!-note| and |\vfoot!-note| insert the reference mark in front of the
footnote itself, but |\vfoot!-note| doesn't insert the reference mark into the
text.
Thus, when you use |\vfoot!-note| you can
explicitly insert the reference mark
wherever it belongs without concern about the context
and place the |\vfootnote| in the next paragraph.
If you find that the footnote lands on
the page following the one where it belongs, move the |\vfootnote| back
to the previous paragraph.
There are rare circumstances where you'll need to
alter the text of your document in order to get a footnote to appear on
the same page as its reference mark.
\example
To quote the mathematician P\'olya is a ploy.\footnote
*{This is an example of an anagram, but not a strict one.}
|
\produces
To quote the mathematician P\'olya is a ploy.*
\par\line{\hskip .5in \vdots\hfil}
\nointerlineskip \bigskip
\footnoterule\par\parindent = 12pt
\textindent{*}This is
an example of an anagram, but not a strict one.
\endexample
\example
$$f(t)=\sigma\sigma t\;\raise 1ex \hbox{\dag}$$
\vfootnote \dag{The $\sigma\sigma$ notation was explained in
the previous section.}
|
\produces
$$f(t)=\sigma\sigma t\>\raise 1ex \hbox{\dag}$$
\par\line{\hskip .5in \vdots\hfil}
\nointerlineskip \bigskip
\footnoterule\par\parindent = 12pt
\textindent{\dag}{The $\sigma\sigma$ notation was explained in
the previous section.}
\endexample
\enddesc
\enddescriptions
\end