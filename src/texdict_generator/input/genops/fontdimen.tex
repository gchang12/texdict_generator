\input macros
\begindescriptions
\begindesc
\cts fontdimen {\<number> \<font>\param{dimen}}
\explain
^^{fonts//parameters of}
These parameters specify various dimensions associated with
the font named by the control sequence \<font>
(as distinguished from the \<font\-name> that names the font files).
Values of these parameters are specified in the metrics
file for \<font>, but you can
retrieve or change their values during a \TeX\ run.
The numbers and meanings of the parameters are:
\display{\halign{\hfil#\hfil\quad&#\hfil\cr
\it Number&\it Meaning\cr
\noalign{\vskip 1\jot}%
1&slant per point\cr
2&interword space\cr
3&interword stretch\cr
4&interword shrink\cr
5&x-height (size of |1ex|)\cr
6&quad width (size of |1em|)\cr
7&extra space\cr}}
\noindent
\TeX\ uses the slant per point for positioning accents.
It uses the interword parameters for producing interword spaces
(see |\spaceskip|, \xref\spaceskip) and the extra space parameter
for the additional space after a period (see |\xspaceskip|,
\xref\xspaceskip).
The values of these parameters for the
\plainTeX\ fonts are enumerated on \knuth{page~433}.
Math symbol fonts have $15$ additional parameters, which we won't discuss here.

Beware: 
assignments to these parameters are \emph{not} undone at the end
of a group.
If you want to change these parameters locally, you'll need to
save and restore their original settings explicitly.
\example
Here's a line printed normally.\par
\dimen0=\fontdimen2\font
\fontdimen2\font=3\fontdimen2\font % triple word spacing
\noindent Here's a really spaced-out line.
\fontdimen2\font=\dimen0
|
\produces
Here's a line printed normally.\par
\dimen0=\fontdimen2\font
\fontdimen2\font=3\fontdimen2\font % triple word spacing
\noindent Here's a really spaced-out line.
\fontdimen2\font=\dimen0
\endexample
\enddesc
\enddescriptions
\end