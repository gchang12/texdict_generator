\input macros
\begindescriptions
\begindesc
\cts tracingpages {\param{number}}
\explain
If this parameter is greater than zero,
\TeX\ will record in the log file its calculations of the cost of
various page breaks that it tries.
^^{page breaks//tracing}
If |\tracing!-online| ^^|\tracingonline|
is greater than zero, this information will also appear
at your terminal.
\TeX\ produces a line of this output
whenever it first places a box or \minref{insertion}
on the current page list, and also whenever it processes a potential
break point for the page.
Examining this output can be helpful when you're trying to determine
the cause of a bad page break.
See \knuth{pages~112--114} for an illustration and explanation of
this output.

Some production forms of \TeX\ ignore the value of |\tracingpages|
so that they can run faster.
If you need to use this parameter, be sure to use a form that
responds to it.
\enddesc
\enddescriptions
\end