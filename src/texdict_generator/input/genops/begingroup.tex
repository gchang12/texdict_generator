\input macros
\begindescriptions
\begindesc
\bix^^{groups}
%
\cts begingroup {}
\cts endgroup {}
\explain
These two commands begin and end a \minref{group}.
A |\begingroup| does not match up with a right brace,
nor an |\endgroup| with a left brace.

\TeX\ treats |\begingroup| and |\endgroup| like any other
\minref{control sequence} when it's scanning its input.  In particular,
you can define a \minref{macro} that contains a |\begingroup|
but not an |\endgroup|, and conversely.
^^{macros//using \b\tt\\begingroup\e\ and \b\tt\\endgroup\e\ in}
This technique is often useful
when you're defining paired macros, one of which establishes
an environment and the other of which terminates that environment.
You can't, however, use |\begingroup| and |\endgroup| as substitutes for
braces other than the ones that surround a group.
\example
\def\a{One \begingroup \it two }
\def\enda{\endgroup four}
\a three \enda
|
\produces
\def\a{One \begingroup \it two }
\def\enda{\endgroup four}
\a three \enda
\endexample
\enddesc
\enddescriptions
\end