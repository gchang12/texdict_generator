\input macros
\begindescriptions
\begindesc
\bix^^{files}
\bix^^{input files}
\easy\cts input {\<filename>}
\explain
\minrefs{file}\minrefs{file name}
This command tells \TeX\ to read its input from file \<filename>.
When that file is exhausted, \TeX\ returns to reading from its previous
input source.  You can nest input files to any level you like
(within reason).

When you're typesetting a large document, it's usually a good idea to
structure your main file as a sequence of |\input| commands that refer
to the subsidiary parts of the document.  That way you can process the
individual parts easily as you're working on drafts.  It's also a good
practice to put all of your \minref{macro} definitions into a separate file and
summon that file with an |\input| command as the first action in your
main file.

\TeX\ uses different rules for scanning file names than it does for scanning
\minref{token}s in general (see \xref{file name}).
If your implementation expects file names to have extensions (usually
indicated by a preceding dot), then \TeX\ provides a default extension
of |.tex|.
\example
\input macros.tex
\input chap1 % equivalent to chap1.tex
|
\endexample
\enddesc
\enddescriptions
\end