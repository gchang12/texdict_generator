\input macros
\begindescriptions
\begindesc
secondprinting{\vglue-.5\baselineskip\vskip0pt}
\cts read {\<number> {\bt to} \<control sequence>}
\explain
^^{input streams//reading with \b\tt\\read\e}
^^{reading a file}
This command tells \TeX\ to read a line from the file
associated with the \minref{input stream}
designated by \<number> and assign the tokens on that line to
\<control sequence>.  The \minref{control sequence} then becomes a 
parameterless \minref{macro}.  No macro expansion takes place
during the reading operation.  If the line contains any unmatched
left braces, \TeX\ will read additional lines until the braces are
all matched.  If \TeX\ reaches the end of the file without matching all the
braces, it will complain.

If \<number> is greater than $15$ or hasn't been associated with a file
using ^|\openin|, \TeX\ prompts you with `\<control sequence> |=|'
on your terminal and waits for you to type a line of input.
It then assigns the input line to \<control sequence>.
If \<number> is less than zero, it reads a line of input from your
terminal but omits the prompt.
\example
\read\auxfile to \holder
% Expanding \holder will produce the line just read.
|
\endexample
\eix^^{input files}
\enddesc
\enddescriptions
\end