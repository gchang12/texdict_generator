\input macros
\begindescriptions
\begindesc
\bix^^{registers//reserving}
\makecolumns 11/2:
\ctspecial newcount \ctsxrdef{@newcount}
\ctspecial newdimen \ctsxrdef{@newdimen}
\ctspecial newskip \ctsxrdef{@newskip}
\ctspecial newmuskip \ctsxrdef{@newmuskip}
\ctspecial newtoks \ctsxrdef{@newtoks}
\ctspecial newbox \ctsxrdef{@newbox}
\ctspecial newread \ctsxrdef{@newread}
\ctspecial newwrite \ctsxrdef{@newwrite}
\ctspecial newfam \ctsxrdef{@newfam}
\ctspecial newinsert \ctsxrdef{@newinsert}
\ctspecial newlanguage \ctsxrdef{@newlanguage}
\explain
These commands
reserve and name an entity of the indicated type:
\ulist
\li |\newcount|, |\newdimen|, |\newskip|, |\newmuskip|, |\newtoks|,
|\newbox| each reserve a \minref{register} of the indicated type.
^^{count registers//reserved by \b\tt\\newcount\e}
^^{dimension registers//reserved by \b\tt\\newdimen\e}
^^{skip registers//reserved by \b\tt\\newskip\e}
^^{muskip registers//reserved by \b\tt\\newmuskip\e}
^^{token registers//reserved by \b\tt\\newtoks\e}
^^{box registers//reserved by \b\tt\\newbox\e}
\li |\newread| and |\newwrite| reserve an input stream and
an output stream \minrefs{input stream}\minrefs{output stream}
respectively.
^^{input streams//reserved by \b\tt\\newread\e}
^^{output streams//reserved by \b\tt\\newwrite\e}
\li |\newfam| reserves a \minref{family} of math fonts.
^^{family//reserved by \b\tt\\newfam\e}
\li |\newinsert| reserves an insertion type.
(Reserving an insertion type involves reserving several different registers.)
^^{insertions//numbers reserved by \b\tt\\newinsert\e}
\li |\newlanguage| reserves a set of hyphenation patterns.
\endulist
You should use these commands whenever you need one of these entities,
other than in a very local region,
in order to avoid numbering conflicts.

There's an important difference among these commands:
\ulist
\li The control sequences defined by
|\newcount|, |\newdimen|, |\newskip|, |\newmuskip|, and |\newtoks|
each designate an entity of the appropriate type.
For instance, after the command:
\csdisplay
\newdimen\listdimen
|
the control sequence |\listdimen| can be used as a dimension.
\li The control sequences defined by
|\newbox|, |\newread|, |\newwrite|, |\newfam|, |\newinsert|,
and |\newlanguage|  each
evaluate to the \emph{number} of an entity of the appropriate type.
For instance, after the command:
\csdisplay
\newbox\figbox
|
the control sequence |\figbox| must be used in conjunction with
a |\box|-like command, e.g.:
\csdisplay
\setbox\figbox = \vbox{!dots}
|
\endulist
\enddesc
\enddescriptions
\end