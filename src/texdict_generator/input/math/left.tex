\input macros
\begindescriptions
\begindesc
\margin{{\tt\\vert} and {\tt\\Vert} were explained elsewhere.}
\easy\cts left {}
\cts right {}
\explain
These commands must be used together in the pattern:
\display
{{\bt \\left} \<delim$_1$> \<subformula> {\bt \\right} \<delim$_2$>}
This construct causes \TeX\ to produce \<subformula>, 
enclosed in the \minref{delimiter}s \<delim$_1$> and \<delim$_2$>.
The vertical size of the delimiter is adjusted to fit the 
vertical size (height plus depth) of \<subformula>.  \<delim$_1$> and
\<delim$_2$> need not correspond.
For instance, you could use `|]|' as a left delimiter
and `|(|' as a right delimiter in a single use of |\left|
and |\right|.

|\left| and |\right| have the important property that they define a
group, i.e., they act like left and right braces.  This grouping
property is particularly useful when you put ^|\over| (\xref{\over}) or
a related command between |\left| and |\right|, since you don't need to
put braces around the fraction constructed by |\over|.

If you want a left delimiter but not a right delimiter, you can use `|.|' in
place of the delimiter you don't want and it will turn into empty space
(of width ^|\nulldelimiterspace|).
\example
$$\left\Vert\matrix{a&b\cr c&d\cr}\right\Vert
  \qquad \left\uparrow q_1\atop q_2\right.$$
|
\dproduces
$$\left\Vert\matrix{a&b\cr c&d\cr}\right\Vert
  \qquad \left\uparrow q_1\atop q_2\right.$$
\endexample
\enddesc
\enddescriptions
\end