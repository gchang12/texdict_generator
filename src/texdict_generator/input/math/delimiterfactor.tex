\input macros
\begindescriptions
\begindesc

\margin{{\tt\\delcode} was explained in two places.  The
combined explanation is now in `General operations'.}
\cts delimiterfactor {\param{number}}
\cts delimitershortfall {\param{number}}
\explain
^^{delimiters//height of}
These parameters together tell \TeX\ how the height of a \minref{delimiter}
should be related to the vertical size of the subformula
with which the delimiter is associated.
|\delimiterfactor| gives the minimum
ratio of the delimiter size to the vertical size of the subformula, and
|\delimitershortfall| gives the maximum by which the height of the
delimiter will be reduced from that of the vertical size of the subformula.

Suppose that the \minref{box} containing the subformula
has height $h$ and depth $d$, and let $y=2\,\max(h,d)$.
Let the value of |\delimiterfactor| be $f$ and the value of
|\delimitershortfall| be $\delta$.
Then \TeX\ takes the minimum delimiter size to be at least $y \cdot
f/1000$ and at least $y-\delta$.  In particular, if |\delimiterfactor|
is exactly $1000$ then \TeX\ will try to make a delimiter at least as tall
as the formula to which it is attached.
See \knuth{page~152 and page~446 (Rule 19)}
for the exact details of how \TeX\ uses these parameters.
\PlainTeX\ sets |\delimiter!-factor| to $901$ and 
|\delimiter!-shortfall| to |5pt|.
\enddesc
\enddescriptions
\end