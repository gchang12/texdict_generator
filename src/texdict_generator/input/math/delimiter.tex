\input macros
\begindescriptions
\begindesc
\cts delimiter {\<number>}
\explain
This command produces a delimiter whose characteristics are given by
\<number>.  \<number> is normally written in hexadecimal notation.
You can use the |\delimiter| command instead of a character in any context
where \TeX\ expects a delimiter (although the command is rarely used
outside of a macro definition).
Suppose that \<number> is the hexadecimal number $cs_1s_2s_3
l_1l_2l_3$.  Then \TeX\ takes the delimiter to have 
\minref{class} $c$, small variant
$s_1s_2s_3$, and large variant $l_1l_2l_3$.  Here $s_1s_2s_3$ indicates
the math character found in position $s_2s_3$ of family $s_1$, and
similarly for $l_1l_2l_3$.  This is the same convention as the one
used for ^|\mathcode| (\xref\mathcode).
\example
\def\vert{\delimiter "026A30C} % As in plain TeX.
|
\endexample
\enddesc
\enddescriptions
\end