\input macros
\begindescriptions
\begindesc
\cts nolimits {}
\explain
In display style, \TeX\ normally places limits above and below a large
operator.  This command tells \TeX\ to place limits after a large
operator rather than above and below it.

The integral operators |\int| and |\oint| are exceptions---\TeX\ places
limits after them in all cases, unless overridden, as in |\int\limits|.
(\plainTeX\ defines ^|\int| and ^|\oint| as macros that specify the
operator symbol followed by |\nolimits|---this is what causes them to
behave differently by default.)
^^|\int//limits after|

If you specify more than one of |\limits|, |\nolimits|, 
and |\display!-limits|, the last command rules.

\example
$$\bigcap\nolimits_{i=1}^Nq_i$$
|
\dproduces
$$\bigcap\nolimits_{i=1}^Nq_i$$
\endexample
\enddesc
\enddescriptions
\end