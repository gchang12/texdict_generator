\input macros
\begindescriptions
\begindesc
\ctspecial * \ctsxrdef{@star}
\explain
The |\*| command indicates a discretionary multiplication symbol
($\times$), which is a binary operation.  This multiplication symbol
behaves like a discretionary hyphen when it appears in a formula within
text\minrefs{text math}.  That is, \TeX\ will typeset the |\times|
symbol \emph{only} if the formula needs to be broken at that point.
There's no point in using |\*| in a displayed formula \minrefs{display
math} since \TeX\ never breaks displayed formulas on its own.

\example
Let $c = a\*b$. In the case that $c=0$ or $c=1$, let
$\Delta$ be $(\hbox{the smallest $q$})\*(\hbox{the
largest $q$})$ in the set of approximate $\tau$-values.
|
\produces
Let $c = a\*b$. In the case that $c=0$ or $c=1$, let
$\Delta$ be $(\hbox{the smallest $q$})\*(\hbox{the
largest $q$})$ in the set of approximate $\tau$-values.

\eix^^{operations}
\endexample
\enddesc
\enddescriptions
\end