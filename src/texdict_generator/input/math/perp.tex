\input macros
\begindescriptions
\begindesc
\xrdef {relations}
\bix^^{relations}
\dothreecolumns 39
\easy\ctsdisplay asymp {}
\ctsdisplay cong {}
\ctsdisplay dashv {}
\ctsdisplay vdash {}
\ctsdisplay perp {}
\ctsdisplay mid {}
\ctsdisplay parallel {}
\ctsdisplay doteq {}
\ctsdisplay equiv {}
\ctsdisplay ge {}
\ctsdisplay geq {}
\ctsdisplay le {}
\ctsdisplay leq {}
\ctsdisplay gg {}
\ctsdisplay ll {}
\ctsdisplay models {}
\ctsdisplay ne {}
\ctsdisplay neq {}
\ctsdisplay notin {}
\ctsdisplay in {}
\ctsdisplay ni {}
\ctsdisplay owns {}
\ctsdisplay prec {}
\ctsdisplay preceq {}
\ctsdisplay succ {}
\ctsdisplay succeq {}
\ctsdisplay bowtie {}
\ctsdisplay propto {}
\ctsdisplay approx {}
\ctsdisplay sim {}
\ctsdisplay simeq {}
\ctsdisplay frown {}
\ctsdisplay smile {}
\ctsdisplay subset {}
\ctsdisplay subseteq {}
\ctsdisplay supset {}
\ctsdisplay supseteq {}
\ctsdisplay sqsubseteq {}
\ctsdisplay sqsupseteq {}
\egroup
\explain
These commands produce the symbols for various relations.
Relations are one of \TeX's \minref{class}es of math symbols.
\TeX\ puts different amounts of space
around different \minref{class}es of math symbols.
When \TeX\ needs to break a line of text
within a math formula, \minrefs{line break} it will consider
placing the break after a relation---but only if
the relation is at the outermost level of the formula,
i.e., not enclosed in a group.

In addition to the commands listed here, \TeX\ treats  `^|=|' and the
``arrow'' commands (\xref{arrows}) as relations.

Certain relations have more than one command that you can use
to produce them:
\ulist \compact
\li `$\ge$' (|\ge| and |\geq|).
\li `$\le$' (|\le| and |\leq|).
\li `$\ne$' (|\ne|, |\neq|, and |\not=|).
\li `$\ni$' (|\ni| and |\owns|).
\endulist

\xrdef{\not}
You can produce negated relations by prefixing them with |\not|, as follows:

\nobreak
\threecolumns 21
\basicdisplay {$\not\asymp$}{\\not\\asymp}\ctsidxref{asymp}
\basicdisplay {$\not\cong$}{\\not\\cong}\ctsidxref{cong}
\basicdisplay {$\not\equiv$}{\\not\\equiv}\ctsidxref{equiv}
\basicdisplay {$\not=$}{\\not=}\ttidxref{=}
\basicdisplay {$\not\ge$}{\\not\\ge}\ctsidxref{ge}
\basicdisplay {$\not\geq$}{\\not\\geq}\ctsidxref{geq}
\basicdisplay {$\not\le$}{\\not\\le}\ctsidxref{le}
\basicdisplay {$\not\leq$}{\\not\\leq}\ctsidxref{leq}
\basicdisplay {$\not\prec$}{\\not\\prec}\ctsidxref{prec}
\basicdisplay {$\not\preceq$}{\\not\\preceq}\ctsidxref{preceq}
\basicdisplay {$\not\succ$}{\\not\\succ}\ctsidxref{succ}
\basicdisplay {$\not\succeq$}{\\not\\succeq}\ctsidxref{succeq}
\basicdisplay {$\not\approx$}{\\not\\approx}\ctsidxref{approx}
\basicdisplay {$\not\sim$}{\\not\\sim}\ctsidxref{sim}
\basicdisplay {$\not\simeq$}{\\not\\simeq}\ctsidxref{simeq}
\basicdisplay {$\not\subset$}{\\not\\subset}\ctsidxref{subset}
\basicdisplay {$\not\subseteq$}{\\not\\subseteq}\ctsidxref{subseteq}
\basicdisplay {$\not\supset$}{\\not\\supset}\ctsidxref{supset}
\basicdisplay {$\not\supseteq$}{\\not\\supseteq}\ctsidxref{supseteq}
\basicdisplay {$\not\sqsubseteq$}{\\not\\sqsubseteq}%
   \ctsidxref{sqsubseteq}
\basicdisplay {$\not\sqsupseteq$}{\\not\\sqsupseteq}%
   \ctsidxref{sqsupseteq}
\egroup

\example
We can show that $AB \perp AC$, and that
$\triangle ABF \not\sim \triangle ACF$.
|
\produces
We can show that $AB \perp AC$, and that
$\triangle ABF \not\sim \triangle ACF$.

\eix^^{relations}
\endexample
\enddesc
\enddescriptions
\end