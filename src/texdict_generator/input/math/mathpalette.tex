\input macros
\begindescriptions
\begindesc
\cts mathpalette {\<argument$_1$> \<argument$_2$>}
\explain
^^{math symbols}
This command provides a convenient way of 
producing a math construct that works in all four \minref{style}s.
To use it, you'll normally need to define an additional macro,
which we'll call |\build|.
The call on |\math!-palette| should then have the form
|\mathpalette|\allowbreak|\build|\<argument>.

|\build| tests what style \TeX\ is in and typesets \<argu\-ment> accordingly.
It should be defined to have two parameters.
When you call |\math!-palette|, it will in turn call |\build|,
with |#1| being a
command that selects the current style and |#2| being \<argument>.
Thus, within the definition of |\build| you can typeset something
in the current style by preceding it with `|#1|'.
See \knuth{page~360} for examples of using |\mathpalette|
and \knuth{page~151} for a further explanation of how it works. 

\eix^^{styles}
\enddesc
\enddescriptions
\end