\input macros
\begindescriptions
\begindesc
\cts hfilneg {}
\cts vfilneg {}
\explain
^^{glue//negative}
These commands cancel the effect of a preceding |\hfil|
or |\vfil|. While |\hfil| and |\vfil|
produce infinitely stretchable positive \minref{glue}, |\hfilneg| 
and |\vfilneg| produce infinitely stretchable negative glue.
(Thus, $n$ |\hfilneg|s cancel $n$ ^|\hfil|s, and similarly for
|\vfilneg|.)
The main use of |\hfilneg| and |\vfilneg|
is to counteract the effect of an |\hfil| or |\vfil|
inserted by a \minref{macro}.

|\hfilneg| and |\vfilneg| have
the curious property that if they are the only infinitely stretchable
glue in a box, they produce exactly the same effect as |\hfil|
and |\vfil|.

\example
\leftline{\hfil on the right\hfilneg}
% Cancel the \hfil that \leftline produces to the right
% of its argument.
|
\produces
\leftline{\hfil on the right \hfilneg}
% Cancel the \hfil that \leftline produces to the right
% of its argument.
%
\nextexample
\def\a{\hbox to 1pc{\hfil 2}\vfil}
\vbox to 4pc{\hbox{1} \vfil \a
   \vfilneg \hbox to 2pc{\hfil 3}}
|
\produces
\smallskip
\def\a{\hbox to 1pc{\hfil 2}\vfil}
\vbox to 4pc{\hbox{1} \vfil \a
   \vfilneg \hbox to 2pc{\hfil 3}}
\endexample
\enddesc
\enddescriptions
\end