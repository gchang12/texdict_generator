\input macros
\begindescriptions
\begindesc
^^{boxes//last box in a list}
^^{kerns//last kern in a list}
^^{penalties//last penalty in a list}
^^{glue//last glue item in a list}
\margin{This section has been moved in its entirety from the chapter
(`Commands for general operations')}
\cts lastkern {}
\cts lastskip {}
\cts lastpenalty {}
\cts lastbox {}
\explain
These control sequences yield the value of the last item on the current
list.  They aren't true commands because they can only appear as
part of an argument.
If the last item on the list isn't of the indicated type,
they yield a zero value (or an empty box, in the case of |\lastbox|).
For example,
if the last item on the current list is a \minref{kern},
|\lastkern| yields the dimension of that kern; if it
isn't a kern, it yields a dimension of $0$.

Using |\lastbox| has the additional effect of removing the last box from
the list.
If you want the original |\last!-box| to remain on the list, you
have to add a copy of it to the list.
|\last!-box| is not permitted in a math list or in the main
vertical~list.

These control sequences are
most useful after macro calls that might have inserted entities of
the indicated kinds. 

\example
\def\a{two\kern 15pt}
one \a\a\hskip 2\lastkern three\par
% Get three times as much space before `three'.
\def\a{\hbox{two}}
one \a
\setbox0 = \lastbox % Removes `two'.
three \box0.
|
\produces
\def\a{two\kern 15pt}
one \a\a\hskip 2\lastkern three\par
% get three times as much space before `three'
\def\a{\hbox{two}}
one \a
\setbox0 = \lastbox % removes `two'
three \box0.
\endexample
\enddesc
\enddescriptions
\end