\input macros
\begindescriptions
\begindesc
\cts item {\<argument>}
\cts itemitem {\<argument>}
\explain
^^{itemized lists}
These commands are useful for creating ^{itemized lists}.  The entire paragraph
following \<argument> is indented by |\parindent|
^^|\parindent//indentation for itemized lists|
(for |\item|) or by |2\parindent| (for |\itemitem|).
(See \xrefpg{\parindent} for an explanation of |\parindent|.)
Then \<argument>,
followed by an en space, is placed just to
the left of the text of the
first line of the paragraph so that it falls within the paragraph indentation
as specified by |\parindent|.

If you want to include more than one
paragraph in an item, put |\item{}| in front of the additional paragraphs.
\example
{\parindent = 18pt
\noindent Here is what we require:
\item{1.}Three eggs in their shells,
but with the yolks removed.
\item{2.}Two separate glass cups containing:
\itemitem{(a)}One-half cup {\it used} motor oil.
\itemitem{(b)}One cup port wine, preferably French.
\item{3.}Juice and skin of one turnip.}
|
\produces
{\parindent = 18pt
\noindent Here is what we require:
\item{1.}Three eggs in their shells,
but with the yolks removed.
\item{2.}Two separate glass cups containing:
\itemitem{(a)}One-half cup {\it used} motor oil.
\itemitem{(b)}One cup port wine, preferably French.
\item{3.}Juice and skin of one turnip.}
\endexample
\enddesc
\enddescriptions
\end