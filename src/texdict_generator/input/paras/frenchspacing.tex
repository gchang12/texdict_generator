\input macros
\begindescriptions
\begindesc
\cts frenchspacing {}
\cts nonfrenchspacing {}
\explain
^^{interword spacing}
\TeX\ normally adjusts the spacing between words to account for
punctuation marks.  For example, it inserts extra space at the end of a
sentence and adds some stretch to the \minref{glue} following any
punctuation mark there.  The |\frenchspacing| command tells \TeX\ to make
the interword spacing independent of punctuation, while the
|\nonfrenchspacing| command tells \TeX\ to use its normal spacing rules.  
If you don't specify
|\frenchspacing|, you'll get \TeX's normal spacing.

See \xrefpg{periodspacing} for advice on how to control \TeX's treatment
of punctuation at the end of sentences.

\example
{\frenchspacing  An example: two sentences. Right? No.\par}
{An example: two sentences. Right? No. \par}%
|
\produces
{\frenchspacing  An example: two sentences. Right? No.\par}
{An example: two sentences. Right? No. \par}%
\endexample

\enddesc
\enddescriptions
\end