\input macros
\begindescriptions
\begindesc
\easy\cts narrower {}
\explain
^^{paragraphs//narrow}
This command makes paragraphs narrower, increasing the left and right
margins by |\parindent|, the
current paragraph ^{indentation}.
It achieves this by increasing
both |\leftskip| and |\rightskip| by |\parindent|.
Normally you place |\narrower| at the
beginning of a \minref{group} containing the paragraphs that you want to
make narrower.  If you forget to enclose |\narrower| within a group,
you'll find that all the rest of your document will have narrow
paragraphs.

|\narrower| affects just those paragraphs that end after you invoke it.
If you end a |\narrower| group before you've ended
a paragraph, \TeX\ won't make that paragraph narrower.

\example
{\parindent = 12pt \narrower\narrower\narrower
This is a short paragraph. Its margins are indented
three times as much as they would be
had we used just one ``narrower'' command.\par}
|
\produces
{\parindent = 12pt \narrower\narrower\narrower
This is a short paragraph. Its margins are indented
three times as much as they would be
had we used just one ``narrower'' command.\par}
\endexample
\enddesc
\enddescriptions
\end