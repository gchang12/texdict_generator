\input macros
\begindescriptions
\begindesc
^^{overlapping text}
\cts llap {\<argument>}
\cts rlap {\<argument>}
\explain
These commands enable you to produce text that overlaps
whatever happens to be to the left or to the right of the current
position.  |\llap| backspaces by the width of \<argument> and then
typesets \<argument>.  |\rlap| is similar, except that it typesets 
\<argument> first and then backspaces.  |\llap| and |\rlap| are useful for
placing text outside of the current margins.
Both |\llap| and |\rlap| do their work by creating
a \minref{box} of zero~width.

You can also use |\llap| or |\rlap| to construct special characters by
^{overprinting}, but don't try it unless you're sure that the characters
you're using have the same width (which is the case for a monospaced
font such as ^|cmtt10|, the Computer Modern $10$-point ^{typewriter font}).
^^{Computer Modern fonts}
\example
\noindent\llap{off left }\line{\vrule $\Leftarrow$
left margin of examples\hfil right margin of examples
$\Rightarrow$\vrule}\rlap{ off right}
|
\produces
\noindent\llap{off left }\line{\vrule $\Leftarrow$
left margin of examples\hfil right margin of examples
$\Rightarrow$\vrule}\rlap{ off right}
\endexample

%\example
%{\tt O\llap{!|}}
%|
%\produces
%{\cm \tt O\llap{\char `|}}
%\endexample

\nobreak % don't lose the \see
\enddesc
\enddescriptions
\end