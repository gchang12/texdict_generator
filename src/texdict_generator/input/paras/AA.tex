\input macros
\begindescriptions
\begindesc
\xrdef{fornlets}
\bix^^{ligatures}
^^{special symbols}
^^{European alphabets}
%
\ctsx AA {Scandinavian letter \AA}
\ctsx aa {Scandinavian letter \aa}
\ctsx AE {\AE\ ligature}
\ctsx ae {\ae\ ligature}
\ctsx L {Polish letter \L}
\ctsx l {Polish letter \l}
\ctsx O {Danish/Norwegian letter \O}
\ctsx o {Danish/Norwegian letter \o}
\ctsx OE {\OE\ ligature}
\ctsx oe {\oe\ ligature}
\ctsx ss {German letter \ss}
\explain
These commands produce various letters and ligatures from European
alphabets.
They are useful for occasional words and phrases in these
languages---but if you need to typeset a large amount of text in a European
language, you should probably be using a version of \TeX\ adapted
to that language.\footnote{The \TeX\ Users Group (\xref{resources}) can
provide you with information about European language versions of \TeX.}

You'll need a space after these commands when you use them within a word,
so that
\TeX\ will treat the following letters as part of the word
rather than as part of the command.
You needn't be in \minref{math mode} to use these commands.
\example
{\it les \oe vres de Moli\`ere}
|
\produces
{\it les \oe vres de Moli\`ere}
\endexample
\eix^^{ligatures}
\enddesc
\enddescriptions
\end