\input macros
\begindescriptions
\begindesc
\easy\cts raggedright {}
\cts ttraggedright {}
\explain
These commands cause \TeX\ to typeset your document 
``^{ragged right}''.  Interword spaces all 
have their natural size, i.e., they all have the same width and
don't stretch or shrink.
Consequently the right margin is generally not even.
The alternative, which is \TeX's default, is to typeset your document
justified,
^^{justification}
i.e., with uniform left and right margins.
In justified text, interword spaces are stretched in order to 
make the right margin even.
Some typographers prefer ragged right because 
it avoids distracting ``rivers'' of white space on the printed page.
\minrefs{justified text}

You should use the |\ttraggedright| command when typesetting text in a
monospaced font and the |\raggedright| command when typesetting text in any
other font.  

Most of the time you'll want to apply these commands to an entire document,
but you can limit their effects by enclosing them
in a \minref{group}.
\example
\raggedright ``You couldn't have it if you {\it did\/}
want it,'' the Queen said. ``The rule is, jam tomorrow
and jam yesterday---but never jam {\it today\/}.''
``It {\it must\/} come sometimes to `jam today,%
thinspace'' Alice objected. ``No, it can't'', said the
Queen. ``It's jam every {\it other\/} day: today isn't
any {\it other\/} day.''
|
\produces
\raggedright ``You couldn't have it if you {\it did\/}
want it,'' the Queen said. ``The rule is, jam tomorrow
and jam yesterday---but never jam {\it today\/}.''
``It {\it must\/} come sometimes to `jam today,%
'\thinspace'' Alice objected. ``No, it can't'', said the
Queen. ``It's jam every {\it other\/} day: today isn't
any {\it other\/} day.''
\endexample
\enddesc
\enddescriptions
\end