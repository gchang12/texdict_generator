\input macros
\begindescriptions
\begindesc
\bix^^{ligatures}
\cts noboundary {}
\explain
You can defeat a ligature
or kern that \TeX\ applies to the
first or last character of a word by putting |\noboundary| just before
or just after the word.  
Certain fonts intended for languages other than English
contain a special boundary
character that \TeX\ puts at the beginning
and end of each word.
The boundary character occupies no space and is invisible when printed.
It enables \TeX\ to provide different typographical
treatment to characters at the beginning or end of a word,
since
the boundary character can be part of a sequence of
characters to be kerned or replaced by a ligature.
(None of the standard \TeX\ fonts contain this boundary character.)
The effect of |\noboundary| is to delete the
boundary character if it's there, thus preventing \TeX\
from recognizing the ligature or kern.
\eix^^{ligatures}
\enddesc
\enddescriptions
\end