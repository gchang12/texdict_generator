\input macros
\beginconcepts
\concept box

A \defterm{box}  is a rectangle of material to be typeset.  A single
\refterm{character} is a box by itself,
and an entire page is also a box.
\TeX\ forms a page as a nest of boxes within boxes within boxes. The
outermost box is the page itself, the innermost boxes are mostly
single characters, and single lines are boxes that are
somewhere in the middle.

\TeX\ carries out most of its box-building activities implicitly as it
constructs paragraphs and pages.
You can construct boxes explicitly
using a number of \TeX\ \refterm{commands}, notably 
^|\hbox| \ctsref{\hbox},
^|\vbox| \ctsref{\vbox}, and
^|\vtop| \ctsref{\vtop}.
The ^|\hbox| command
constructs a box by appending smaller boxes horizontally from left to right;
it operates on a \refterm{horizontal list} and yields
an \refterm{hbox} ^^{hboxes} (horizontal box).
^^{horizontal lists}
The ^|\vbox| and |\vtop| commands
construct a box by appending smaller boxes vertically from top to bottom;
^^{vboxes}
they operate on a \refterm{vertical list}
and yield a \refterm{vbox} ^^{vboxes} (vertical box).
^^{vertical lists}
These horizontal and vertical lists can include not just smaller boxes but
several other kinds of entities as well, e.g.,  \refterm{glue} and
kerns.
^^{kerns//as list items}

A box has \refterm{height}, \refterm{depth}, and \refterm{width},
^^{height} ^^{depth} ^^{width}
like this:
\vdisplay{\offinterlineskip\sevenrm
   \halign{#&#&\kern3pt \hfil#\hfil\cr
      &\hrulefill\cr
      &\vrule
         \vtop to .7in{\vss \hbox to .9in{\hss baseline\hss}\vskip4pt}%
       \vrule
      &\labelledheight{.7in}{height}\cr
      %
       \vbox to 0pt{
          \vss
          \hbox{reference point \hbox to 15pt{\rightarrowfill}%
          \hskip3pt}%
       \kern-4.5pt}&{\box\refpoint}\hrulefill\cr
      %
      \omit
      &\vrule\hfil\vrule
      &\labelledheight{.4in}{depth}\cr
      %
      &\hrulefill\cr
      %
   \noalign{\vskip3pt}%
      &\leftarrowfill { width }\rightarrowfill\cr
}}

^^{baselines}
The \refterm{baseline} is like one of
the light guidelines on a pad of ruled paper.  
The boxes for letters such as `g'
extend below the baseline; the boxes for letters such as `h' don't. 
The height of a box is the distance that the box extends above its
baseline, while its depth is the distance that it extends below its
baseline. \bix^^{reference point}
The \minref{reference point}
of a box is the place where its baseline intersects its left edge.

{\tighten
\TeX\ builds an hbox $H$ from a horizontal list by assuming
a reference point for $H$ and then appending the items in the list to $H$
one by one from left to right.
Each box in the list is placed so that its baseline coincides with the
baseline of $H$\kern-2pt,
i.e., the component boxes are lined up horizontally.%
\footnote{If a box is moved up or down with ^|\raise| or 
^|\lower|, \TeX\ uses its reference point before the move when
placing it.}
The height of $H$ is the
height of the tallest box in the list, and the depth of $H$ is the depth
of the deepest box in the list.
The width of $H$ is the sum of the
widths of all the items in the list.
If any of these items are \refterm{glue} and \TeX\ needs to stretch or shrink
the glue,
the width of $H$ will be larger or smaller accordingly.
See \knuth{page~77} for the~details.
\par}

Similarly, \TeX\ builds a vbox $V$ from a vertical list by assuming a
temporary reference point for $V$ and then appending the items in the
list to $V$ one by one from top to bottom.  Each box in the list is
placed so that its reference point is lined up vertically with the
reference point of \Vperiod.\footnote{If a box is moved left or right with
^|\moveleft| or ^|\moveright|, \TeX\ uses its reference point before the
move when placing it.} As each box other than the first one is added to
\Vcomma, \TeX\ puts \minref{interline glue} just above it.  (This
^{interline glue} has no analogue for hboxes.)  The width of $V$ is the
width of the widest box in the list, and the vertical extent (height
plus depth) of $V$ is the sum of the vertical extents of all the
items in the list.

\bix^^|\vbox|
\bix^^|\vtop|
The difference between |\vbox| and |\vtop| is in how they partition
the vertical extent of $V$ into a height and a depth.
Choosing the reference point of $V$ determines that partition.
\ulist
\li For |\vbox|, \TeX\ places the reference point on a horizontal line
with the reference point of the last component box 
or rule of \Vcomma, except
that if the last box (or rule) is followed by glue or a kern, \TeX\ places the
reference point at the very bottom of \Vperiod.%
\footnote{The depth is limited by
the parameter ^|\boxmaxdepth| \ctsref{\boxmaxdepth}.}

\li For |\vtop|, \TeX\ places the reference point on a horizontal line
with the reference point of the first component box or rule of \Vcomma,
except that if the first box (or rule)
is preceded by glue or a kern, \TeX\ places
the reference point at the very top of \Vperiod.

\endulist
\noindent
Roughly speaking, then, |\vbox| puts the reference point near the bottom
of the vbox and |\vtop| puts it near the top.
When you want to align a
row of vboxes so that their tops line up horizontally,
you should usually use |\vtop| rather than |\vbox|. 
See \knuth{pages~78 and 80--81} for the
details of how \TeX\ builds vboxes.
\eix^^|\vbox|
\eix^^|\vtop|
\eix^^{reference point}

You have quite a lot of freedom in constructing boxes.  The typeset
material in a box can extend beyond the boundaries of the box as it does
for some letters (mostly italic or slanted ones).  The component boxes
of a larger box can overlap.  A box can have negative width, depth, or
height, though boxes like that are not often needed.

You can save a box in a box \refterm{register} and retrieve it later.
Before using a box register,
^^{box registers}
you should reserve it and give it a name
with the ^|\newbox| command \ctsref{\@newbox}.  See 
\conceptcit{register} for more information about box
registers.
\endconcept



\endconcepts
\end