\input macros
\beginconcepts
\concept rule

A \defterm{rule} is a solid black rectangle.
A rule, like a \refterm{box},
has \refterm{width}, \refterm{height}, and \refterm{depth}.
The vertical dimension of the rectangle
is the sum of its height and its depth.
An ordinary horizontal or vertical straight line is a special case of a rule.
 
\bix^^{horizontal rules}
\bix^^{vertical rules}
\bix^^|\hrule|
\bix^^|\vrule|
A rule can be either horizontal or vertical.  The distinction between a
horizontal rule and a vertical one has to do with how you produce the
rule, since a vertical rule can be short and fat (and therefore look
like a horizontal line), while a horizontal rule can be tall and skinny
(and therefore look like a vertical line).  \TeX's notion of a rule is
more general than that of typographers, who think of a rule as a line
and would not usually call a square black box a rule.
 
You can produce a horizontal rule using the
|\hrule| command and a vertical rule using
the |\vrule| command \ctsref{\vrule}.
For example, the control sequence |\hrule| by itself
produces a thin rule that runs across the page, like this:

{\offinterlineskip
\nobreak\medskip
\hrule
\medskip}

The command `|\vrule height .25in|' produces a vertical rule
that runs $.25$~inches down the page like this:
\nobreak\vskip \abovedisplayskip
\leftline{\vrule height .25in}
\vskip \belowdisplayskip

There are two differences between horizontal rules and vertical rules:
\olist
\li For a horizontal rule, \TeX\ defaults the width to the width of the
smallest \refterm{box} or \refterm{alignment} that encloses it.  For a
vertical rule, \TeX\ defaults the height and depth in the same way.  (The
default is the size that you get if you don't give a size explicitly for that
dimension.)

^^{horizontal lists//rule in}
^^{vertical lists//rule in}
\li 
{\tighten
A horizontal rule is an inherently vertical item that cannot participate in
a \refterm{horizontal list},
while a vertical 
rule is an inherently horizontal item
that cannot participate in a \refterm{vertical list}.  This behavior
may seem strange at first but there is good reason for it:
a horizontal rule ordinarily runs visually from left 
to right and thus separates items in a vertical list,
while a vertical rule ordinarily runs visually from top to bottom
and thus separates items in a horizontal list.
%(Look at the rules that are shown above.)
\par}
\endolist

{\tighten
If you construct a rule with three explicit dimensions, it will look the
same whether you make it a horizontal rule or a vertical rule.
For example, the command `|\vrule height1pt depth2pt width3in|' produces this
horizontal-looking rule:
\par}

{\offinterlineskip
\nobreak\medskip\nobreak\vskip3pt
\leftline{\vrule height1pt depth2pt width3in}
\medskip}

You'll find a precise statement of \TeX's treatment of rules on
\knuth{pages~221--222}.
\eix^^{horizontal rules}
\eix^^{vertical rules}
\eix^^|\hrule|
\eix^^|\vrule|
\endconcept


\endconcepts
\end