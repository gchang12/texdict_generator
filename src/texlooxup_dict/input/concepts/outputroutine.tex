\input macros
\beginconcepts
\concept {output routine}

When \TeX\ has accumulated
at least enough material to fill up a page, it chooses a breakpoint
and places the material before the breakpoint in |\box255|. It then
calls the
current \defterm{output routine}, which processes the material and eventually
sends it to the \dvifile.
^^{\dvifile//material from output routine}
The output routine can perform further
processing, such as inserting headers, footers, and footnotes.  
\refterm{\PlainTeX:\plainTeX} provides
a default output routine that inserts a centered page number
at the bottom of each page.  
By providing a different output routine you can achieve such
effects as double-column output.
You can think of the output routine as having a single responsibility:
disposing of the material in |\box255| one way or another.

The current output routine is defined by the value of ^|\output|
\ctsref{\output}, which is a list of \refterm{tokens:token}.  When \TeX\
is ready to produce a page, it just expands the token list.

You can make some simple changes to the actions of the \plainTeX\
output routine without actually modifying it.  For example, by assigning
a list of \refterm{tokens:token} to |\headline| or
|\footline| \ctsref{\footline} you can have \TeX\ produce a different
header or footer than it ordinarily would.

The output routine is also
responsible for collecting any \refterm{insertions:insertion};
combining those insertions and any
``decorations'' such as headers and
footers with the main contents of the page and packaging all
of this material in a box; and
eventually sending that box to the \dvifile\ 
^^{\dvifile//material from output routine}
with the ^|\shipout|
command \ctsref{\shipout}.
Although this is what an output routine most often does,
a special-purpose output routine might behave differently.
\endconcept



\endconcepts
\end