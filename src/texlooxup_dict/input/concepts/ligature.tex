\input macros
\beginconcepts
\concept ligature

A \defterm{ligature} is a single character that replaces a
particular sequence of adjacent characters in a typeset document.
For example, the word `|office|' is typeset as \hbox{``office''},
not \hbox{``of{f}ice''}, by high-quality typesetting systems.
Knowledge of ligatures is built into the
\refterm{fonts:font} that you use, so there's nothing explicit you need do
in order to get \TeX\ to produce them.  (You could defeat the ligature
in ``office'', as we did just above, by writing `|of{f}ice|' in your input.)
\TeX\ is also capable of using its ligature mechanism to typeset the
first or last letter of a word differently than the same letter as it would
appear in the middle of a word.
You can defeat this effect (if you ever encounter it) by using the
^|\noboundary| command (\xref\noboundary).

Sometimes you may need a ligature from a European language.
^^{European languages}
\TeX\ won't
produce these automatically unless you're using a font designed for that
language.  A number of these ligatures, e.g., `\AE', are available as
commands (see ``Letters and ligatures for European alphabets'',
\xref{fornlets}).
\endconcept



\endconcepts
\end