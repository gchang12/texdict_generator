\input macros
\beginconcepts
\concept argument

^^{commands//arguments of}
An \defterm{argument} contains text that is passed to a
\refterm{command}.
The arguments of a command complete the description of what
the command is supposed to do.
The command can either be a \refterm{primitive} command or a 
\refterm{macro}.

Each primitive command ^^{primitive//command}
has its own convention about the form of its
arguments.  For instance, the sequence of \refterm{tokens:token}:

\csdisplay
\hskip 3pc plus 1em
|
consists of the command `|\hskip|' and the arguments
`|3pc plus 1em|'.  But if you were to write:

\csdisplay
\count11 3pc plus 1em
|
you'd get an entirely different effect.
\TeX\ would treat `|\count11|' as a command with argument `|3|',
followed by the ordinary text tokens `|pc plus 1em|'
(because count registers expect a number to be assigned to them)%
---probably not what
you intended.  The effect of the command, by the way, would be to
assign $3$ to count register $11$ (see the discussion of ^|\count|,
\xref\count).

Macros, on the other hand, all follow the same convention
for their arguments.
^^{macros//arguments of}
Each argument passed to a macro
corresponds to a \refterm{parameter}
^^{parameters//and arguments}
in the definition of that
macro. ^^{macros//parameters of}
A macro parameter is either ``delimited'' or ``undelimited''.
The macro definition determines the number and nature of the macro parameters
and therefore the number and nature of the macro arguments.

The difference between a delimited argument and an undelimited argument
lies in the way that \TeX\ decides where the argument ends.
^^{delimited arguments}
^^{undelimited arguments}
\ulist
\li A delimited argument consists of the tokens
from the start of the argument up to, but not including, the
particular sequence of tokens that serves as the delimiter for that argument.
The delimiter is specified in the macro definition.  Thus you 
supply a delimited argument to a macro by writing the argument itself
followed by the delimiter.  A delimited argument can be empty, i.e., have
no text at all in it.  Any braces in a delimited argument must be paired
properly, i.e., every left brace must have a corresponding right brace
and vice versa.

\li An undelimited argument consists of a single token or a sequence of
tokens enclosed in braces, like this:
`|{Here is {the} text.}|'.  Despite appearances, the outer braces don't
form a \refterm{group}---\TeX\ uses them only to determine what the
argument is.  Any inner braces, such as the ones around `|the|', must be
paired properly.  If you make a mistake and put in too many right
braces, \TeX\ will complain about an unexpected right brace.  \TeX\ will
also complain if you put in too many left braces, but you'll probably
get \emph{that} complaint long after the place where you intended to
end the argument (see \xref{mismatched}).
\endulist
\noindent 
See \conceptcit{macro} for more information
about parameters and arguments.  You'll find the precise rules pertaining
to delimited and undelimited arguments in \knuth{pages~203--204}.
\endconcept


\endconcepts
\end