\input macros
\beginconcepts
\concept mathcode

A \defterm{mathcode} is a number that \TeX\ uses to identify and
describe a math character,
^^{math characters//described by mathcodes}
i.e., a character that has a
particular role in a math formula.  A mathcode conveys three pieces of
information about a character: its \refterm{font} position, its
\refterm{family}, and its \refterm{class}.
Each of the $256$ possible
input characters has a mathcode, which is defined by the \TeX\ program
but can be changed.

^^{family//as part of mathcode}
\TeX\ has sixteen families of fonts, numbered $0$--$15$.  Each
family contains three fonts: one for \refterm{text size}, one for
\refterm{script size}, and one for \refterm{scriptscript size}.  \TeX\
chooses the size of a particular character, and therefore its font,
according to the context.  The class of a character specifies its role
in a formula (see \knuth{page~154}).  For example, the equals sign `|=|'
is in class $3$ (Relation).  \TeX\ uses its knowledge of character
classes when it is deciding how much space to put between different
components of a math formula.

The best way to understand what mathcodes are all about is to see how
\TeX\ uses them. So we'll show you what \TeX\ does with a
character token $t$ of \refterm{category code}~11 or~12 in a math
formula:

\olist\compact
\li It looks up the character's mathcode.
\li It determines a family $f$ from the mathcode.
\li It determines the size $s$ from the context.
\li It selects a font $F$ by picking the font for size $s$ in family $f$.
\li It determines a character number $n$ from the mathcode.
\li It selects as the character $c$ to be typeset the character
at position $n$ of font $F$.
\li It adjusts the spacing around $c$ according to the class of $t$ and
the surrounding context.
\li It typesets the character $c$.
\endolist

The context dependence in
items (3) and (7) implies that \TeX\ cannot typeset a math character
until it has seen the entire formula containing the
math character.  For example, in the formula
`|$a\over b$|', \TeX\ doesn't know what size the `|a|' should be until it
has seen the |\over|.

{\tighten
The mathcode of a character is encoded according to the formula $4096c
+ 256f + n$, where $c$ is the class of the character, $f$ is its
\refterm{family}, and $n$ is its \refterm{\ascii\ character} code within
the family.  You can change \TeX's interpretation of an input character
in math mode by assigning a value to the ^|\mathcode|
table entry \ctsref{\mathcode}
for that character.  The character must have a
\refterm{category code} of $11$ (letter) or $12$ (other) for \TeX\ to
look at its |\mathcode|.
}\par

^^{family//variable}
You can define a mathematical character to have a ``variable'' family by
giving it a class of $7$.  Whenever \TeX\ encounters that character in a
math formula, it takes the family of the character to be the current
value of the |\fam| parameter \ctsref{\fam}.  A variable family enables
you to specify the font of ordinary text in a math formula.  For
instance, if the roman characters are in family $0$, the assignment
|\fam = 0|
will cause ordinary text in a math formula to be set in roman type
rather than in something else like math italic type. If the value of
|\fam| is not in the range from $0$ to $15$, \TeX\ takes the value to be
$0$, thus making classes $0$ and $7$ equivalent.
\TeX\ sets |\fam| to $-1$ whenever it enters math mode.
\endconcept



\endconcepts
\end