\input macros
\beginconcepts
\concept group

A \defterm{group} is a part of your manuscript that \TeX\ treats as a unit.
You indicate a group by enclosing it in the braces
`|{|' and `|}|' (or any other characters with the appropriate
\refterm{category codes}).
^^|{//starting a group|
^^|}//ending a group|

The most important property of a group is that any nonglobal
definition or assignment that you make inside a group disappears when
the group ends.  For instance, if you write:

\csdisplay
Please don't pour {\it any} more tea into my hat.
|
the |\it| \refterm{control sequence} causes \TeX\ to set the word
`|any|' in italic type but does not affect the rest of the text.
As another example, if you use the |\hsize| parameter
\ctsref{\hsize} to change the line length within a group, the line length
reverts to its previous value once \TeX\ has gotten past the group.

Groups are also useful as a way of controlling spacing.  For instance, if you
write:

\csdisplay
\TeX for the Impatient and the Outpatient too.
|
\noindent
you'll get:
\display{%
\TeX for the Impatient and the Outpatient too.
}
\noindent
since the control sequence |\TeX| (which produces the \TeX\
logo) absorbs the following space. 
What you probably want is:
\display{%
{\TeX} for the Impatient and the Outpatient too.
}
\noindent
One way to get it is to enclose `|\TeX|' in a group:
\csdisplay
{\TeX} for the Impatient and the Outpatient too.
|
The right brace prevents the control sequence from absorbing the space.
\endconcept



\endconcepts
\end