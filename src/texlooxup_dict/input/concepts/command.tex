\input macros
\beginconcepts
\concept command

A \defterm{command} instructs \TeX\ to carry out a certain action.
Every \refterm{token} that reaches \TeX's stomach \seeconcept{\anatomy}
acts as a command, except for those that are parts of arguments to
other commands (see below).
^^{tokens//as commands}
A command can be invoked by a
\refterm{control sequence}, by an \refterm{active character}, or by an
ordinary character.  It might seem odd that \TeX\ treats an ordinary
character as a command, but in fact that's what it does:
when \TeX\ sees 
an ordinary character
it constructs a \refterm{box} containing that character typeset in
the current font.

A command can have arguments.
The arguments of a command are single tokens or
groups of tokens that complete the description of what
the command is supposed to do.
For example, the command `|\vskip 1in|' tells \TeX\ to skip
$1$ inch vertically.  It has an argument `|1in|',
which consists of three tokens.
The description of what |\vskip| is supposed to do would be incomplete
without specifying how far it is supposed to skip.
The tokens in the arguments to a command are not themselves considered
to be commands.

Some examples of different kinds of \TeX\ commands are: 
\ulist\compact
\li Ordinary characters, such as `|W|', which instructs \TeX\
to produce a box containing a typeset `W'
\li Font-setting commands,
such as |\bf|, which begins boldface type
\li Accents, such as |\`|, which produces a grave accent as in `\`e'
\li Special symbols and ligatures, such as |\P| (\P) and |\ae| (\ae)
\li Parameters, such as |\parskip|, the amount of glue that
\TeX\ puts between paragraphs
\li Math symbols, such as |\alpha| ($\alpha$) and |\in| ($\in$)
\li Math operators, such as |\over|, which produces a fraction
\endulist
\endconcept



\endconcepts
\end