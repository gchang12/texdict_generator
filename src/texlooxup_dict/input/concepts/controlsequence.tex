\input macros
\beginconcepts
\concept {control sequence}

A \defterm{control sequence} is a name for a \TeX\ \refterm{command}.
A control sequence always starts with an ^{escape character}, usually
a backslash (|\|).
\indexchar \
A control sequence takes one of two forms:

\ulist

\li A \refterm{control word} is a control sequence consisting of an
\refterm{escape character} followed by one or more letters.
^^{control words}
The control
word ends when \TeX\ sees a nonletter.  For instance, when \TeX\ reads
`\hbox{|\hfill!visiblespace,!visiblespace!.the|}', it sees six 
\refterm{tokens:token}:
the control sequence `|\hfill|', comma, space, `|t|', `|h|', `|e|'.  The
space after `|\hfill|' ends the control sequence and
is absorbed by \TeX\ when it scans the control sequence.
(For the text `|\hfill,!visiblespace!.the|', on the other hand,
the comma both ends the control sequence and counts as a character in its
own right.)

\li A \refterm{control symbol}
^^{control symbols}
is a control sequence consisting of an
^{escape character} followed by any character other than a letter---%
even a space or an end of line.
A control symbol is self-delimited, i.e., \TeX\ knows where it ends without
having to look at what character comes after it.
The character after a control symbol is never absorbed by
the control symbol.
\endulist
\noindent See \xrefpg{spaces} for more information about spaces after control
sequences. 

\TeX\ provides a great many predefined control sequences.  The
\refterm{primitive} control sequences are built into the \TeX\ computer
program and thus are available in all forms of \TeX.
^^{primitive//control sequence}
Other
predefined control sequences are provided by \refterm{\plainTeX}, the
form of \TeX\ described in this book.

You can augment the predefined control sequences with ones of your own,
using commands such as ^|\def| and ^|\let| to define them.
\chapterref{eplain} of this book contains a
collection of control sequence definitions that you may find
useful.  In addition, your computing facility may 
be able to provide a collection of
locally developed \TeX\ macros.
\endconcept



\endconcepts
\end