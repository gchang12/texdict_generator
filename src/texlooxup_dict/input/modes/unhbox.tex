\input macros
\begindescriptions
\begindesc
\cts unhbox {\<register>}
\cts unvbox {\<register>}
^^{boxes//extracting contents of}
\explain
These commands produce the list contained in box register \<register> and make
that box register void.
|\unhbox| applies to box registers containing 
hboxes and |\unvbox| applies to box registers containing vboxes.
You should use these commands in preference to |\unhcopy| and |\unvcopy|
(see below)
when you don't care about what's in the box register after you've used it,
so as not to exhaust \TeX's memory by filling it with obsolete boxes.
\example
\setbox0=\hbox{The Mock Turtle sighed deeply, and
drew the back of one flapper across his eyes. }
\setbox1=\hbox{He tried to speak, but sobs choked 
               his voice. }
\unhbox0 \unhbox1
% \box0 \box1 would set two hboxes side by side
% (and produce a badly overfull line).
\box1 % produces nothing
|
\produces
\setbox0=\hbox{The Mock Turtle sighed deeply, and
drew the back of one flapper across his eyes. }
\setbox1=\hbox{He tried to speak, but sobs choked 
               his voice. }
\unhbox0 \unhbox1
% \hbox0 \hbox1 would set two hboxes side by side
% (and produce a badly overfull line).
\box1 % Produces nothing.
\endexample
\enddesc
\enddescriptions
\end