\input macros
\begindescriptions
\begindesc
\margin{This subsection is a merger of two previous subsections.
The commands have also been reordered.}
\bix^^{struts}
\cts strut {}
\explain
This command produces a box whose width is zero
and whose height (|8.5pt|) and depth (|3.5pt|)
are those of a more or less typical line of type in |cmr10|,
the default \plainTeX\ font.
Its main use is in forcing lines to have the same height
when you've disabled \TeX's interline glue with
|\offinter!-lineskip| ^^|\offinterlineskip| or a similar command,
e.g., when you're constructing an alignment.
If the natural height of a line is
too short, you can bring it up to standard by including a |\strut|
in the line.  The strut will force the height and depth of the line
to be larger, but it won't print anything or consume any horizontal
space.

If you're setting type in a font that's bigger or smaller than |cmr10|,
you should redefine |\strut| for that context.
\example
\noindent % So we're in horizontal mode.
\offinterlineskip % So we get the inherent spacing.
% The periods in this vbox are not vertically equidistant.
\vtop{\hbox{.}\hbox{.(}\hbox{.x}
   \hbox{.\vrule height 4pt depth 0pt}}\qquad
% The periods in this vbox are vertically equidistant
% because of the struts.
\vtop{\hbox{.\strut}\hbox{.(\strut}\hbox{.x\strut}
   \hbox{.\vrule height 4pt depth 0pt\strut}}
|

\produces
\noindent % So we're in horizontal mode.
\offinterlineskip % So we get the inherent spacing.
% The periods in this vbox are not vertically equidistant.
\vtop{\hbox{.}\hbox{.(}\hbox{.x}
   \hbox{.\vrule height 4pt depth 0pt}}\qquad
% The periods in this vbox are vertically equidistant
% because of the struts.
\vtop{\hbox{.\strut}\hbox{.(\strut}\hbox{.x\strut}
   \hbox{.\vrule height 4pt depth 0pt\strut}}
\endexample
\enddesc
\enddescriptions
\end