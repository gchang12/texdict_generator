\input macros
\begindescriptions
\begindesc
\cts topglue {\<glue>}
\explain
\margin{Command added; recent addition to \TeX}
This command\footnote{|\topglue| was added to \TeX\ in version 3.0,
later than the other enhancements introduced by ^{\newTeX}
(\xref{newtex}).  It is first described in the \emph{eighteenth\/}
edition of \texbook.} causes the space from the top of the page to the
top of the first box on the page to be \<glue> precisely.
The top of the page is considered to be at the baseline of an
imaginary line of text just above the top line of the page.
More precisely, it's a distance |\topskip| above the origin as given by
|\hoffset| and |\voffset|.

This command is useful because \TeX\ ordinarily adjusts the glue
produced by |\topskip| in a complex way.  By using |\topglue| you can
control the position of the first box on the page without worrying about
those adjustments.

\enddesc
\enddescriptions
\end