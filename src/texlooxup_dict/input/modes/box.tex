\input macros
\begindescriptions
\begindesc
\bix^^{box registers}
%
\cts setbox {\<register>\thinspace{\bt =}\thinspace\<box>}
\cts box {\<register>}
\explain
^^{assignments//of boxes}
These commands respectively set and retrieve the contents
of the box register 
whose number is \<register>.  
Note that you set a box register a little differently than
you set the other kinds of registers: you use 
|\setbox|$\,n$~|=| rather than |\box|$\,n$~|=|.

\emph{Retrieving the contents of a box register
with these commands has the side effect of emptying it,
so that the box register become void.}  If you don't want that to happen, 
you can use |\copy| (see below) to retrieve the contents.
You should use |\box| in preference to |\copy|
when you don't care about what's in a box register after you've used it,
so as not to exhaust \TeX's memory by filling it with obsolete boxes.
\example
\setbox0 = \hbox{mushroom}
\setbox1 = \vbox{\copy0\box0\box0}
\box1
|
\produces
\setbox0 = \hbox{mushroom}
\setbox1 = \vbox{\copy0\box0\box0}
\box1
\endexample
\enddesc
\enddescriptions
\end