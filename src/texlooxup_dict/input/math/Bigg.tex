\input macros
\begindescriptions
\begindesc
\makecolumns 16/4:
\easy\cts big {}
\cts bigl {}
\cts bigm {}
\cts bigr {}
\cts Big {}
\cts Bigl {}
\cts Bigm {}
\cts Bigr {}
\cts bigg {}
\cts biggl {}
\cts biggm {}
\cts biggr {}
\cts Bigg {}
\cts Biggl {}
\cts Biggm {}
\cts Biggr {}
\explain
^^{delimiters//enlarging}
These commands make \minref{delimiter}s bigger than their normal size.
The commands in the four columns
produce successively larger sizes.  The difference between |\big|,
|\bigl|, |\bigr|, and |bigm| has to do with the \minref{class} of the
enlarged delimiter:
\ulist\compact
\li |\big| produces an ordinary symbol.
\li |\bigl| produces an opening symbol.
\li |\bigr| produces a closing symbol.
\li |\bigm| produces a relation symbol.
\endulist
\noindent
\TeX\ uses the class of a symbol in order to decide how much space to put 
around that symbol.

These commands, unlike |\left| and |\right|,
do \emph{not} define a group.

\example
$$(x) \quad \bigl(x\bigr) \quad \Bigl(x\Bigr) \quad
   \biggl(x\biggr) \quad \Biggl(x\Biggr)\qquad
[x] \quad \bigl[x\bigr] \quad \Bigl[x\Bigr] \quad
   \biggl[x\biggr] \quad \Biggl[x\Biggr]$$
|
\dproduces
$$(x) \quad \bigl(x\bigr) \quad \Bigl(x\Bigr) \quad
\biggl(x\biggr) \quad \Biggl(x\Biggr)\qquad
[x] \quad \bigl[x\bigr] \quad \Bigl[x\Bigr] \quad
\biggl[x\biggr] \quad \Biggl[x\Biggr]$$
\endexample
\enddesc
\enddescriptions
\end