\input macros
\begindescriptions
\begindesc
\bix^^{displays//multiline}
\cts displaylines
   {{\bt \rqbraces{\<line>\ths\\cr$\ldots$\<line>\ths\\cr}}}
\explain
This command produces a multiline math display in which each line is
centered independently of the other lines.
You can use the |\noalign| command (\xref \noalign) to change the amount
of space between two lines of a multiline display.

If you want to attach equation numbers to some or all of the equations
in a multiline math display, you should use |\eqalignno| or
|\leqalignno|.
\example
$$\displaylines{(x+a)^2 = x^2+2ax+a^2\cr
                (x+a)(x-a) = x^2-a^2\cr}$$
|
\dproduces\centereddisplays
$$\displaylines{
(x+a)^2 = x^2+2ax+a^2\cr
(x+a)(x-a) = x^2-a^2\cr
}$$
\endexample
\enddesc
\enddescriptions
\end