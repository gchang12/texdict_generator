\input macros
\begindescriptions
\begindesc
\bix^^{registers}
\makecolumns 11/2:
\cts count {\<register> {\bt =} \<number>}
\cts dimen {\<register> {\bt =} \<dimen>}
\cts skip {\<register> {\bt =} \<glue>}
\cts muskip {\<register> {\bt =} \<muglue>}
\cts toks {\<register> {\bt =} \<token variable>}
\aux\cts toks {\<register> {\bt =} \rqbraces{\<token list>}}
\aux\cts count {\<register>}
\aux\cts dimen {\<register>}
\aux\cts skip {\<register>}
\aux\cts muskip {\<register>}
\aux\cts toks {\<register>}
\explain
^^{assignments//of registers}
The first six commands listed here assign something to a register.
The |=|'s in the assignments are optional.
The remaining five control sequences are not true commands
because they can only appear as part of an argument.
They yield the contents of the specified register.
Although you can't use these control sequences by themselves as commands
in text,  you can use ^|\the| to convert them to text so that
you can typeset their values.

You can name and reserve registers
with the ^|\newcount| command and its relatives
(\xref{\@newcount}).
Using these commands is a safe way to obtain registers that 
are known not to have any conflicting usage.

^^{count registers}
A |\count| register contains an integer, which can be either positive or
negative.
Integers can be as large as you're ever likely to need them to be.\footnote
{Here's the only exercise in this book: find out what's the largest
integer that \TeX\ will accept.}
\TeX\ uses count registers $0$--$9$ to keep track of the
page number (see \knuth{page~119}).
|\count255| is the only count register available for use
without a reservation.
\example
\count255 = 17 \number\count255
|
\produces
\count255 = 17 \number\count255
\endexample

\medskip\noindent
^^{dimension registers}
A |\dimen| register contains a dimension.
Registers |\dimen0| through |\dimen9| and |\dimen255| are available
for scratch use.

\example
\dimen0 = 2.5in
\hbox to \dimen0{$\Leftarrow$\hfil$\Rightarrow$}
|
\produces
\dimen0 = 2.5in
\hbox to \dimen0{$\Leftarrow$\hfil$\Rightarrow$}
\doruler{\8\8\8}3{in}
\endexample

\medskip\noindent
^^{skip registers}
A |\skip| register contains the dimensions of glue.
Unlike a |\dimen| register, it
records an amount of shrink and stretch as well as a natural size.
Registers |\skip0| through |\skip9| and |\skip255| are available
for use without a reservation.

\example
\skip2 = 2in
$\Rightarrow$\hskip \skip2 $\Leftarrow$
|
\produces
\skip2 = 2in
$\Rightarrow$\hskip \skip2 $\Leftarrow$\par
\noindent\hphantom{$\Rightarrow$}\ruler{\8\8}2{in}
\endexample

\medskip\noindent
^^{muskip registers}
A |\muskip| register is like a |\skip| register,
but the glue in it is always measured in ^|mu|
\seeconcept{mathematical unit}.
The size of a |mu| depends on the current font.
For example, it's usually a little
smaller in a subscript than in ordinary text.
Registers |\muskip0| through |\muskip9| and |\muskip255| are available
for use without a reservation.

\example
\muskip0 = 24mu % An em and a half, no stretch or shrink.
$\mathop{a \mskip\muskip0 b}\limits^{a \mskip\muskip0 b}$
% Note the difference in spacing.
|
\produces
\muskip0 = 24mu % an em and a half
$\mathop{a \mskip\muskip0 b}\limits^{a \mskip\muskip0 b}$
% Note the difference in spacing
\endexample

\medskip\noindent
^^{token registers}
You can assign either a token variable
(a register or a parameter) or a token list
to a |\toks| register.
When you assign a token list to a token register,
the tokens in the token list are \emph{not} expanded.

Once the tokens in a token list have been inserted into text
using ^|\the|, they are
expanded just like tokens that were read in directly.
They have the category codes that they received when \TeX\ first
saw them in the~\hbox{input}.

\example
\toks0 = {the \oystereaters\ were at the seashore}
% This assignment doesn't expand \oystereaters.
\def\oystereaters{Walrus and Carpenter}
\toks1 = \toks0
% the same tokens are now in \toks0 and \toks1
Alice inquired as to whether \the\toks1.
|
\produces
\toks0 = {the \oystereaters\ were at the seashore}
% This assignment doesn't expand \oystereaters
\def\oystereaters{Walrus and Carpenter}
\toks1 = \toks0
% the same tokens are now in \toks0 and \toks1
Alice inquired as to whether \the\toks1.
\endexample
\enddesc
\enddescriptions
\end