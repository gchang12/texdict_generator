\input macros
\begindescriptions
\begindesc
\cts tracingstats {\param{number}}
\explain
If this parameter is $1$ or greater, \TeX\ will
include a report on the resources that it used to run your job
(see \knuth{page~300} for a list and explanation of these resources).
Moreover, if |\tracingstats| is $2$ or greater,
\TeX\ will report on its memory usage whenever it does a
^|\shipout| (\xref \shipout) for a page.
The report appears at the end of the log file. 
^^{log file//tracing statistics in}
If ^|\tracingonline| is greater than zero, the information will also appear
at your terminal.
If you're having trouble with \TeX\ exceeding one of its
capacities, the information provided by |\tracingstats| may help you
pinpoint the cause of the difficulty.

Some production forms of \TeX\ ignore the value of |\tracingstats|
so that they can run faster.
If you need to use this parameter, be sure to use a form that
responds to it.

The following example shows a sample of
the tracing output you'd get on one implementation
of \TeX.  It may be different on other implementations.
{\codefuzz = 1in
\example
\tracingstats=1
|
\logproduces
Here is how much of TeX's memory you used:
 4 strings out of 5540
 60 string characters out of 72328
 5956 words of memory out of 262141
 921 multiletter control sequences out of 9500
 14794 words of font info for 50 fonts, out of 72000 for 255
 14 hyphenation exceptions out of 607
 7i,4n,1p,68b,22s stack positions out of 300i,40n,60p,3000b,4000s
|
\endexample
}% end scope of codefuzz
\enddesc
\enddescriptions
\end