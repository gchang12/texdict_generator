\input macros
\begindescriptions
\begindesc
\margin{{\tt\\delcode} was explained in two places.  This explanation
combines them.  (The other place was in the math section.)}
\cts delcode {\<charcode>\tblentry{number}}
\explain
^^{delimiter codes}
This table entry specifies the \minref{delimiter} code for the input character
whose \ascii\ code is \<charcode>.
The delimiter code tells \TeX\ how to find the best output character to use
for typesetting the indicated input character as a delimiter.

\<number> is normally written in hexadecimal notation.
Suppose that \<number> is the hexadecimal number $s_1s_2s_3\,
l_1l_2l_3$.  Then when the character is used as a delimiter,
\TeX\ takes the character to have small variant
$s_1s_2s_3$ and large variant $l_1l_2l_3$.  Here $s_1s_2s_3$ indicates
the math character found in position $s_2s_3$ of family $s_1$, and
similarly for $l_1l_2l_3$.  This is the same convention as the one
used for ^|\mathcode| (\xref \mathcode),
except that |\mathcode| also specifies a class.
\example
\delcode `( = "028300  % As in plain TeX.
|
\endexample
\enddesc
\enddescriptions
\end