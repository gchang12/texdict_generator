\input macros
\begindescriptions
\begindesc
\cts chardef {\<control sequence>=\<charcode>}
\explain
^^{characters//defined by \b\tt\\chardef\e}
This command defines \<control sequence> 
to be \<charcode>.
Although |\chardef| is most often used to define characters, you can also
use it to give a name to a number in the range $0$--$255$ even when you
aren't using that number as a character code.
\example
\chardef\percent = `\% 21\percent, {\it 19\percent}
% Get the percent character in roman and in italic
|
\produces
\chardef\percent = `\%
21\percent, {\it 19\percent}
% You'll get the percent character in roman and in italic
\endexample
\enddesc
\enddescriptions
\end