\input macros
\begindescriptions
\begindesc
\margin{Order of {\tt\\aftergroup} and {\tt\\afterassignment} changed.}
\cts aftergroup {\<token>}
\explain
When \TeX\ encounters this command during input,
it saves \<token>.
After the end of the current \minref{group},
it inserts \<token> back into the input and expands it.
If a group contains several |\aftergroup|s, the corresponding tokens
are \emph{all} inserted following the end of the group, in the order
in which they originally appeared.

The example that follows shows how you can use |\aftergroup| to postpone
processing a token that you generate within a \minref{conditional test}.
\example
\def\neg{negative} \def\pos{positive}
% These definitions are needed because \aftergroup applies 
% to a single token, not to a sequence of tokens or even 
% to a brace-delimited text.
\def\arith#1{Is $#1>0$? \begingroup
   \ifnum #1>-1 Yes\aftergroup\pos
   \else No\aftergroup\neg\fi
   , it's \endgroup. }
\arith 2
\arith {-1}
|
\produces
\def\neg{negative} \def\pos{positive}
% These definitions are needed because \aftergroup applies 
% to a single token, not a sequence of tokens or even 
% a group.
\def\arith#1{Is $#1>0$? \begingroup
   \ifnum #1>-1 Yes\aftergroup\pos
   \else No\aftergroup\neg\fi
   , it's \endgroup. }
\arith 2
\arith {-1}
\endexample
\eix^^{groups}
\enddesc
\enddescriptions
\end