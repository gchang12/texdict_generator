\input macros
\begindescriptions
\begindesc
\makecolumns 4/2:
\easy%
\ctsact { \xrdef{@lbrace}
\cts bgroup {}
\ctsact } \xrdef{@rbrace}
\cts egroup {}
\explain
The left and right braces are commands that begin and end a 
\minref{group}.
The |\bgroup| and |\egroup| \minref{control sequence}s are equivalent
to `|{|' and `|}|', except that
\TeX\ treats |\bgroup| and |\egroup| like any other
\minref{control sequence} when it's scanning its input.

|\bgroup| and |\egroup| can be useful when you're 
defining paired macros, one of which 
starts a brace-delimited construct (not necessarily a group)
and the other one of which ends that construct.
^^{macros//using \b\tt\\bgroup\e\ and \b\tt\\egroup\e\ in}
You can't define such macros using ordinary braces---if you try,
your macro definitions will contain unmatched braces
and will therefore be unacceptable to \TeX.
Usually you should use these commands only when you can't use
ordinary braces.

\example
Braces define the {\it boundaries\/} of a group.
|
\produces
Braces define the {\it boundaries\/} of a group.
\nextexample
\def\a{One \vbox\bgroup}
% You couldn't use { instead of \bgroup  here because
% TeX would not recognize the end of the macro
\def\enda#1{{#1\egroup} two}
% This one is a little tricky, since the \egroup actually
% matches a left brace and the following right brace
% matches the \bgroup.  But it works!!
\a \enda{\hrule width 1in}
|
\produces
\def\a{One \vbox\bgroup}
% You couldn't use { instead of \bgroup  here because
% TeX would not recognize the end of the macro
\def\enda#1{{#1\egroup} two}
% This one is a little tricky, since the \egroup actually 
% matches a left brace and the following right brace
% matches the \bgroup.  But it works!
\a \enda{\hrule width 1in}
\endexample
\enddesc
\enddescriptions
\end