\input macros
\beginconcepts
\concept file

A \defterm{file} is a stream of information that \TeX\ interprets or
creates.  Files are managed by the ^{operating system} that supervises your
\TeX\ run.  \TeX\ deals with files in four different contexts:
\olist
\li A ``^{source file}'' is one that \TeX\ reads with its ``eyes''
\seeconcept{\anatomy} and interprets according to its ordinary rules.
Your primary input file---the one you specify after `|**|' or
on the command line when
you invoke \TeX---is a source file, and so is any file that you call for
with an ^|\input| command \ctsref \input.

\li A ``^{result file}'' is one that contains the results of 
running \TeX.  A \TeX\ run creates two result files: the
\dvifile\ and the log file.
^^{\dvifile//as a result file}
^^{log file//as a result file}
The \dvifile\ contains the information needed to print your document;
the
log file contains a record of what happened during the run, including any
error messages that \TeX\ generated.
If your primary source file is named
|screed.tex|, your \dvifile\ and log file will be named |screed.dvi|
and |screed.log|.\footnote{This is the usual convention, but
particular implementations of \TeX\ are free to change it.}

\li To read from a file with the ^|\read|
command \ctsref{\read} you need to associate the file with an input stream.
^^{input streams//reading with \b\tt\\read\e}
You can have up to $16$ input streams active
at once, numbered $0$--$15$.  
The |\read| command reads a single line and makes it the value of a
designated \refterm{control sequence}, so reading with 
|\read| is very different from reading with ^|\input| (which brings in an
entire file).
\TeX\ takes any input stream number not between
$0$ and $15$ to refer to the terminal,
so `|\read16|', say, reads the next line that you type at the terminal.

\li To write to a file with the |\write|
command \ctsref \write\ you need to associate the file
with an output stream.
^^|\write//output stream for|
^^{output streams}
You can have up to $16$ output streams active
at once, numbered $0$--$15$.  
Input and output streams are independent.
Anything sent to an output stream with a negative number goes to the log
file; anything sent to an output stream with a number greater than $15$
goes both to the log file and to the terminal.
Thus `|\write16|', say, writes a line on the terminal and also sends
that line
to the log~file.

\endolist

You must open a stream file before you can use it.
An input stream file is opened with an ^|\openin|
command  \ctsref \openin\ and an output stream file is opened with an
^|\openout| command \ctsref\openout.
For tidiness
you should close a stream file when you're done with it, although
\TeX\ will do that at the end of the run if you don't.
The two commands for closing a stream file are ^|\closein| \ctsref\closein\
and ^|\closeout| \ctsref\closeout.
An advantage of closing a stream when
you're done with it is that you can then reuse the stream for a different file.
Doing this can be essential when you're reading a long sequence of files.

Although you can assign numbers yourself to input and output streams,
it's better to do it with the ^|\newread| and
^|\newwrite| \ctsref{\@newwrite} commands.
You can have more than one stream associated with a particular file,
but you'll get (probably undiagnosed) garbage unless all of the streams
are input streams.  Associating more than one stream with an input file
can be useful when you want to use the same input file for two different
purposes.

\TeX\ ordinarily defers the actions of opening, writing to, or closing
an output stream until it ships out a page with ^|\shipout|
(see \knuth{page~227}
for the details).  This behavior applies even to messages written to the
terminal with |\write|.  But you can get \TeX\ to perform an action
on an output stream immediately by preceding the action command with
^|\immediate| \ctsref\immediate.  For example:
\csdisplay
\immediate\write16{Do not pass GO!! Do not collect $200!!}
|
\endconcept



\endconcepts
\end