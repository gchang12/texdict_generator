\input macros
\beginconcepts
\concept badness

The \defterm{badness} of a line is a measure of how far the interword
spaces ^^{interword spacing}
in the line deviate from their natural values,
i.e., the values specified in the \refterm{fonts:font} used in the line.
The greater the
deviation, the greater the badness.  Similarly, the badness of a page is
a measure of how far the spaces between the boxes that
make up the page deviate from their ideal values.  (Ordinarily, most of these
boxes are single lines of paragraphs.)

More precisely, the badness
is a measure of how much the \refterm{glue} associated with these spaces needs
to stretch or shrink to fill the line or page exactly.
\TeX\ computes the badness as approximately $100$
times the cube of the ratio by which it must stretch or shrink the glue
in order to compose a line or a page of the required size.
^^{line breaks//badness for}^^{page breaks//badness for}
For example, stretching the glue by twice its stated stretch yields a ratio of
$2$ and a badness of $800$; stretching it by half its stated stretch yields
a ratio of $.5$ and a badness of $13$.
\TeX\ treats a badness greater than $10000$ as
equal to $10000$.

\TeX\ uses the badness of a line when it's breaking a paragraph into lines
\seeconcept{line break}.  It uses this information in two stages:

\olist
\li When \TeX\ is choosing line breaks,
it will eventually accept lines whose badness is less than or equal to
the value of |\tolerance| (\xref \tolerance).  If \TeX\ cannot avoid setting
a line whose badness exceeds this
value, it will set it as an underfull or overfull \refterm{hbox}.
\TeX\ will set
an overfull or underfull hbox only as a last resort, i.e., only if there's no
other way to break the paragraph into lines.
\li Assuming that all lines are tolerably bad, \TeX\ uses the badness of lines
in order to evaluate the different ways of breaking the paragraph into lines.
During this evaluation it associates ``demerits'' with each potential line.
The badness increases the number of \refterm{demerits}.
\TeX\ then
breaks the paragraph into lines in a way that minimizes the 
total demerits for the paragraph.
Most
often \TeX\ arranges the paragraph in a way that minimizes the badness of the
worst line.  See \knuth{pages~97--98} for the details of how \TeX\
breaks a paragraph into lines.  
\endolist

\TeX's procedure for assembling a sequence of lines and other vertical
mode material into pages is similar to its procedure for line breaking.
However, assembling pages is
not as complicated because \TeX\ only considers one page at a time
when it looks for page breaks.
Thus the only decision it must make is where to end the current page.
In contrast, when \TeX\ is choosing line breaks it 
considers several of them simultaneously.
(Most word processors choose line breaks one at a time,
and thus don't do as good a job at it as \TeX\ does.)
See \knuth{pages~111--113} for the details of how \TeX\ chooses its
page breaks.
\endconcept



\endconcepts
\end