\input macros
\beginconcepts
\concept character

{\tighten
\TeX\ works with \defterm{characters} in two contexts:
as input characters, which it reads, and as output characters,
which it typesets.
\TeX\ transforms most input characters
into the output characters that depict them.
For example, it normally
transforms the input letter `|h|' into the letter `h' typeset in the current
font.
That is not the case, however, for an input character such as `|$|' that has a
special meaning.
\par}

\TeX\ gets its input characters by reading them from input files (or from your
terminal) and by expanding \refterm{macros:macro}.  These are the
\emph{only} ways that \TeX\ can acquire an input character.
Each input character has a code number corresponding to its position in the
\refterm{\ascii} code table.  ^^{\ascii}
For instance, the letter `|T|' has \ascii\ code~$84$.

When \TeX\ reads
a character, it attaches a \refterm{category code}
^^{category codes//attached during input}
to it.  The category code affects how \TeX\ interprets the
character once it has been read in.  \TeX\ determines
(and remembers) the category codes of the characters in a macro when it
reads the macro's definition.  As \TeX\ reads characters with its eyes
\seeconcept{\anatomy} it does some ``filtering'',
such as condensing
sequences of spaces to a single space.  See \knuth{pages~46--48} for the
details of this filtering.

The \ascii\
``^{control characters}'' have codes $0$--$31$ and $127$--$255$.
They either don't
show up or cause strange behavior on most terminals if you try to
display them.  Nonetheless they are sometimes needed in \TeX\ input,
so \TeX\ has a special notation for them.
\xrdef{twocarets}
If you type `|^^|$c$', where $c$ is any character, you get the character
whose \ascii\ code is either $64$ greater or $64$ less than $c$'s
\ascii\ code.  The largest acceptable code value using this notation
is $127$, so the notation is unambiguous.
Three particularly common instances of this
notation are `|^^M|' (the \ascii\ \asciichar{return} character),
`|^^J|' (the \ascii\ \asciichar{line\ feed} character) and `|^^I|'
(the \ascii\ \asciichar{horizontal\ tab} character).
\ttidxref{^^M}\ttidxref{^^J}\ttidxref{^^I}

{\tighten
\TeX\ also has another notation for indicating \ascii\ code values
that works for all character codes from $0$ to $255$.
\xrdef{hexchars}
If you type `|^^|$xy$', where $x$
and $y$ are any of the ^{hexadecimal digit}s `|0123456789abcdef|',
you get the single character with the specified code.
(Lowercase letters are required here.)
\TeX\ opts for the ``hexadecimal digits''
interpretation whenever it has a choice, so you must not follow a character
like `|^^a|' with a lowercase hexadecimal digit---if you do, you'll get the
wrong interpretation.
If you need to use this
notation you'll find it handy to have a table of \ascii\ codes.
\par}

An output character is a character to be typeset.
A command for producing an output character has the meaning
``Typeset
character number $n$ from the current \refterm{font}'',
where $n$ is determined by the command.
\TeX\ produces your typeset document by combining such characters
with
other typographical \hbox{elements} in boxes, and arranging them 
on the page.

An input character whose category code is $11$ (^{letter}) or $12$ (other)
^^{other characters}
acts as a command to produce the corresponding output character.  In
addition you can get \TeX\ to produce character $n$ by issuing the
command `|\char |$n$' \ctsref{\char}, ^^|\char| where $n$ is a
\refterm{number} between $0$ and $255$.  The commands `|h|',
|\char`h|, and |\char104| all have the same effect.  ($104$ is the
\ascii\ code for `h'.)

\endconcept


\endconcepts
\end