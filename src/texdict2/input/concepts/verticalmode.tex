\input macros
\beginconcepts
\concept {vertical mode}

^^{vboxes//vertical mode for} When \TeX\ is assembling either a
\refterm{vbox} or the main vertical list from which pages are derived,
it is in one of two \defterm{vertical modes}: ^{ordinary
vertical mode} for assembling the main vertical list, and ^{internal
vertical mode} for assembling vboxes.  Whenever \TeX\ is in a vertical mode
its stomach \seeconcept{\anatomy} is constructing a \refterm{vertical
list} of items (boxes, glue, penalties, etc.).
\TeX\ typesets the items in the list
one below another, top to bottom.

A vertical list can't contain any items produced by 
inherently horizontal commands, e.g.,
^^{vertical lists//can't contain horizontal commands}
|\hskip| or an ordinary (nonspace) character.
\footnote{\TeX\ \emph{ignores} any space characters
that it encounters while it's in a vertical mode.}

\ulist
\li If \TeX\ is  assembling a vertical list in ordinary vertical mode and
encounters an inherently horizontal command, it switches to ordinary
\refterm{horizontal mode}. 
\li If \TeX\ is  assembling a vertical list in internal vertical mode and
encounters an inherently horizontal command, it complains.
\endulist

Two commands that you might at first think are inherently vertical are
in fact inherently horizontal: |\valign| \ctsref{\valign} and |\vrule|
\ctsref{\vrule}.
^^|\valign//inherently horizontal|
^^|\vrule//inherently horizontal|
See \knuth{page~283} for a list of the
inherently horizontal commands.

It's particularly important to be aware that \TeX\ considers an ordinary
character other than a space character to be inherently horizontal.  If
\TeX\ suddenly starts a new paragraph when you weren't expecting it,
a likely cause is a
character that \TeX\ encountered while in vertical mode.
You can convince \TeX\
not to treat that character as inherently horizontal by enclosing it in
an \refterm{hbox} since the |\hbox| command, despite its name, is
neither inherently horizontal nor inherently vertical.
\endconcept


\endconcepts
\end