\input macros
\beginconcepts
\concept {category code}

The \defterm{category code} of a \refterm{character} determines that
character's role in \TeX.
^^{characters//category code of}
For instance, \TeX\ assigns a certain role to
letters, another to space characters, and so forth.  \TeX\ attaches a
code to each character that it reads.  When \TeX\ reads the
letter `|r|', for example, it ordinarily
attaches the category code $11$ (letter)
to it. For simple \TeX\ applications you won't need to worry about
category codes, but they become important when you are trying to achieve
special effects.

Category codes apply only to characters that \TeX\ reads from input
files.  Once a character has gotten past \TeX's ^{gullet} 
\seeconcept{\anatomy} and been interpreted, its category code no
longer matters.  A character that you produce with the ^|\char| command
\ctsref{\char} does not have a category code because |\char|
is an instruction to \TeX\ to produce a certain character in a certain
\refterm{font}.  For instance, the ^{\ascii} code for `|\|' 
(the usual escape character) is $92$.  If
you type `|\char92 grok|', it is \emph{not} equivalent to |\grok|.
Instead it tells \TeX\ to
typeset `$c$\kern.075em grok', where $c$ is the character in position $92$
of the code table for the current font.

You can use the ^|\catcode| command \ctsref{\catcode} to reassign the
category code of any character.  By changing category codes you can
change the roles of various characters.  For instance, if you type
`|\catcode`\@ = 11|', the category code of the at sign (|@|) will be set
to ``letter''.  You then can use `|@|' in the name of a control
sequence.

Here is a list of the category codes defined by \TeX,
(see \xref{twocarets} for an explanation of
the |^^| notation),
together with the characters in each category (as assigned
by \TeX\ and \refterm{\plainTeX}):

\xrdef{catcodes}
\vskip\abovedisplayskip
%k \vskip 0pt plus 2pt % to fix bad page break
{%k \interlinepenalty = 10000
\halign{\indent\hfil\strut#&\qquad#\hfil\cr
\it Code&\it Meaning\cr
\noalign{\vskip\tinyskipamount}
0&Escape character \quad |\| ^^{escape character//category code of}
   {\recat!ttidxref[\//category code of]]
   \cr
1&Beginning of group \quad |{| ^^{groups}
   {\recat!ttidxref[{//category code of]]
   \cr
2&End of group   \quad |}|
   {\recat!ttidxref[}//category code of]]
   \cr
3&Math shift   \quad |$| ^^{math shift}
   {\recat!ttidxref[$//category code of]]
   \cr
4&Alignment tab   \quad |&| ^^{tabs} ^^{alignments//tab character for}
   \ttidxref{&//category code of} \cr
5&End of line   \quad |^^M| \tequiv \ascii\ \asciichar{return}
   ^^{end of line} \ttidxref{^^M//category code of}\cr
6&Macro parameter   \quad |#|
   ^^{macros//parameters of}
   ^^{parameters//indicated by \b\tt\#\e}
   \ttidxref{#//category code of} \cr
7&Superscript   \quad |^| and |^^K| ^^{superscripts}
   \ttidxref{^^K}
   \ttidxref{^//category code of} \cr
8&Subscript   \quad |_| and |^^A| ^^{subscripts}
   \ttidxref{^^A}
   \ttidxref{_//category code of} \cr
9&Ignored character  \quad |^^@| \tequiv \ascii\ \asciichar{null}
  ^^{ignored characters} \indexchar ^^@ \cr
10&Space  \quad \visiblespace\ and |^^I| \tequiv \ascii\ 
  \asciichar{horizontal\ tab}
  ^^{horizontal tab}
  ^^{space characters//category code of} \indexchar ^^I
  {\recat!ttidxref[ ]]
  \cr
11&Letter  \quad |A| \dots |Z| and |a| \dots |z| ^^{letter}\cr
12&Other character \quad (everything not listed above or below) 
  ^^{other characters}\cr
13&Active character  \quad |~| and |^^L| \tequiv\ascii\ \asciichar{form\ feed} 
   ^^{active characters} ^^{form feed} \indexchar ~ \indexchar ^^L \cr
14&Comment character  \quad |%| ^^{comments}
   {\recat!ttidxref[%//category code of]]
   \cr
15&Invalid character  \quad |^^?| \tequiv \ascii\ \asciichar{delete}
  ^^{invalid character} \indexchar ^^? \cr
}}
\vskip\belowdisplayskip
%k \vskip 0pt plus 2pt % to fix bad page break

\noindent Except for categories $11$--$13$,
all the characters in a particular category produce the same effect.
\margin{Misleading material removed.}
For instance, suppose
that you type:
\csdisplay
\catcode`\[ = 1 \catcode`\] = 2
|
Then the left and right bracket characters become
beginning-of-group and end-of-group characters equivalent to
the left and right brace characters.  With these definitions `|[a b]|'
is a valid group, and so are \hbox{`|[a b}|'} and~\hbox{`|{a b]|'}.

The characters in categories $11$ (letter) and $12$ 
(other character) act as \refterm{commands:command}
that mean
``typeset the character with this code from the current font''.
The only distinction between letters and ``other'' characters is
that letters can appear in \refterm{control word}s but
``other'' characters~can't.

A character in category $13$ (active) acts like a control sequence
all by itself.  \TeX\ complains if it encounters an active character that
doesn't have a definition associated with it.
^^{active characters}

If \TeX\ encounters an ^{invalid character} (category $15$) 
in your input, it will complain about it. 

The `|^^K|' and `|^^A|' characters have been included in categories 
$8$ (subscript) and $9$ (superscript), even though these meanings
don't follow the standard \refterm{\ascii} interpretation.
That's because some keyboards, notably some at Stanford
University where \TeX\ originated,
have down arrow and up arrow keys that generate these characters.
\ttidxref{^^A}
\ttidxref{^^K}

There's a subtle point about the way \TeX\ assigns category codes that
can trip you up if you're not aware of it.  \TeX\ sometimes needs to
look at a character twice as it does its initial scan: first to find the
end of some preceding construct, e.g., a control sequence, and later to
turn that character into a token.  \TeX\ doesn't assign the category
code until its \emph{second} look at the character.  For example:

\csdisplay
\def\foo{\catcode`\$ = 11 }% Make $ be a letter.
\foo$ % Produces a `$'.
\foo$ % Undefined control sequence `\foo$'.
|
\noindent
This bit of \TeX\ code produces `\$' in the typeset output.  When
\TeX\ first sees the `|$|' on the second line,
it's looking for the end of a control sequence name.  Since
the `|$|' isn't yet a letter, it marks the end of `|\foo|'.  Next,
\TeX\ expands the `|\foo|' macro and changes the category code of `|$|'
to $11$ (letter).  Then \TeX\ reads the `|$|' ``for real''.  Since 
`|$|' is now a letter, \TeX\ produces a box
containing the `|$|' character in the current font.
When \TeX\ sees the third line, it treats `|$|' as a letter and thus
considers it to be part of the control sequence name.
As a result it complains about an undefined control sequence |\foo$|.
 
\TeX\ behaves this way even when the terminating character is an
end of line.  For example, suppose that the macro |\fum| activates the
end-of-line character.  Then if |\fum| appears on a line $\ell$ by
itself, \TeX\ will first interpret the end of line of $\ell$ as 
the end of the |\fum| control sequence and then will \emph{reinterpret}
the end of line of $\ell$ as an active character.
\endconcept



\endconcepts
\end