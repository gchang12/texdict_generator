\input macros
\beginconcepts
\concept {ASCII}

\defterm{\ascii} is the abbreviation of ``American Standard Code for
Information Interchange''.  There are $256$ \ascii\
^{characters}, each with its own code number, but
the meanings of only the first~$128$ have been standardized.  You can
find these meanings in an \ascii\ ``code table'' such as the one on
\knuth{page~367}.  Characters $32$--$126$ are ``printable characters'',
^^{printable characters} such as letters, numbers, and punctuation
marks. The remaining characters are ``^{control characters}'' that are
typically used (in the computer industry, not in \TeX) to control
input\slash output and data communications devices.  For instance,
\ascii\ code $84$ corresponds to the letter `T', while \ascii\
code~$12$ corresponds to the ``form feed'' function (interpreted by most
printers as ``start a new page'').  Although the \ascii\ standard
specifies meanings for the control characters, many manufacturers of
equipment such as modems and printers have used the control characters
for purposes other than the standard ones.

The meaning of a
character in \TeX\ is usually consistent with its meaning in standard \ascii,
and \refterm{fonts:font} that contain \ascii\
printable characters usually have those characters in the same positions as
their \ascii\ counterparts.
But some fonts, notably those used for math, replace the \ascii\
printable characters by other characters unrelated to the \ascii\ characters.
For instance, the Computer Modern math font
^^{Computer Modern fonts}
|cmsy10| has the math symbol
`{$\forall$}' in place of the \ascii\ digit `8'.


\endconcept


\endconcepts
\end