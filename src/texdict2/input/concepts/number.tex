\input macros
\beginconcepts
\concept number

In \TeX, a \defterm{number} is a positive or negative integer.
You can write a number in \TeX\ in four different ways:
\olist\compact
\li as an ordinary decimal integer, e.g., |52|
\li as an octal number, e.g., |'14| ^^{octal numbers}
\li as a hexadecimal number, e.g., |"FF0| ^^{hexadecimal numbers}
\li as the code for an \refterm{\ascii\ character}, e.g., |`)|
or |`\)|
\endolist
\noindent 
Any of these forms can be preceded by `|+|' or `|-|'.

An octal number can have only the digits |0|--|7|.
^^{octal numbers}
A hexadecimal number can have digits |0|--|9| and
|A|--|F|, representing
values from $0$ to $15$.
^^{hexadecimal numbers}
You can't, alas, use lowercase letters when you write a hexadecimal number.
If you need an explanation of octal and hexadecimal numbers,
you'll find one on \knuth{pages~43--44}.

A decimal, octal, or hexadecimal number 
ends at the first character that can't be part of the number.
Thus a decimal number ends when \TeX\ sees, say, a letter, even though a
letter between `|A|' and `|F|' would not end a hexadecimal number.
You can end a number with one or more spaces and
\TeX\ will ordinarily ignore them.\footnote{
When you're defining a macro that ends in a number, you should always
put a space after that number; otherwise \TeX\ may later combine that
number with something else.}

The fourth form above specifies a number as the 
\minref{\ascii} code for a character.
^^{characters//\ascii\ codes for}
\TeX\ ignores spaces after this form of number also.
You can write a number in this form either as |`|$c$ or as |`\|$c$.
The second form, though longer, has the advantage that you can use it
with \emph{any} character, even `|\|', `|%|', or `|^^M|'.
It does have one rather technical disadvantage: when \TeX\ is expanding
a token sequence for a command such as |\edef| or |\write|,
^^|\edef//expansion of {\tt\\'\it c} in|
^^|\write//expansion of {\tt\\'\it c} in|
occurrences of `|\|$c$' within numbers will also be expanded if they can be.
That's rarely the effect you want.

The following are all valid representations of the decimal number
$78$:
\csdisplay
78   +078   "4E   '116   `N   `\N
|


You can't use a number in text by itself since a number isn't
a command.
However, you can insert the decimal form of a number in text
by putting a ^|\number| command (\xref\number) in front of it
or the roman numeral form by putting a ^|\romannumeral| command
in front of it.

You can also use ^{decimal constant}s, i.e., numbers with a fractional part,
for specifying dimensions \seeconcept{dimension}.
A decimal constant has a ^{decimal point}, which
can be the first character of the constant.  
You can use a comma instead of a period to represent the decimal point.
A decimal constant can be preceded by a plus or minus sign.
Thus `|.5in|', 
`|-3.22pt|', and `|+1,5\baselineskip|' are valid dimensions.
You can't, however, use decimal constants
in any context \emph{other} than as the ``factor'' part of a dimension,
i.e., its multiplier.


\endconcept


\endconcepts
\end