\input macros
\beginconcepts
\concept register

A \defterm{register} is a named location for storing a value.
It is much like a variable in a programming language.
\TeX\ has five kinds of registers, as shown in the following table:

\vdisplay{\tabskip 10pt\halign{\tt #\hfil &#\hfil\cr
{\it Register type}&{\it Contents}\cr
box&a \refterm{box} \idxref{box registers}\cr
count&a \refterm{number} \idxref{count registers}\cr
dimen&a \refterm{dimension} \idxref{dimension registers}\cr
muskip&\refterm{muglue} \idxref{muglue registers}\cr
skip&\refterm{glue} \idxref{glue registers}\cr
toks&a \refterm{token} list\idxref{token registers}\cr}}

The registers of each type are numbered from $0$ to $255$.
You can access register $n$ of category $c$ by using the form `|\|$cn$',
e.g., |\muskip192|.
You can use a register
anywhere that information of the appropriate type is called for.  For
instance, you can use |\count12|
in any context calling for a number or |\skip0|
in any context calling for glue.

You put information into a register by \refterm{assigning:assignment}
something to it:

\csdisplay
\setbox3 = \hbox{lagomorphs are not mesomorphs}
\count255 = -1
|
The first assignment constructs an hbox and assigns it to
box register~$3$.
You can
subsequently use `|\box3|' wherever a box is called for, and you will
get just that hbox.\footnote{But note carefully: using a box register
also empties it so that its contents become void.  The other kinds of
registers don't behave that way. You can use the |\copy| command
\ctsref{\copy} to retrieve the contents of a box register without
emptying it.}
The second assignment assigns $-1$ to count register~$255$.

A register of a given type, e.g., a glue register, behaves just like
a parameter of that type.
^^{parameters//like registers}
You retrieve its value or assign to it
just as you would with a \refterm{parameter}.
Some \TeX\ parameters, e.g., |\pageno|,
are implemented as registers, in fact.

\PlainTeX\
uses many registers for its own purposes, so you should not just
pick an arbitrary
register number when you need a register.  Instead you should ask
\TeX\ to reserve a register by using one of the commands
^|\newbox|, ^|\newcount|, ^|\newdimen|, ^|\newmuskip|, ^|\newskip|,
or ^|\newtoks|
\ctsref{\@newbox}.  These commands are outer, so you can't
use them in a macro definition.
If you could,
you'd use up a register every time the macro was called and probably run out
of registers before long.

Nonetheless you can with some caution use any register temporarily
within a \refterm{group}, even one that \TeX\ is using for something
else.
After \TeX\ finishes executing the commands in a group,
it restores the contents of every register
to what they were before it started executing the group.
When you use an explicitly numbered register inside a group,
you must be sure that the register isn't modified by any
\refterm{macro}
that you might call within the group.
Be especially careful
about using arbitrary registers in a group that calls macros 
that you didn't write yourself.

{\tighten
\TeX\ reserves certain registers for special purposes: |\count0| through
|\count9| for page numbering information and
^^{page numbering}
^|\box255| for the contents
of a page just before it is offered to the \refterm{output routine}.
Registers |\dimen0|--|\dimen9|, |\skip0|--|\skip9|,
|\muskip0|--|\mu!-skip9|, |\box0|--|\box9|,
and the |255| registers other than |\box255|
are generally available as ``scratch'' registers.
Thus \plainTeX\ provides only one scratch register, |\count255|, for
counts.
See \knuth{pages~122 and 346} for conventions to follow
in choosing register numbers.
\par}

You can examine the contents of registers during a \TeX\ run with the
^|\showthe| command \ctsref\showthe, e.g., with `|\showthe\dimen0|'.
\endconcept


\endconcepts
\end