\input macros
\beginconcepts
\concept macro

{% Use a brace here so that definitions of explanatory macros remain local.
% The closing brace is at the end of the concept.
A \defterm{macro} is a definition that gives a name to a pattern of
\TeX\ input text.\footnote{More precisely, the definition gives a name
to a sequence of tokens.} The name can be either a \refterm{control
sequence} or an \refterm{active character}.  The pattern is called the
``replacement text''.  The primary command for defining macros is the
|\def| control sequence.

\def\arctheta{\cos \theta + i \sin \theta}
As a simple example, suppose that you have a document in which
the sequence `$\cos \theta + i \sin \theta$' occurs many times.
Instead of writing it out each time, you can define a macro for it:
\csdisplay
\def\arctheta{\cos \theta + i \sin \theta}
|
Now whenever you need this sequence, you can just ``call'' the macro
by writing `|\arctheta|'
and you'll get it.  For example, `|$e^{\arctheta}$|' will give you
`$e^{\arctheta}$'.

\bix^^{macros//parameters of}
But the real power of macros lies in the fact that a macro can have
parameters.  When you call a macro that has parameters, you provide 
arguments that are substituted for those parameters.  For example, suppose
you write:
\pix\indexchar #
\def\arc#1{\cos #1 + i \sin #1}
\csdisplay
\def\arc#1{\cos #1 + i \sin #1}
|

The notation |#1| \xrdef{@msharp} indicates the first parameter
of the macro, which in this case has only one parameter.  You now can
produce a similar form, such as `$\arc{2t}$', with the macro call `|\arc
{2t}|'.

More generally, a macro can have up to nine parameters, which you
indicate as `|#1|', `|#2|', etc\null. in the macro definition.  \TeX\
provides two kinds of parameters: delimited parameters and undelimited
parameters.  Briefly, a delimited parameter has an \refterm{argument}
that's delimited, or ended, by a
specified sequence of tokens (the delimiter), while an undelimited
parameter has an argument that doesn't need a delimiter to end it.
First we'll explain how macros work when they have only undelimited
parameters, and then we'll explain how they work when
they have delimited parameters.

^^{parameters//undelimited}
If a macro has only undelimited parameters, those parameters must appear
one after another in the macro definition \emph{with nothing between
them or between the last parameter and the left brace in front of the 
replacement text}.
A call on such a macro consists of the macro name followed by
the arguments of the call, one for each parameter.  Each argument is
either:

\ulist \compact
\li a single \refterm{token} other than a left or right brace, or

\li a sequence of tokens enclosed between a left brace and
a matching right brace.\footnote{The
argument can have nested pairs of braces within it, and each of these
pairs can indicate either a \refterm{group} or a further macro
argument.}
\endulist

When \TeX\ encounters a macro, it expands the macro in its gullet
\seeconcept{\anatomy}
by substituting each argument for the corresponding
parameter in the replacement text.  The resulting text may contain other macro
calls.  When \TeX\ encounters such an embedded macro call, it expands
that call immediately without looking at what follows the
call.\footnote{In computer science terminology, the expansion is ``depth
first'' rather than ``breadth first''.  Note that you can modify the
order of expansion with commands such as |\expandafter|.} When \TeX's
gullet gets to a \refterm{primitive} \refterm{command} that
cannot be further expanded, \TeX\ passes that command to \TeX's stomach.
The order of expansion is sometimes critical, so in order to help
you understand it we'll give you an example of \TeX\ at work.

Suppose you provide \TeX\ with the following input:
\csdisplay
\def\a#1#2{\b#2#1\kern 2pt #1}
\def\b{bb}
\def\c{\char49 cc}
\def\d{dd}
\a\c{e\d} % Call on \a.
|
Then the argument corresponding to |#1| is |\c|,
and the argument corresponding to |#2| is |e\d|.
\TeX\ expands the macro call in the following steps:

{\vskip\abovedisplayskip\obeylines % ugly
|\b e\d\c\kern 2pt \c|
|bbe\d\c\kern 2pt \c|
|\d\c\kern 2pt \c|\quad(`|b|', `|b|', `|e|' sent to stomach)
|dd\c\kern 2pt \c|
|\c\kern 2pt \c|\quad(`|d|', `|d|' sent to stomach)
|\char49 cc\kern 2pt \c|
|\c|\quad(`|\char|', `|4|', `|9|', `|c|', `|c|', %
`|\kern|', `|2|', `|p|', `|t|' sent to stomach)
|\char49 cc|
(`|\char49|', `|c|', `|c|' sent to stomach)
\vskip\belowdisplayskip}

\noindent Note that the letters `|b|', `|c|', `|d|', and `|e|' and the
control sequences `|\kern|' and `|\char|' are all primitive
commands that cannot be expanded further.

\bix^^{parameters//delimited}
A macro can also have ``delimited parameters'', which can be mixed with
the undelimited ones in any combination.  The idea of a delimited
parameter is that \TeX\ finds the corresponding argument by looking for
a certain sequence of tokens that marks the end of the argument---the
delimiter.  That is, when \TeX\ is looking for such an argument, it
takes the argument to be all the tokens from \TeX's current position up
to but not including the delimiter.

You indicate a delimited parameter by writing `|#|$n$' ($n$ 
must be between $0$
and $9$) followed by one or more tokens that act as the delimiter.  The
delimiter extends up to the next `|#|' or `|{|'---which makes sense
since `|#|' starts another parameter and `|{|' starts the replacement text.

The delimiter can't be `|#|' or `|{|', so you can tell a delimited
parameter from an undelimited one by looking at what comes after it. 

If the character after the parameter is `|#|' or `|{|', you've got an
undelimited parameter; otherwise you've got a delimited one.  Note
the difference in arguments for the two kinds of parameters---an
undelimited parameter is matched either by a single token or by
a sequence of tokens enclosed in braces, while a
delimited parameter is matched by any number of tokens, even zero.

An example of a macro that uses two delimited parameters is:
\def\diet#1 #2.{On #1 we eat #2!}
\csdisplay
\def\diet#1 #2.{On #1 we eat #2!!}
|
Here the first parameter is delimited by a single space
and the second parameter is delimited by a period.  If you write:
\csdisplay
\diet Tuesday turnips.
|
you'll get the text ``\diet Tuesday turnips.''.
But if the delimiting tokens are enclosed in a group, \TeX\ doesn't consider
them as delimiting.  So if you write:
\csdisplay
\diet {Sunday mornings} pancakes.
|
you'll get the text `\diet {Sunday mornings} pancakes.'
even though there's a space between `|Sunday|' and `|morning|'.
When you use a space as a delimiter,
an end-of-line character ordinarily also delimits the argument
since \TeX\ converts the end-of-line to a space before the macro
mechanism ever sees it.
\eix^^{parameters//delimited}
\eix^^{macros//parameters of}

Once in a while you might need to define a macro that has `|#|' as a
meaningful character within it.
You're most likely to need to do this when you're defining a macro
that in turn defines a second macro.
What then do you do about
the parameters of the second macro to avoid getting \TeX\ confused?
The answer is that you write
two `|#|'s for every one that you want
when the first macro is expanded.  For example, suppose you 
write the macro definition:
\def\first#1{\def\second##1{#1/##1}}
\csdisplay
\def\first#1{\def\second##1{#1/##1}}
|
Then the call `|\first{One}|' defines `|\second|' as:
\csdisplay
\def\second#1{One/#1}
|
and the subsequent call `|\second{Two}|' produces the text
\def\second#1{One/#1}%
`\second {Two}'.

A number of commands provide additional ways of defining macros
(see pp.~\xrefn{mac1}--\xrefn{mac2}).
For the complete rules pertaining to macros, see \knuth{Chapter~20}.
}% close brace at the start of the `macro' concept.
\endconcept


\endconcepts
\end