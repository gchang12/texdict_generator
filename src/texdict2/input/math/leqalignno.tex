\input macros
\begindescriptions
\begindesc
\cts eqalign {}
   {{\bt \rqbraces{\<line> \\cr $\ldots$ \<line> \\cr}}}
\cts eqalignno {}
   {{\bt \rqbraces{\<line> \\cr $\ldots$ \<line> \\cr}}}
\cts leqalignno {}
   {{\bt \rqbraces{\<line> \\cr $\ldots$ \<line> \\cr}}}
\explain
^^{equation numbers}
These commands produce a multiline math display
in which certain corresponding parts of the lines are lined up vertically.
The |\eqalignno| and |\leqalignno| commands also let you
provide equation numbers for some or all of the lines.
|\eqalignno| puts the equation numbers on the right and
|\leqalignno| puts them on the left.

Each line in the display is ended by |\cr|.  Each of the parts to be aligned
(most often an equals sign) is preceded by
`|&|'.  An `|&|' also precedes each equation number, which comes at the
end of a line.
You can put more than one of these commands in a single display in order
to produce several groups of equations.  In this case, only the rightmost
or leftmost group can be produced by |\eqalignno| or |\leqalignno|.

You can use the |\noalign| command (\xref \noalign) to change the amount
of space between two lines of a multiline display.
\example
$$\left\{\eqalign{f_1(t) &= 2t\cr f_2(t) &= t^3\cr
         f_3(t) &= t^2-1\cr}\right\}
  \left\{\eqalign{g_1(t) &= t\cr g_2(t) &= 1}\right\}$$
|
\dproduces
$$\left\{\eqalign{f_1(t) &= 2t\cr f_2(t) &= t^3\cr
   f_3(t) &= t^2-1\cr}\right\}
\left\{\eqalign{g_1(t) &= t\cr g_2(t) &= 1}\right\}$$
\nextexample
$$\eqalignno{
\sigma^2&=E(x-\mu)^2&(12)\cr
   &={1 \over n}\sum_{i=0}^n (x_i - \mu)^2&\cr
   &=E(x^2)-\mu^2\cr}$$
|
\produces
\abovedisplayskip = -\baselineskip
$$\eqalignno{
\sigma^2&=E(x-\mu)^2&(12)\cr
   &={1 \over n}\sum_{i=0}^n (x_i - \mu)^2&\cr
   &=E(x^2)-\mu^2\cr}$$
\nextexample
$$\leqalignno{
\sigma^2&=E(x-\mu)^2&(6)\cr
   &=E(x^2)-\mu^2&(7)\cr}$$
|
\produces
\abovedisplayskip = -\baselineskip
$$\leqalignno{
\sigma^2&=E(x-\mu)^2&(6)\cr
   &=E(x^2)-\mu^2&(7)\cr}$$
\nextexample
$$\eqalignno{
  &(x+a)^2 = x^2+2ax+a^2&(19)\cr
  &(x+a)(x-a) = x^2-a^2\cr}$$
% same effect as \displaylines but with an equation number
|
\dproduces
$$\eqalignno{
&(x+a)^2 = x^2+2ax+a^2&(19)\cr
&(x+a)(x-a) = x^2-a^2\cr
}$$
% same effect as \displaylines but with an equation number

\eix^^{displays//multiline}
\endexample
\enddesc
\enddescriptions
\end