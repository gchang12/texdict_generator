\input macros
\begindescriptions
\begindesc
\xrdef{specsyms}
\dothreecolumns 34
\easy\ctsdisplay infty {}
\ctsdisplay Re {}
\ctsdisplay Im {}
\ctsdisplay angle {}
\ctsdisplay triangle {}
\ctsdisplay backslash {}
\ctsdisplay vert {}
\ctsydisplay | @bar {}
\ctsdisplay Vert {}
\ctsdisplay emptyset {}
\ctsdisplay bot {}
\ctsdisplay top {}
\ctsdisplay exists {}
\ctsdisplay forall {}
\ctsdisplay hbar {}
\ctsdisplay ell {}
\ctsdisplay aleph {}
\ctsdisplay imath {}
\ctsdisplay jmath {}
\ctsdisplay nabla {}
\ctsdisplay neg {}
\ctsdisplay lnot {}
\actdisplay ' @prime \ (apostrophe)
\ctsdisplay prime {}
\ctsdisplay partial {}
\ctsdisplay surd {}
\ctsdisplay wp {}
\ctsdisplay flat {}
\ctsdisplay sharp {}
\ctsdisplay natural {}
\ctsdisplay clubsuit {}
\ctsdisplay diamondsuit {}
\ctsdisplay heartsuit {}
\ctsdisplay spadesuit {}
\egroup
\explain
^^{music symbols} ^^{card suits}
These commands produce various symbols.  They are called
``^{ordinary symbol}s'' to distinguish them from other classes of
symbols such as relations. You can only use 
an ordinary symbol
within a math formula, so if you need an ordinary symbol within ordinary text
you must enclose it in dollar signs (|$|).

The commands |\imath| and |\jmath| are useful when you need to put an
accent on top of an `$i$' or a `$j$'.

An apostrophe (|'|) is a short way of writing a superscript |\prime|.  (The
|\prime| command by itself generates a big ugly prime.)

The |\!|| and ^|\Vert| commands are synonymous, as
are the ^|\neg| and ^|\lnot| commands.
\margin{explanation of {\tt\\vert} added}
The |\vert| command produces the same result as `|!||'.
\indexchar |

The symbols produced by |\backslash|, |\vert|, and |\Vert|
are \minref{delimiter}s.  These symbols can be produced in larger sizes
by using ^|\bigm| et al.\ (\xref \bigm).  

\example
The Knave of $\heartsuit$s, he stole some tarts.
|
\produces
The Knave of $\heartsuit$s, he stole some tarts.
\nextexample
If $\hat\imath < \hat\jmath$ then $i' \leq j^\prime$.
|
\produces
If $\hat\imath < \hat\jmath$ then $i' \leq j^\prime$.
\nextexample
$${{x-a}\over{x+a}}\biggm\backslash{{y-b}\over{y+b}}$$
|
\dproduces
$${{x-a}\over{x+a}}\biggm\backslash{{y-b}\over{y+b}}$$
\endexample
\enddesc
\enddescriptions
\end