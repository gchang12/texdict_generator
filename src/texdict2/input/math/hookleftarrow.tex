\input macros
\begindescriptions
\begindesc
\bix^^{arrows}
\xrdef{arrows}
%
{\symbolspace=24pt \makecolumns 34/2:
\easy%
\ctsdisplay leftarrow {}
\ctsdisplay gets {}
\ctsdisplay Leftarrow {}
\ctsdisplay rightarrow {}
\ctsdisplay to {}
\ctsdisplay Rightarrow {}
\ctsdisplay leftrightarrow {}
\ctsdisplay Leftrightarrow {}
\ctsdisplay longleftarrow {}
\ctsdisplay Longleftarrow {}
\ctsdisplay longrightarrow {}
\ctsdisplay Longrightarrow {}
\ctsdisplay longleftrightarrow {}
\ctsdisplay Longleftrightarrow {}
\basicdisplay {$\Longleftrightarrow$}{\\iff}\pix\ctsidxref{iff}\xrdef{\iff}
\ctsdisplay hookleftarrow {}
\ctsdisplay hookrightarrow {}
\ctsdisplay leftharpoondown {}
\ctsdisplay rightharpoondown {}
\ctsdisplay leftharpoonup {}
\ctsdisplay rightharpoonup {}
\ctsdisplay rightleftharpoons {}
\ctsdisplay mapsto {}
\ctsdisplay longmapsto {}
\ctsdisplay downarrow {}
\ctsdisplay Downarrow {}
\ctsdisplay uparrow {}
\ctsdisplay Uparrow {}
\ctsdisplay updownarrow {}
\ctsdisplay Updownarrow {}
\ctsdisplay nearrow {}
\ctsdisplay searrow {}
\ctsdisplay nwarrow {}
\ctsdisplay swarrow {}
}
\explain
These commands provide arrows of different kinds.  They
are classified as relations (\xref{relations}).
The vertical arrows in the list are also \minref{delimiter}s, so you can make
them larger by using |\big| et al.\ (\xref \big).

The command |\iff| differs from |\Longleftrightarrow| in that
it produces extra space to the left and right of the arrow.

You can place symbols or other legends on top of a left or right arrow
with |\buildrel| (\xref \buildrel).

\example
$$f(x)\mapsto f(y) \iff x \mapsto y$$
|
\dproduces
$$f(x)\mapsto f(y) \iff x \mapsto y$$

\eix^^{arrows}
\endexample
\enddesc
\enddescriptions
\end