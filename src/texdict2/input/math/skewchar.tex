\input macros
\begindescriptions
\begindesc
\cts skewchar {\<font>\param{number}}
\explain
The |\skewchar| of a font
is the character in the font whose kerns,
as defined in the font's metrics file, determine the positions
of math accents. That is, suppose that \TeX\ is applying a math accent
to the character `|x|'.  \TeX\ checks if the character pair
`|x\skewchar|' has a kern; if so, it moves the accent by the amount of
that kern. The complete algorithm that \TeX\ uses to position math
accents (which involves many more things) is in \knuth{Appendix~G}.

If the value of |\skewchar| is not in the range $0$--$255$,
\TeX\ takes the kern value to be zero.

Note that \<font> is a control sequence
that names a font, not a \<font\-name> that names font files.
Beware: 
an assignment to |\skewchar| is \emph{not} undone at the end
of a group.
If you want to change |\skewchar| locally, you'll need to
save and restore its original value explicitly.
\enddesc
\enddescriptions
\end