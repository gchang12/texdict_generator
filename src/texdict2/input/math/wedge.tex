\input macros
\begindescriptions
\begindesc
\bix^^{operations}
\xrdef{binops}
\dothreecolumns 34
\easy\ctsdisplay vee {}
\ctsdisplay wedge {}
\ctsdisplay amalg {}
\ctsdisplay cap {}
\ctsdisplay cup {}
\ctsdisplay uplus {}
\ctsdisplay sqcap {}
\ctsdisplay sqcup {}
\ctsdisplay dagger {}
\ctsdisplay ddagger {}
\ctsdisplay land {}
\ctsdisplay lor {}
\ctsdisplay cdot {}
\ctsdisplay diamond {}
\ctsdisplay bullet {}
\ctsdisplay circ {}
\ctsdisplay bigcirc {}
\ctsdisplay odot {}
\ctsdisplay ominus {}
\ctsdisplay oplus {}
\ctsdisplay oslash {}
\ctsdisplay otimes {}
\ctsdisplay pm {}
\ctsdisplay mp {}
\ctsdisplay triangleleft {}
\ctsdisplay triangleright {}
\ctsdisplay bigtriangledown {}
\ctsdisplay bigtriangleup {}
\ctsdisplay ast {}
\ctsdisplay star {}
\ctsdisplay times {}
\ctsdisplay div {}
\ctsdisplay setminus {}
\ctsdisplay wr {}
\egroup
\explain
These commands produce the symbols for various binary operations.
Binary operations are one of \TeX's \minref{class}es of math symbols.
\TeX\ puts different amounts of space around different classes of math
symbols.  When \TeX\ needs to break a line of text within a math
formula, \minrefs{line break} it will consider placing the break
after a binary operation---but only if
the operation is at the outermost level of
the formula, i.e., not enclosed in~a~group.

In addition to these commands, \TeX\ also treats `|+|' and `|-|'
as binary operations. It considers `|/|' to be an ordinary symbol,
despite the fact that mathematically it is a binary operation,
because it looks better with less space around it.

\example
$$z = x \div y \quad \hbox{if and only if} \quad
z \times y = x \;\hbox{and}\; y \neq 0$$
|
\dproduces
$$z = x \div y \quad \hbox{if and only if} \quad
z \times y = x \;\hbox{and}\; y \neq 0$$
\endexample
\enddesc
\enddescriptions
\end