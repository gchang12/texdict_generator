\input macros
\begindescriptions
\begindesc
\cts kern {\<dimen>}
\explain
The effect of this command depends on the mode that \TeX\ is in when 
it encounters it:
\ulist

\li In a horizontal mode, \TeX\ moves its position to the right (for a positive
kern) or to the left (for a negative kern).

\li In a vertical mode, \TeX\ moves its position down the page (for a positive
kern) or up the page (for a negative kern).

\endulist
\noindent
Thus a positive kern produces empty space while a negative kern
causes \TeX\ to back up over something that it's already produced.
This notion of a kern ^^{kerns}
is different from the notion of a kern in some computerized
typesetting systems---in \TeX, positive kerns push two letters \emph{apart}
instead of bringing them closer together.

A kern is similar to
\minref{glue}, except that (a)~a kern can neither stretch nor shrink,
and (b)~\TeX\ will only break a line or a page at a kern if the kern
is followed by glue and is not part of a math formula.
If \TeX\ finds a kern at the
end of a line or a page, it discards the kern.
If you want to get the effect of a kern that never disappears,
use ^|\hglue| or ^|\vglue|.

You can use |\kern| in math mode, but you can't use |mu| units
\seeconcept{mathematical unit}
for \<dimen>.  If you want |mu| units, use |\mkern|
(\xref\mkern) instead.
^^{line breaks//kerns at}
^^{page breaks//kerns at}

\example
\centerline{$\Downarrow$}\kern 3pt % a vertical kern
\centerline{$\Longrightarrow$\kern 6pt % a horizontal kern
   {\bf Heed my warning!!}\kern 6pt % another horizontal kern
   $\Longleftarrow$}
\kern 3pt % another vertical kern
\centerline{$\Uparrow$}
|
\produces
\centerline{$\Downarrow$}\kern 3pt % a vertical kern
\centerline{$\Longrightarrow$\kern 6pt % a horizontal kern
   {\bf Heed my warning!}\kern 6pt % another horizontal kern
   $\Longleftarrow$}
\kern 3pt % another vertical kern
\centerline{$\Uparrow$}
\endexample
\enddesc
\enddescriptions
\end