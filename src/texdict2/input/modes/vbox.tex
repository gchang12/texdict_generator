\input macros
\begindescriptions
\begindesc
\cts vtop {\<vertical mode material>}
\aux\cts vtop {{\bt to} \<dimen> \rqbraces{\<vertical mode material>}}
\aux\cts vtop {{\bt spread} \<dimen> \rqbraces{\<vertical mode material>}}
\cts vbox {\rqbraces{\<vertical mode material>}}
\aux\cts vbox {{\bt to} \<dimen> \rqbraces{\<vertical mode material>}}
\aux\cts vbox {{\bt spread} \<dimen> \rqbraces{\<vertical mode material>}}
\explain
^^{hbox//constructing with \b\tt\\hbox\e}
These commands
produce a \minref{vbox} (vertical \minref{box})
containing \<vertical mode material>.
The braces around \<vertical mode material> define a group.
\TeX\ is in internal vertical mode when it's assembling the box.
\TeX\ won't change the size of the box once it's been produced.

The difference between |\vtop| and |\vbox| lies in where \TeX\ puts
the reference point of the constructed vbox.
Ordinarily, the reference point gotten from |\vtop| tends to be at or near
the top of the constructed vbox,
while
the reference point gotten from |\vbox| tends to be at or near
the bottom of the constructed vbox.
Thus a row of vboxes all constructed with |\vtop|
will tend to have their tops nearly in a line,
while a row of vboxes all constructed with |\vbox|
will tend to have their bottoms nearly in a line.

|\vtop| and |\vbox| are often useful
when you want to keep some text together on a single page.
(For this purpose, it usually doesn't matter which command you use.)
If your use of these commands
prevents \TeX\ from breaking pages in an acceptable
way, \TeX\ will complain that it's found an overfull or underfull vbox while
|\output| is active.

The height of a vbox depends on 
the arguments to |\vtop| or |\vbox|.
For |\vbox|, \TeX\ determines the height as follows:
\ulist\compact
\li If you specify only \<vertical mode material>,
the vbox will have its natural height.
\li If you specify |to| \<dimen>,
the height of the vbox will be \<dimen>.
\li If you specify |spread| \<dimen>, the height of the vbox will be
its natural height plus \<dimen>, i.e.,
the height of the vbox will be stretched vertically by \<dimen>.
\endulist
\noindent
For |\vtop|,
\TeX\ constructs the box using its rules for |\vbox| and then
apportions the vertical extent between the height and the depth as
described below.

Ordinarily, the width of a constructed vbox is the width of the widest
item inside it.\footnote
{More precisely, it's the distance from the reference point to the rightmost
edge of the constructed vbox.  Therefore,
if you move any of the items right using ^|\moveright| or
^|\moveleft| (with a negative distance),
the constructed vbox might be wider.}
The rules for apportioning the vertical extent between the
height and the depth are more complicated:
\ulist
\li For |\vtop|,
the height is the height of its first item, if that item is a box or rule.
Otherwise the height is zero. The depth is whatever vertical
extent remains after the height is subtracted.
\li For |\vbox|,
the depth is the depth of its last item, if that item is a box or rule.
Otherwise the depth is zero. The height is whatever vertical
extent remains after the depth is subtracted.%
\footnote{In fact, there's a further complication.
Suppose that after the depth has been determined
using the two rules just given, the depth turns out to be greater than
^|\boxmaxdepth|.
Then the depth is reduced to |\boxmaxdepth| and the height is adjusted
accordingly.} 
\endulist

The |\vfil| command (\xref\vfil) is useful for filling out a vbox
^^|\vfil//filling a vbox|
with empty space when the material in the box isn't as tall as
the vertical extent of the box.
\example
\hbox{\hsize = 10pc \raggedright\parindent = 1em
\vtop{In this example, we see how to use vboxes to
produce the effect of double columns.  Each vbox
contains two paragraphs, typeset according to \TeX's
usual rules except that it's ragged right.\par
This isn't really the best way to get true double
columns because the columns}
\hskip 2pc
\vtop{\noindent
aren't balanced and we haven't done anything to choose
the column break automatically or even to fix up the
last line of the first column.\par
However, the technique of putting running text into a
vbox is very useful for placing that text where you
want it on the page.}}
|
\produces
\hbox{\hsize = 10pc \raggedright\parindent = 1em
\vtop{In this example, we see how to use vboxes to
produce the effect of double columns.  Each vbox
contains two paragraphs, typeset according to \TeX's
usual rules except that it's ragged right.\par
This isn't really the best way to get true double
columns because the columns}
\hskip 2pc
\vtop{\noindent
aren't balanced and we haven't done anything to choose
the column break automatically or even to fix up the
last line of the first column.\par
However, the technique of putting running text into a
vbox is very useful for placing that text where you
want it on the page.}}
\nextexample
\hbox{\hsize = 1in \raggedright\parindent = 0pt
\vtop to .75in{\hrule This box is .75in deep. \vfil\hrule}
\qquad
\vtop{\hrule This box is at its natural depth. \vfil\hrule}
\qquad
\vtop spread .2in{\hrule This box is .2in deeper than
                  its natural depth.\vfil\hrule}}
|
\produces
\hbox{\hsize = 1in \raggedright\parindent=0pt
\vtop to .75in{\hrule This box is .75in deep. \vfil\hrule}
\qquad
\vtop{\hrule This box is at its natural depth. \vfil\hrule}
\qquad
\vtop spread .2in{\hrule This box is .2in deeper than
                  its natural depth.\vfil\hrule}}
\nextexample
% See how \vbox lines up boxes at their bottoms
% instead of at their tops.
\hbox{\hsize = 1in \raggedright
\vbox to .5in{\hrule This box is .5in deep.\vfil\hrule}
\qquad
\vbox to .75in{\hrule This box is .75in deep.\vfil\hrule}}
|
\produces
\hbox{\hsize = 1in \raggedright
\vbox to .5in{\hrule This box is .5in deep.\vfil\hrule}
\qquad
\vbox to .75in{\hrule This box is .75in deep.\vfil\hrule}}
\vskip 16pt % to avoid running into the next command description
\endexample
\enddesc
\enddescriptions
\end