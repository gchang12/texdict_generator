\input macros
\begindescriptions
\begindesc
\cts halign {{\bt \rqbraces{\<preamble> \\cr \<row> \\cr $\ldots$ \<row> \\cr}}}
   \xrdef{@and}\xrdef{@pound}
\aux\cts halign {{\bt to \<dimen>%
   \rqbraces{\<preamble> \\cr \<row> \\cr $\ldots$ \<row> \\cr}}}
\aux\cts halign {{\bt spread \<dimen>%
   \rqbraces{\<preamble> \\cr \<row> \\cr $\ldots$ \<row> \\cr}}}
\explain
This command produces a horizontal \minref{alignment} consisting of a
sequence of rows, where each row in turn contains a sequence of column
entries.  \TeX\ adjusts the widths of the column entries to accommodate
the widest one in each column.

A horizontal alignment can only appear when \TeX\ is in a vertical
\minref{mode}.  We recommend that you first study alignments in general
(\xref{alignment}) before you attempt to use this command.

An alignment consists of a ^{preamble}
followed by the text to be aligned. The preamble,
which describes the layout of the rows that follow, consists of a
sequence of column templates, separated by `|&|' and ended
by |\cr|.
\bix^^{template}
\bix^^{entry (column or row)}
Each row consists of a sequence of column
entries, also separated by `|&|' and ended by |\cr|. Within
a template, `|#|' indicates where \TeX\ should insert the
corresponding text of a column entry. 
In contrast, |\settabs| uses a fixed implicit template of `|#|',
i.e., it just inserts the text as is.

\TeX\ typesets each column entry in restricted horizontal mode,
i.e., as the contents of an \minref{hbox},
and implicitly encloses it in a group.

The |to| form of this command instructs \TeX\
to make the width of the alignment be \<dimen>,
adjusting the space between columns as necessary.
The |spread| form of this command instructs \TeX\
to make the alignment wider by \<dimen> than its natural width.
These forms are like the corresponding forms of |\hbox| \ctsref\hbox.

See |\tabskip| \ctsref\tabskip{} for an example using the
|to| form.

\example
\tabskip = 1em \halign{%
   \hfil\it#\hfil&\hfil#\hfil&#&\hfil\$#\cr
   United States&Washington&dollar&1.00\cr
   France&Paris&franc&0.174\cr
   Israel&Jerusalem&shekel&0.507\cr
   Japan&Tokyo&yen&0.0829\cr}
|
\produces
\tabskip = 1em \halign{%
   \hfil\it#\hfil&\hfil#\hfil&#&\hfil\$#\cr
   United States&Washington&dollar&1.00\cr
   France&Paris&franc&0.174\cr
   Israel&Jerusalem&shekel&0.507\cr
   Japan&Tokyo&yen&0.0829\cr}
\endexample
\enddesc
\enddescriptions
\end