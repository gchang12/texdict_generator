\input macros
\begindescriptions
\begindesc
\easy\cts indent {}
\explain
If \TeX\ is in vertical mode, as it is after ending a paragraph,
this command inserts the ^|\parskip| interparagraph glue,
puts \TeX\ into horizontal mode, starts a paragraph, and
indents that paragraph by |\parindent|.
If \TeX\ is already in horizontal mode, this command merely produces
a blank space of width |\parindent|.
Two |\indent|s in a row 
produce two indentations.
^^{indentation}

As the example below shows, an |\indent| at a point where \TeX\
would start a paragraph anyway is redundant.
When \TeX\ is in vertical mode and sees a letter or some other
inherently horizontal command, it starts a paragraph by
switching to horizontal mode,
doing an |\indent|, and processing the horizontal command.

\example
\parindent = 2em  This is the first in a series of three 
paragraphs that show how you can control indentation. Note
that it has the same indentation as the next paragraph.\par
\indent This is the second in a series of three paragraphs.
It has \indent an embedded indentation.\par
\indent\indent This doubly indented paragraph
is the third in the series.
|
\produces
\parindent = 2em  This is the first in a series of three 
paragraphs that show how you can control indentation. Note
that it has the same indentation as the next paragraph.\par
\indent This is the second in a series of three paragraphs.
It has \indent an embedded indentation.\par
\indent\indent This doubly indented paragraph
is the third in the series.
\endexample
\enddesc
\enddescriptions
\end