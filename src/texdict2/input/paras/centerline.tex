\input macros
\begindescriptions
\begindesc
\bix^^{centering}
\bix^^{flush left}
\bix^^{flush right}
\bix^^{justification}
\easy\cts centerline {\<argument>}
\cts leftline {\<argument>}
\cts rightline {\<argument>}
\explain
The |\centerline| command produces an \minref{hbox} exactly as wide
as the current line and places \<argument> at the center of the box.
The |\leftline| and |\rightline| commands are analogous; they
place \<argument> at the left end or at the right end of the box.
If you want to apply one of these commands to
several consecutive lines, you must apply
it to each one individually.
See \xrefpg{eplaincenter} for an alternate approach.

Don't use these commands within a paragraph---if you do, 
\TeX\ probably won't be able to break the paragraph into lines and
will complain about an overfull hbox.
\example
\centerline{Grand Central Station}
\leftline{left of Karl Marx}
\rightline{right of Genghis Khan}
|
\produces
\centerline{Grand Central Station}
\leftline{left of Karl Marx}
\rightline{right of Genghis Khan}

\eix^^{centering}
\eix^^{flush left}
\eix^^{flush right}
\eix^^{justification}

\endexample
\enddesc
\enddescriptions
\end