\input macros

\begindescriptions

\begindesc
\cts vcenter {\rqbraces{\<vertical mode material>}}
\ctsbasic {\\vcenter to \<dimen> \rqbraces{\<vertical mode material>}}{}
\ctsbasic {\\vcenter spread \<dimen> \rqbraces{\<vertical mode material>}}{}
\explain
Every math formula has an invisible
``^{axis}'' that \TeX\ treats as a kind of
horizontal centering line for that formula.
For instance, the axis of a formula consisting of a
fraction is at the center of the fraction bar.
The |\vcenter| command tells \TeX\ to place the \<vertical mode material>
in a \minref{vbox} and to center the vbox
with respect to the axis of the formula it is currently constructing.

The first form of the command
centers the material as given.  The second and third
forms expand or shrink the material vertically as in the |\vbox| command
(\xref \vbox).

\example
$${n \choose k} \buildrel \rm def \over \equiv \>
\vcenter{\hsize 1.5 in \noindent the number of
combinations of $n$ things taken $k$ at a time}$$
|
\dproduces
$${n \choose k} \buildrel \rm def \over \equiv \>
\vcenter{\hsize 1.5 in \noindent the number of
combinations of $n$ things taken $k$ at a time}$$
\endexample
\enddesc

\enddescriptions
\end