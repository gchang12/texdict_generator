\input macros

\begindescriptions

\begindesc
\cts futurelet {\<control sequence> \<token$_1$> \<token$_2$>}
\explain
This command tells \TeX\ to make \<token$_2$> the meaning of
\<control sequence> (as would be done with |\let|), and then to
process \<token$_1$> and \<token$_2$> normally.
|\futurelet| is useful at the end of macro definitions
because it gives you a way of looking beyond the token that \TeX\
is about to process before it processes it.
\example
\def\predict#1{\toks0={#1}\futurelet\next\printer}
% \next will acquire the punctuation mark after the
% argument to \predict
\def\printer#1{A \punc\ lies ahead for \the\toks0. }
\def\punc{%
\ifx\next;semicolon\else
\ifx\next,comma\else
``\next''\fi\fi}
\predict{March}; \predict{April}, \predict{July}/
|
\produces
\def\predict#1{\toks0={#1}\futurelet\next\printer}
\def\printer#1{A \punc\ lies ahead for \the\toks0. }
\def\punc{%
\ifx\next;semicolon\else
\ifx\next,comma\else
``\next''\fi\fi
}
\predict{March};
\predict{April},
\predict{July}/
\endexample
\enddesc

\enddescriptions
\end