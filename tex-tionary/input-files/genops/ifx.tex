\input macros

\begindescriptions

\begindesc
\ctspecial ifx {\<token$_1$> \<token$_2$>}\ctsxrdef{@ifx}
\explain
This command tests if \<token$_1$> and \<token$_2$> agree.
Unlike |\if| and |\ifcat|, |\ifx| does \emph{not} expand the tokens
following |\ifx|, so \<token$_1$> and \<token$_2$> are the two
tokens immediately after |\ifx|.
There are three cases:
\olist
\li If one token is a \minref{macro} and the other one isn't,
the tokens don't agree.
\li If neither token is a macro, the tokens agree if:
\olist
\li both tokens are characters (or control sequences denoting characters) and
their \minref{character} codes and \minref{category code}s agree, or
\li both tokens refer to the same \TeX\ command,
font, etc.
\endolist
\li If both tokens are macros, the tokens agree if:
\olist\compact
\li their ``first level'' expansions, i.e.,
their replacement texts, are identical, and
\li they have the same status with respect to ^|\long| (\xref\long)
and ^|\outer| (\xref\outer).
\endolist
Note in particular that \emph{any two undefined control
sequences agree}.
\endolist
\noindent
This test is generally more useful than |\if|.
\example
\ifx\alice\rabbit true\else false\fi;
% true since neither \rabbit nor \alice is defined
\def\a{a}%
\ifx a\a true\else false\fi;
% false since one token is a macro and the other isn't
\def\first{\a}\def\second{\aa}\def\aa{a}%
\ifx \first\second true\else false\fi;
% false since top level expansions aren't the same
\def\third#1:{(#1)}\def\fourth#1?{(#1)}%
\ifx\third\fourth true\else false\fi
% false since parameter texts differ
|
\produces
\ifx\alice\rabbit true\else false\fi;
% true since neither \rabbit nor \alice is defined
\def\a{a}%
\ifx a\a true\else false\fi;
% false since one token is a macro and the other isn't
\def\first{\a}\def\second{\aa}\def\aa{a}%
\ifx \first\second true\else false\fi;
% false since top level expansions aren't the same
\def\third#1:{(#1)}\def\fourth#1?{(#1)}%
\ifx\third\fourth true\else false\fi
% false since parameter texts differ
\endexample
\enddesc

\enddescriptions
\end