\input macros

\begindescriptions

\begindesc
\cts topskip {\param{glue}}
\explain
\TeX\ inserts glue at the top of each
page in order to ensure that the baseline of the first box on the page
always is the same distance $d$ from the top of the page.
|\topskip| determines the amount of that glue,
called the ``|\topskip| glue'', by specifying
what $d$ should be (provided that the first box
on the page isn't too tall).
$d$ is given by the natural size of the |\topskip| glue.
If the height of the first box on the page exceeds $d$,
so that the glue would be negative, \TeX\ simply inserts no
|\topskip| glue at all on that page.

To understand better the effect of these rules, assume that |\topskip|
has no stretch or shrink and that the first item on the page is indeed a box.
Then if the height of that box is no greater than |\topskip|,
its baseline will be |\topskip|
from the top of the page independently of its height.  On the other hand,
if the height of the box is $e$ greater than |\topskip|, its baseline will be
|\topskip|\tplus$e$ from the top of the page.
See \knuth{pages~113--114} for the remaining details of how
|\topskip| works.
\PlainTeX\ sets |\topskip| to |10pt|.
\enddesc

\enddescriptions
\end