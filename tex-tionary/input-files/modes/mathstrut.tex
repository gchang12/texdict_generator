\input macros

\begindescriptions

\begindesc
\cts mathstrut {}
\explain
This command produces a phantom formula whose width is zero
and whose height and depth are the same as those of a left parenthesis.
|\mathstrut| is in fact defined as `|\vphantom(|'.
Its main use is for getting radicals, underbars, and overbars to line
up with other radicals, underbars, and overbars in a formula.
It is much like ^|\strut| (\xref \strut),
except that it adjusts itself to the different \minref{style}s
that can occur in math formulas.
\example
$$\displaylines{
\overline{a_1a_2} \land \overline{b_1b_2}
\quad{\rm versus}\quad \overline{a_1a_2\mathstrut}
\land \overline{b_1b_2\mathstrut}\cr
\sqrt{\epsilon} + \sqrt{\xi} \quad{\rm versus}\quad
\sqrt{\epsilon\mathstrut} + \sqrt{\xi\mathstrut}\cr}$$
|
\dproduces
\kern 4pt
$$\displaylines{
\overline{a_1a_2} \land \overline{b_1b_2}
\quad{\rm versus}\quad \overline{a_1a_2\mathstrut}
\land \overline{b_1b_2\mathstrut}\cr
\sqrt{\epsilon} + \sqrt{\xi} \quad{\rm versus}\quad
\sqrt{\epsilon\mathstrut} + \sqrt{\xi\mathstrut}\cr}$$
\endexample
\eix^^{struts}
\enddesc

\enddescriptions
\end