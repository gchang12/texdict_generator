\input macros

\begindescriptions

\begindesc
\cts hidewidth {}
\explain
This command tells \TeX\ to ignore the width of the next column entry in a
horizontal alignment.  It's useful when you have an entry that is longer
than most of the others in the same column,
and you'd rather have that entry stick out of the column than
make all the entries
in the column wider.  If the |\hidewidth| is at the left of the
entry, the entry sticks out to the left; if the |\hidewidth| is at the
right of the entry, the entry sticks out to the~right.
\example
\tabskip = 25pt\halign{%
\hfil\it#\hfil&\hfil#\hfil&#&\hfil\$#\cr
United States&\hidewidth Washington&
dollar&1.00\cr
France&Paris&franc&0.174\cr
Israel&Jerusalem&shekel&0.507\cr
Japan&Tokyo&yen&0.0829\cr}
|
\produces
\tabskip = 25pt\halign{%
\hfil\it#\hfil&\hfil#\hfil&#&\hfil\$#\cr
United States&\hidewidth Washington&
dollar&1.00\cr
France&Paris&franc&0.174\cr
Israel&Jerusalem&shekel&0.507\cr
Japan&Tokyo&yen&0.0829\cr}
\endexample
\enddesc

\enddescriptions
\end