\input macros

\begindescriptions

\begindesc
\easy\cts hskip {\<dimen$_1$> {\bt plus} \<dimen$_2$> {\bt minus} \<dimen$_3$>}
\cts vskip {\<dimen$_1$> {\bt plus} \<dimen$_2$> {\bt minus} \<dimen$_3$>}
^^{vertical skip}^^{vertical glue}
^^{horizontal skip}^^{horizontal glue}
\bix^^{horizontal space}
\bix^^{vertical space}
\explain
These commands produce horizontal and vertical glue respectively.
In the simplest and most common case when only \<dimen$_1$> is present,
|\hskip| skips to the right by \<dimen$_1$>
and |\vskip| skips down the page by \<dimen$_1$>.
More generally, these commands
produce \minref{glue} whose natural size is
\<dimen$_1$>, whose stretch is \<dimen$_2$>, and whose shrink is \<dimen$_3$>.
Either the |plus| \<dimen$_2$>, the |minus |\<dimen$_3$>,
or both can be omitted.
If both are present, the |plus| must come before the |minus|.
An omitted value
is taken to be zero. Any of the \<dimen>s can be negative.

You can use |\hskip| in math mode, but you can't use |mu| units
\seeconcept{mathematical unit}
for any of the dimensions.  If you want |mu| units, use |\mskip|
(\xref\mskip) instead.

\example
\hbox to 2in{one\hskip 0pt plus .5in two}
|
\produces
\hbox to 2in{one\hskip 0pt plus 2in two}

\doruler{\8\8}2{in}
\nextexample
\hbox to 2in{Help me!! I can't fit
{\hskip 0pt minus 2in} inside this box!!}
|
\produces
\hbox to 2in{Help me! I can't fit
{\hskip 0pt minus 2in} inside this box!}

\doruler{\8\8}2{in}
\nextexample
\vbox to 4pc{\offinterlineskip% Just show effects of \vskip.
\hbox{one}\vskip 0pc plus 1pc \hbox{two}
\vskip .5pc \hbox{three}}
|
\produces
\smallskip
\vbox to 4pc{\offinterlineskip% Just show effects of \vskip.
\hbox{one}\vskip 0pc plus 1pc \hbox{two}
\vskip .5pc \hbox{three}}
\endexample
\enddesc

\enddescriptions
\end