\input macros

\begindescriptions

\begindesc
\ctsact  ^^M \xrdef{@newline}
\explain
This construct produces the ^{end of line} character.
It normally has two effects when \TeX\ encounters it in
your input:
\olist
\li It acts as a command, producing either an input space
(if it comes at the end of a nonblank line)
or a |\par| token (if it comes at the end of a blank line).
^^|\par//from empty line|
\li It ends the input line, causing \TeX\ to ignore the remaining
characters on the line.
\endolist
\noindent
However, |^^M| does \emph{not} end the line when it appears in the
context |`\^^M|, denoting the ASCII code for control-M (the number $13$).
You can change the meaning of |^^M|
by giving it a different \minref{category code}.
See \xrefpg{twocarets} for a more general explanation of the |^^| notation.
\example
Hello.^^MGoodbye.
Goodbye again.\par
The \char `\^^M\ character.\par
% The fl ligature is at position 13 of font cmr10
\number `\^^M\ is the end of line code.\par
Again, \number `^^M is the end of line code,
isn't it? % 32 is the ASCII code for a space
|
\produces
{\catcode `\^ = 7 % disable indexing use within this display
Hello.^^MGoodbye
Goodbye again.\par
The \char `\^^M\ character.\par
\number `\^^M\ is the end of line code.\par
Again, \number `^^M is the end of line code,
isn't it?}
\endexample
\enddesc

\enddescriptions
\end