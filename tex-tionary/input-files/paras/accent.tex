\input macros

\begindescriptions

\begindesc
\cts accent {\<charcode>}
\explain
^^{accents}
This command puts an accent over the character following this command.
The accent is the character at position \<charcode> in the current font.
\TeX\ assumes that the accent has been designed to fit over a character
$1$\thinspace ex high in the same font as the accent.  If the
character to be accented
is taller or shorter, \TeX\ adjusts the position accordingly.  You can
change \minref{font}s between the accent and the next character, thus
drawing the accent character and the character to be accented
from different fonts.  If
the accent character isn't really intended to be
an accent, \TeX\ won't complain; it
will just typeset something ridiculous.
\example
l'H\accent94 otel des Invalides
% Position 94 of font cmr10 has a circumflex accent.
|
\produces
l'H\accent94 otel des Invalides
% Position 94 of font cmr10 has a circumflex accent.
\endexample
\see Math accents (\xref{mathaccent}).
\enddesc

\enddescriptions
\end