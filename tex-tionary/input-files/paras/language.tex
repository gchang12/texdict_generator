\input macros

\begindescriptions

\begindesc
\cts language {\param{number}}
\explain
Different languages have different sets of hyphenation rules.
This parameter determines the set of ^{hyphenation rules} that \TeX\ uses.
By changing |\language| you can get \TeX\
to hyphenate portions of text or entire documents according to the
hyphenation rules appropriate to a particular language.
^^{European languages}
Your ^{local information} about \TeX\ will tell you if any
additional sets of hyphenation rules are available (besides the
ones for English)
and what the appropriate values of |\language| are.
The default value of |\language| is $0$.

\TeX\ sets the current language to $0$ at the start of every paragraph,
and compares |\language| to the current language whenever it adds
a character to the current paragraph.
If they are not the same, \TeX\ adds a ^{whatsit} indicating the
language change.
This whatsit is the clue in later processing that the language rules
should change.
\enddesc

\enddescriptions
\end