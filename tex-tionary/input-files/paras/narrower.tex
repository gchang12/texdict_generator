\input macros

\begindescriptions

\begindesc
\easy\cts narrower {}
\explain
^^{paragraphs//narrow}
This command makes paragraphs narrower, increasing the left and right
margins by |\parindent|, the
current paragraph ^{indentation}.
It achieves this by increasing
both |\leftskip| and |\rightskip| by |\parindent|.
Normally you place |\narrower| at the
beginning of a \minref{group} containing the paragraphs that you want to
make narrower.  If you forget to enclose |\narrower| within a group,
you'll find that all the rest of your document will have narrow
paragraphs.

|\narrower| affects just those paragraphs that end after you invoke it.
If you end a |\narrower| group before you've ended
a paragraph, \TeX\ won't make that paragraph narrower.

\example
{\parindent = 12pt \narrower\narrower\narrower
This is a short paragraph. Its margins are indented
three times as much as they would be
had we used just one ``narrower'' command.\par}
|
\produces
{\parindent = 12pt \narrower\narrower\narrower
This is a short paragraph. Its margins are indented
three times as much as they would be
had we used just one ``narrower'' command.\par}
\endexample\enddesc

\begindesc
\cts leftskip {\param{glue}}
\cts rightskip {\param{glue}}
\explain
These parameters tell \TeX\ how much glue to place
at the left and at the right end of each line of the current
paragraph.  We'll just explain how |\leftskip| works since |\rightskip|
is analogous.

^^{indentation} You can increase the left margin by setting |\leftskip|
to a fixed nonzero \minref{dimension}.  If you give |\leftskip| some
stretch, you can produce ^{ragged left} text, i.e.,
text that has an uneven left margin.

Ordinarily, you should enclose any \minref{assignment} to |\leftskip|
in a \minref{group} together with the affected text
in order to keep its effect from continuing to
the end of your document.  However, it's pointless to change
|\leftskip|'s value inside a group that is in turn
contained within a paragraph---the value of |\leftskip| at the
\emph{end} of a paragraph
is what determines how \TeX\ breaks the paragraph into lines.  \minrefs{line
break}

\example
{\leftskip = 1in The White Rabbit trotted slowly back
again, looking anxiously about as it went, as if it had
lost something.  {\leftskip = 10in % has no effect
It muttered to itself, ``The Duchess!! The Duchess!! She'll
get me executed as sure as ferrets are ferrets!!''}\par}%
|
\produces
{\leftskip = 1in The White Rabbit trotted slowly back
again, looking anxiously about as it went, as if it had
lost something.  {\leftskip = 10in % has no effect
It muttered to itself, ``The Duchess! The Duchess!
She'll get me executed as sure as ferrets are ferrets!''}\par}%
\nextexample
\pretolerance = 10000 % Don't hyphenate.
\rightskip = .5in plus 2em
The White Rabbit trotted slowly back again, looking
anxiously about as it went, as if it had lost something.
It muttered to itself, ``The Duchess!! The Duchess!! She'll
get me executed as sure as ferrets are ferrets!!''
|
\produces
\pretolerance = 10000 % Don't hyphenate.
\rightskip = .5in plus 2em
The White Rabbit trotted slowly back again, looking
anxiously about as it went, as if it had lost something.
It muttered to itself, ``The Duchess! The Duchess! She'll
get me executed as sure as ferrets are ferrets!''
\endexample
\enddesc

\enddescriptions
\end