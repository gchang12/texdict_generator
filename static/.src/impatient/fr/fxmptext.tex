~% This is part of the book TeX for the Impatient.
~% Copyright (C) 2003 Paul W. Abrahams, Kathryn A. Hargreaves, Karl Berry.
~% Copyright (C) 2004 Marc Chaudemanche pour la traduction fran�aise.
~% See file fdl.tex for copying conditions.
% TeX ignores anything on a line after a %
% The next two lines define fonts for the title
\font\xmplbx = cmbx10 scaled \magstephalf
\font\xmplbxti = cmbxti10 scaled \magstephalf
% Now here's the title.
\leftline{\xmplbx Exemple !xmpnum:\quad\xmplbxti Saisir du texte simple}
\vglue .5\baselineskip % skip an extra half line
~\count255 = \pageno
~\xdef\examplepage{\number\count255}
~\markinfo{Exemple 1~: Saisir du texte simple}
~\ifrewritetocfile
~\write\tocfile{\string\tocsectionentry{Saisir du texte simple}{}{\examplepage}}%
~\fi
~^^{notes de pied de page} ^^{commentaires} ^^{ponctuation} ^^{marques de citation}
~^^{caract\`eres espace} ^^{tirets} ^^{paragraphes//terminer un}
~\edef\examplepageno{\number\count255}%
Il est simple de pr\'eparer un texte ordinaire pour \TeX\ puisque
\TeX\ n'a pas besoin de savoir comment couper les lignes dans
votre source. Il traite la fin d'une ligne comme un espace%
\footnote \dag{\TeX\ traite aussi une tabulation	comme un espace, comme
vous le voyez dans ce {\it pied de page}.}.   Si vous ne voulez pas
d'espace ici, mettez un signe de pour%
centage (le caract\`ere de commentaire) \`a la fin de la ligne.
   \TeX\ ignore les espaces en d\'ebut de ligne, et traite
plusieurs          espaces     comme un seul,
m\^eme apr\`es un point.      Vous indiquez un nouveau paragraphe en
sautant une ligne (ou plus d'une ligne).

Quand \TeX\ voit un point suivi d'un espace (ou de la fin de la
ligne, ce qui est \'equivalent), il pense normalement que vous avez
fini une phrase et ins\`ere un petit espace suppl\'ementaire apr\`es le
point.  Il traite les points d'interrogation et d'exclamation de la
m\^eme mani\`ere.   

  Mais les r\`egles de \TeX\ pour rep\'erer un point ont parfois besoin
de r\'eglage. \TeX\ pense qu'une lettre capitale avant un point
ne termine pas une phrase, donc vous devez faire quelque chose de
l\'eg\`erement diff\'erent si, disons, vous \'ecrivez en parlant d'ADN\null.  
% Le \null emp\`eche TeX de percevoir la capitale `N'
% comme \'etant suivie du point.
C'est une bonne id\'ee de relier des mots ensemble dans des r\'ef\'erences
comme ``voir fig.~8'' et dans des noms comme V.~I\null. L\'enine et dans
$\ldots$ pour que \TeX\ ne puise jamais les couper \`a un mauvais endroit
entre deux lignes.  (Les trois points indiquent une ellipse.)

Vous pouvez mettre des citations entre des paires de simples
``quotes'' gauches et droites pour avoir les bonnes marques de
citation doubles gauches et droites.  ``pour des marques de citation
simple et double adjacentes, ins\'erez un `espace fin'\thinspace''. Vous
pouvez avoir des tirets demi-cadratins--comme ceci, et des tirets 
cadratins---comme cela.

\bye % end the document
:::
\xmpheader !xmpnum/{Indentation}% !xmpheaddef
~^^{indentation} ^^{marges} ^^{paragraphes//\'etroit}
\noindent Maintenant, regardons comment contr\^oler l'indentation. 
Si un traitement de texte sait le faire, \TeX\ le sait s\^urement. 
Notez que ce paragraphe n'est pas indent\'e.

Habituellement, on veut soit indenter les paragraphes, soit
laisser de l'espace suppl\'ementaire entre eux. Comme nous n'avons
rien chang\'e maintenant, ce paragraphe est indent\'e.

{\parindent = 0pt \parskip = 6pt
% l'accolade gauche d\'ebute un groupe contenant le texte non indent\'e
Faisons ces deux paragraphes d'une mani\`ere diff\'erente,
sans indentation et avec six points suppl\'ementaires
entre eux.

Voici un autre paragraphe, compos\'e sans indentation. Si nous 
ne mettons pas d'espace entre ces paragraphes, Nous aurons du mal
\`a savoir o\`u finit l'un et o\`u commence l'autre.
\par % Le paragraphe *doit* \^etre termin\'e dans le groupe.
}% L'accolade droite termine le groupe contenant du texte nin indent\'e.

Il est aussi possible d'indenter les deux cot\'es de paragraphes entiers
Les trois paragraphes suivants illustrent cela~:
\smallskip % Procure un peu d'espace suppl\'ementaire ici.
% Skips like this and \vskip below end a paragraph.
{\narrower
``Nous avons indent\'e ce paragraphe des deux cot\'es d'une indentation de
paragraphe. C'est souvent un bon moyen de pr\'esenter de longues citations

``Vous  pouvez faire plusieurs paragraphes de cette fa\c con si vous 
le voulez. C'est le second paragraphe indent\'e simplement.''\par}

{\narrower \narrower Vous pouvez m\^eme faire des paragraphes 
doublement indent\'es si vous en avez besoin. Ceci en est un 
exemple.\par}
\vskip 1pc % Saut vers le bas d'un pica pour s\'eparation visuelle.
Dans ce paragraphe, nous revenons aux marges normales, comme vous
pouvez le voir. Rallongeons la sauce pour que les marges soient 
clairement visibles.

{\leftskip .5in Maintenant, nous indentons la marge de gauche d'un 
demi pouce et laissons la marge droite \`a sa position usuelle\par}
{\rightskip .5in Finalement, nous indentons la marge droite d'un 
demi pouce et laissons la marge gauche \`a sa position usuelle.\par}
\bye % end the document
:::
\xmpheader !xmpnum/{Polices et caract\`eres sp\'eciaux}% !xmpheaddef
\chardef \\ = `\\ % Let \\ denote a backslash.
~^^{polices} ^^{caract\`eres//sp\'eciaux} ^^{accents}
~^^{symboles musicaux} ^^{couleurs de carte \`a jouer}
~^^|$| ^^|&| ^^|#| ^^|_| ^^|%| ^^|^| ^^|~| ^^|{| ^^|}| \indexchar \
{\it Voici quelques mots en police italique}, {\bf quelques 
autres en police grasse}, {\it et un\/ {\bf m\'elange} 
des deux, avec trois\/ {\rm mots en romain} ins\'er\'es}.
Quand une police italique est suivie par une non italique, on 
ins\`ere une ``correction d'italique'' ({\tt \\/}) pour que 
l'espacement rende bien.
Voici un mot plus {\sevenrm petit}---mais les polices standard
de \TeX\ ne vous en donneront pas de plus petit que {\fiverm celui-l\`a}.

Si vous voulez un de ces dix caract\`eres~:
\medskip
\centerline{\$ \quad \& \quad \# \quad \_ \quad \% \quad
   \char `\^ \quad \char `\~ \quad $\{$ \quad
   $\}$ \quad $\backslash$}
% Le \quad ins\`ere un espace cadratin entre les caract\`eres.
\medskip
\noindent  vous devez les \'ecrire d'une mani\`ere sp\'eciale. Regardez 
la page d'en face pour savoir comment.
\TeX\ a les accents et les lettres dont vous avez besoin
pour des mots fran\c cais comme {\it r\^ ole\/} et  {\it \'
el\` eve\/}, pour des mots allemands comme {\it Schu\ss\/}
aussi bien que pour des mots de plein d'autres langues.
Vous trouverez une liste compl\`ete des accents de \TeX\ et des lettres 
de langues Europ\'eennes \`a la !xrefdelim[fornlets] et \`a la !xrefdelim[accents].

Vous trouverez aussi des lettres Grecques comme ``$\alpha$'' et
``$\Omega$'' pour les utiliser en math\'ematique, des couleurs de carte 
comme ``$\spadesuit$'' et ``$\diamondsuit$'', des symboles de musique
comme ``$\sharp$'' et ``$\flat$'', et plein d'autres symboles
sp\'eciaux dont vous trouverez la liste \`a la !xrefdelim[specsyms].
\TeX\ n'acceptera ces sortes de symboles sp\'eciaux que dans son
``mode math\'ematique'', donc vous devez les entourer
entre caract\`eres `{\tt \$}'.
\bye % end the document
:::
\xmpheader !xmpnum/{Espacement interligne}% !xmpheaddef
~^^{espacement//inter-ligne} ^^{lignes de base}
\baselineskip = 2\baselineskip % double spacing
\parskip = \baselineskip % Skip a line between paragraphs.
\parindent = 3em % Increase indentation of paragraphs.

% The following macro definition gives us nice inline
% fractions.  You'll find it in our eplain macros.
\def\frac#1/#2{\leavevmode
   \kern.1em \raise .5ex \hbox{\the\scriptfont0 #1}%
   \kern-.1em $/$%
   \kern-.15em \lower .25ex \hbox{\the\scriptfont0 #2}%
}%

Un jour, vous voudrez imprimer un document avec de l'espace
suppl\'ementaire entre les lignes. Par exemple, les actes du congr\`es
sont imprim\'es ainsi pour que les l\'egislateurs puissent les annoter.
Pour la m\^eme raison, les \'editeurs insistent habituellement pour que 
les manuscrits aient un espacement double. L'espacement double est n\'eammoins
rarement appropri\'e pour les documents finis.

Une ligne de base est une ligne imaginaire qui agit comme les
lignes horizontales d'un papier quadrill\'e. Vous pouvez contr\^oler
l'espacement interligne---ce que les imprimeurs appellent 
``interlignage''---en fixant la taille de l'espace entre les lignes de
base. Regardez la source pour voir comment faire. Vous pouvez
utiliser la m\^eme m\'ethode pour un espacement de $1\;1/2$, en mettant
{\tt 1.5} au lieu de {\tt 2}. (Vous pouvez aussi \'ecrire $1\frac 1/2$
d'une meilleure fa\c con.)
% Here we've used the macro definition given above.

Pour cet exemple, nous avons aussi augment\'e l'indentation de 
paragraphe et saut\'e une ligne de plus entre les paragraphes.
\bye % end the document
:::
\xmpheader !xmpnum/{Espacement, traits et bo\^\i tes}% !xmpheaddef
~^^{listes de description} ^^{\boites//dessiner des} ^^{barre de r\'evision}
Voici un exemple de ``liste descriptive''. En pratique vous
feriez mieux de creer une macro pour avoir des constructions r\'ep\'etitives
et \^etre s\^ur que la largeur des sous-titres soit suffisament grande~:
\bigskip
% Call the indentation for descriptions \descindent
% and set it to 8 picas.
\newdimen\descindent \descindent = 9pc
% Indent paragraphs by \descindent.
% Skip an additional half line between paragraphs.
{\noindent \leftskip = \descindent  \parskip = .5\baselineskip
% Move the description to the left of the paragraph.
\llap{\hbox to \descindent{\bf La reine de C\oe ur\hfil}}%
Une femme \`a moiti\'e folle, prompte a d\'eclarer ``coupez-lui
la t\^ete~!!''\ \`a la moindre provocation.\par
\noindent\llap{\hbox to \descindent{\bf Le chat de Cheshire\hfil}}%
Un chat avec un \'enorme sourire qu'Alice trouve
dans un arbre.\par
\noindent\llap{\hbox to \descindent{\bf La tortue Mock\hfil}}%
une cr\'eature larmoyante, un peu menteuse, qui \'etait un
compagnon du Gryphon.  R\'eput\'e \^etre l'ingr\'edient principal
de la soupe de tortue Mock.
\par}
\bigskip\hrule\bigskip % A line with vertical space around it.
Voici un exemple de mots dans une bo\^\i te trac\'ee, comme
Lewis Carroll l'a \'ecrite~:
\bigskip
% Put 8pt of space between the text and the surrounding rules.
\hbox{\vrule\vbox{\hrule
   \hbox spread 8pt{\hfil\vbox spread 8pt{\vfil
      \hbox{Who would not give all else for twop}%
      \hbox{ennyworth only of Beautiful Soup?}%
   \vfil}\hfil}
\hrule}\vrule}%

\bigskip\line{\hfil\hbox to 3in{\leaders\hbox{ * }\hfil}\hfil}
\bigskip

\line{\hskip -4pt\vrule\hfil\vbox{
Ici, nous obtenons l'effet d'une barre de r\'evision sur le texte
de ce paragraphe. La barre de r\'evision indique un changement.}}
\bye % end the document
:::
\xmpheader !xmpnum/{Odds and ends}% !xmpheaddef
~^^{c\'esure} ^^{th\'eor\`emes} ^^{lemmes} ^^{listes d'\'el\'ements}
~^^{justification \`a gauche} ^^{justification \`a droite} ^^{centrage}
\chardef \\ = `\\ % Let \\ denote a backslash.
\footline{\hfil{\tenit - \folio -}\hfil}
~\global\footline{\hfil{\tenit - \folio\ -}\hfil}
% \footline provides a footer line.
% Here it's a centered, italicized page number.
\TeX\ sait couper les mots, mais n'est pas infaillible.
Si vous discutez de l'ingr\'edient chimique
${\it 5}$-[p-(Flouro\-sul\-fonyl)ben\-zoyl]-l,%
$N^6$-ethe\-no\-adeno\-sine
et que \TeX\ se plaint aupr\`es de vous d'un ``overfull hbox'', essayez
d'ins\'erer des ``c\'esures discr\'etionnaires''. La notation
`{\tt \\-}' indique \`a \TeX\ les c\'esures  dis\-cr\'e\-tionnaire,
qui sont celles qui n'auraient pas \'et\'e ins\'er\'es autrement.
\medskip
{\raggedright   Vous pouvez composer un texte non justifi\'e, c'est-\`a-dire, 
avec une marge droite non align\'ee. Comme utrefois, avant que les traitements 
de texte soient communs, les document tap\'es \`a la machine parce qu'il n'y 
avait pas de solution pratique.
Certaines personnes pr\'ef\`erent que le texte ne soit pas justifi\'e
parce que l'espacement entre les mots est uniforme. La plupart des
livres sont fait avec des marges justifi\'ees, mais pas tous. \par}

\proclaim Assertion 27. Il y a un moyen simple de composer
les en-t\^etes d'asser\-tions, lemmes, th\'eor\`emes, etc.

Voici un exemple de comment composer une liste d'items \`a deux 
niveaux de profondeur.  Si vous avez besoin de niveaux suppl\'ementaires,
vous devrez les programmer vous-m\^eme, h\'elas.
\smallskip
\item {1.} Voici le premier item.
\item {2.} Voici le second item.  Il est constitu\'e de deux
paragraphes.  Nous avons indent\'e le second paragraphe pour que
vous puissiez voir facilement o\`u il commence.

\item{} \indent Le second paragraphe a trois sous-items
sous lui.
\itemitem {(a)} Voici le premier sous-item.
\itemitem {(b)} Voici le second sous-item.
\itemitem {(c)} Voici le troisi\`eme sous-item.
\item {$\bullet$} Ceci est un \'etrange item parce qu'il est
compl\`etement diff\'erent des autres.
\smallskip
\leftline{Voici une ligne justifi\'ee \`a gauche.$\Leftarrow$}
\rightline{$\Rightarrow$Voici une ligne justifi\'ee \`a droite.}
\centerline{$\Rightarrow$Voici une ligne centr\'ee.$\Leftarrow$}
% Don't try to use these commands within a paragraph.
\bye % end the document
:::
\xmpheader !xmpnum/{Utiliser des polices venant d'ailleurs}% !xmpheaddef
~\xrdef{palatino}
~\idxref{police Palatino}
~\idxref{Zapf, Hermann}
~\idxref{police Computer Modern} ^^{\Metafont}
\font\tenrm = pplr % Palatino
%\font\tenrm = pnss10 % lucida
% Define a macro for invoking Palatino.
\def\pal{\let\rm = \tenrm \baselineskip=12.5pt \rm}
\pal % Use Palatino from now on.

Vous n'\^etes pas restreint \`a la police Computer Modern qui est 
fournie avec \TeX. D'autres polices sont possibles provenant d'autres 
sources et vous pouvez les pr\'ef\'erer. Par exemple, nous avons 
compos\'e cette page en Palatino Romain 10 points. Palatino a \'et\'e 
dessin\'ee par Hermann Zapf, consid\'er\'e comme un des plus grand 
typographe du vingti\`eme si\`ecle. Cette page vous donnera une id\'ee 
de ce \`a quoi elle ressemble.

Les polices peuvent provenir soit sous forme vectorielles soit de 
bitmaps. Une police vectorielle d\'ecrit le dessin des caract\`eres, 
tandis qu'une bitmap sp\'ecifie chaque pixel (point) qui forme chaque 
caract\`ere. Une police vectorielle peut \^etre utilis\'ee pour 
g\'en\'erer diff\'e\-rentes tailles de la m\^eme police. Le programme 
Metafont qui est associ\'e \`a \TeX\ procure un moyen 
particuli\`erement puissant de g\'en\'erer des polices bitmap, mais ce 
n'est pas le seul moyen.

Le fait qu'une seule routine puisse g\'en\'erer un grand \'eventail de 
tailles de point pour une police tente beaucoup de vendeurs de polices 
digitales de ne produire qu'une taille de vectorielles pour une police 
comme Palatino Romain. Cela semble une d\'ecision \'economiquement 
sensible, mais c'est un sacrifice esth\'etique. Une police ne peut pas 
\^etre agrandie et r\'etr\'ecie lin\'eairement sans perdre en 
qualit\'e. Les grandes lettres ne doivent pas, en g\'en\'eral, avoir 
les m\^emes proportions que les plus petites. Cela ne rend simplement 
pas beau. Par exemple, une police qui est diminu\'ee lin\'eairement 
aura trop peu d'espace entre ses lettres et la hauteur des minuscules 
sera trop petite.
%For example, a font that's linearly scaled down will
%tend to have too little space between strokes, and its
%x-height will be too~small. % tie added to avoid widow word

Un typographe peut compenser ces changements en procurant 
diff\'e\-rentes vectorielles pour diff\'erentes tailles de point, mais 
il est n\'ecessaire de d\'epenser du temps \`a dessiner ces 
diff\'erentes vectorielles. Un des grands avantages de Metafont est 
qu'il est possible de param\'etrer les descriptions des caract\`eres 
dans une police. Metafont peut alors maintenir la qualit\'e 
typographique des caract\`eres \`a travers une \'echelle de taille de 
point en ajustant le dessin des caract\`eres en cons\'equence.
\bye % end the document
:::
~\idxref{champignons}
\xmpheader !xmpnum/{Un tableau trac\'e}% !xmpheaddef
\bigskip
\offinterlineskip % So the vertical rules are connected.
% \tablerule constructs a thin rule across the table.
\def\tablerule{\noalign{\hrule}}
% \tableskip creates 9pt of space between entries.
\def\tableskip{\omit&height 9pt&&&\omit\cr}
% & separates templates for each column. TeX substitutes
% the text of the entries for #. We must have a strut
% present in every row of the table; otherwise, the boxes
% won't butt together properly, and the rules won't join.
\halign{\tabskip = .7em plus 1em  % glue between columns
% Use \vtop for short multiline entries in the first column.
% Typeset the lines ragged right, without hyphenation.
   \vtop{\hsize=6pc\pretolerance = 10000\hbadness = 10000
      \normalbaselines\noindent\it#\strut}%
  &\vrule #&#\hfil &\vrule #% the rules and middle column
% Use \vtop to get whole paragraphs in the last column.
  &\vtop{\hsize=11pc \parindent=0pt \normalbaselineskip=12pt
    \normalbaselines \rightskip=3pt plus2em #}\cr
% The table rows begin here.
\noalign{\hrule height2pt depth2pt \vskip3pt}
  % The header row spans all the columns.
  \multispan5\bf Quelques Champignons Remarquables\hfil\strut\cr
\noalign{\vskip3pt} \tablerule
  \omit&height 3pt&\omit&&\omit\cr
  \bf Nom&&\bf Nom &&\omit \bf Caract\`eristiques \hfil\cr
\noalign{\vskip -2pt}% close up lines of heading
  \bf Botanique&&\bf Commun&&\omit \bf d'Identification \hfil\cr
\tableskip Pleurotus ostreatus&&Pleurote en forme d'huitre&&
  Poussent \'etag\'es les uns sur les autres sur le bois mort ou affaibli,
  % without the kern, the `f' and `l' would be too close
  chapeau en forme de coquille d'huitre gris-rose, pied court ou absent.\cr
\tableskip Lactarius hygrophoroides&&Lactaire hygrophore&&
  Chapeau et pied de couleur orange rouille, abondance de lait,
  souvent sur la terre des bois pr\`es des ruisseaux.\cr
\tableskip Morchella esculenta&&Morille commune&&Chapeau conique
  avec des trous noirs et les ar\^etes blanches; sans lamelles. Souvent
  pr\`es de vieux pommiers et des ormes morts au printemps.\cr
\tableskip Boletus edulus&&C\`epe de Bordeaux&&Chapeau rouge-brun \`a
  caf\'e au lait avec des pores jaunes (blanc quand il est jeune),
  pied bulbeux, souvent pr\`es des conif\`eres, bouleaux ou~trembles.\cr
\tableskip \tablerule \noalign{\vskip 2pt} \tablerule
}\bye
:::
\xmpheader !xmpnum/{Composer des math\'ematiques}% !xmpheaddef
~^^{math\'ematiques}
Pour un triangle sph\'erique de cot\'es $a$, $b$ et $c$ et 
d'angles oppos\'es $\alpha$, $\beta$ et $\gamma$, nous avons~:
$$\cos \alpha = -\cos \beta \cos \gamma +
  \sin \beta \sin \gamma \cos \alpha \quad
  \hbox{(Loi de Cosinus)}$$
et~:
$$\tan {\alpha \over 2} = \sqrt{
  {- \cos \sigma \cdot \cos(\sigma - \alpha)} \over
  {\cos (\sigma - \beta) \cdot \cos (\sigma - \gamma)}},\quad
  \hbox{o\`u $\sigma = {1 \over 2}(a+b+c)$}$$
Nous avons aussi~:$$\sin x = {{e^{ix}-e^{-ix}}\over 2i}$$
et~:
$$\int _0 ^\infty {{\sin ax \sin bx}\over{x^2}}\,dx
% The \, above produces a thin space
  = {\pi a\over 2}, \quad \hbox{si $a < b$}$$

\noindent Le nombre de combinaisons ${}_nC_r$ de $n$
choses prises parmi $r$ est~:
$$C(n,r) = {}_nC_r = {n \choose r} =
  {{n(n-1) \cdots (n-r+1)} \over {r(r-1) \cdots (1)}} =
  {{n!!}\over {r!!(n-r)!!}}$$

\noindent
La valeur du d\'eterminant $D$ d'ordre $n$~:
$$D = \left|\matrix{a_{11}&a_{12}&\ldots&a_{1n}\cr
  a_{21}&a_{22}&\ldots&a_{2n}\cr
  \vdots&\vdots&\ddots&\vdots\cr
  a_{n1}&a_{n2}&\ldots&a_{nn}\cr}\right| $$
est d\'efinie comme la somme de $n!!$ termes~:
$$\sum\>(\pm)\>a_{1i}a_{2j} \ldots a_{nk}$$
% The \> above produces a medium space.
o\`u $i$, $j$, \dots,~$k$\/ prennent toutes les valeurs possibles
entre $1$ et $n$ et pour  lequel, le signe du produit est 
$+$ si la s\'equence $i$, $j$, \dots,~$k$\/ est une
permutation paire et $-$ autrement.  De plus~:
$$Q(\xi) = \lambda_1 y_1^2 \sum_{i=2}^n \sum_{j=2}^n y_i
b_{ij} y_j,\qquad B = \Vert b_{ij} \Vert = B'$$
\bye
:::
\xmpheader !xmpnum/{Plus de math\'ematiques}% !xmpheaddef
~^^{math\'ematiques}
La valeur absolue de $X$, $|x|$, est d\'efinie par~:
$$|x| = \cases{x, &si $x\ge 0$;\cr
-x,&autrement.\cr}$$
Maintenant quelques \'equations num\'erot\'ees.
Il arrive que pour $k \ge 0$~:
$$x^{k^2}=\overbrace{x\>x\>\cdots\> x}^{2k\ \rm fois}
   \eqno (1)$$

Voici un exemple qui montre certains contr\^oles d'espacement, avec 
un num\'ero  \`a gauche~:
$$[u]!negthin[v][w]\,[x]\>[y]\;[z]\leqno(2a)$$
Le montant d'espace pour les items entre crochets
augmente graduellement de gauche \`a droite.  Nous avons fait
l'espace entre les deux premiers items plus {\it petit\/}
que l'espace naturel.
Il arrive parfois que $$\leqalignno{
u'_1 + tu''_2 &= u'_2 + tu''_1&(2b)\cr
\hat\imath &\ne \hat \jmath&(2c)\cr
\vec {\vphantom{b}a}&\approx \vec b\cr}$$
% The \vphantom is an invisible rule as tall as a `b'.
Le r\'esultat est d'ordre $O(n \log\log n)$.  D'o\`u
$$\sum_{i=1}^n x_i = x_1+x_2+\cdots+x_n
= {\rm Sum}(x_1,x_2,\ldots,x_n). \eqno(3)$$
et
$$dx\,dy = r\,dr\,d\theta!negthin.\eqno(4)$$
Le jeu de tous les $q$ tels que $q\le0$ s'\'ecrit~:
$$\{\,q\mid q\le0\, \}$$
Donc
$$\forall x\exists y\;P(x,y)\Rightarrow
\exists x\exists y\;P(x,y)$$
o\`u
$$P(x,y) \buildrel \rm def \over \equiv 
\hbox{\rm tout pr\'edicat en $x$ et $y$} . $$
\bye
:::
